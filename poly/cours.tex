\documentclass[]{article}
\usepackage{lmodern}
\usepackage{amssymb,amsmath}
\usepackage{ifxetex,ifluatex}
\usepackage{fixltx2e} % provides \textsubscript
\ifnum 0\ifxetex 1\fi\ifluatex 1\fi=0 % if pdftex
  \usepackage[T1]{fontenc}
  \usepackage[utf8]{inputenc}
\else % if luatex or xelatex
  \ifxetex
    \usepackage{mathspec}
    \usepackage{xltxtra,xunicode}
  \else
    \usepackage{fontspec}
  \fi
  \defaultfontfeatures{Mapping=tex-text,Scale=MatchLowercase}
  \newcommand{\euro}{€}
\fi
% use upquote if available, for straight quotes in verbatim environments
\IfFileExists{upquote.sty}{\usepackage{upquote}}{}
% use microtype if available
\IfFileExists{microtype.sty}{\usepackage{microtype}}{}
\usepackage{longtable,booktabs}
\ifxetex
  \usepackage[setpagesize=false, % page size defined by xetex
              unicode=false, % unicode breaks when used with xetex
              xetex]{hyperref}
\else
  \usepackage[unicode=true]{hyperref}
\fi
\hypersetup{breaklinks=true,
            bookmarks=true,
            pdfauthor={},
            pdftitle={},
            colorlinks=true,
            citecolor=blue,
            urlcolor=blue,
            linkcolor=magenta,
            pdfborder={0 0 0}}
\urlstyle{same}  % don't use monospace font for urls
\setlength{\parindent}{0pt}
\setlength{\parskip}{6pt plus 2pt minus 1pt}
\setlength{\emergencystretch}{3em}  % prevent overfull lines
\setcounter{secnumdepth}{0}


\begin{document}

\textbf{Signal \& Filtrage}

\textbf{}\\

2 IMACS

2019 - 2020

alexandre.boyer@insa-toulouse.fr

page Moodle

\textbf{Contenu}

A. Introduction 5

I. Le signal 5

1. Classification des signaux 5

2. Temps continu et temps discret 5

II. Le traitement du signal 5

III. Les systèmes linéaires à temps invariant - Réponse d'un système à
un signal 6

IV. Les outils 7

V. Les objectifs pédagogiques 7

B. L'étude des systèmes linéaires à temps invariant 9

I. Définition d'un système LTI 9

1. Linéarité 9

2. Invariance temporelle 9

3. Passivité 10

4. Causalité 10

II. Equation générale d'un système linéaire 10

III. Représentation 10

IV. Réponses d'un système 11

V. Réponse naturelle d'un système LTI 11

1. Fréquences naturelles d'un système LTI 12

2. Ordre d'un système linéaire 14

VI. Réponse forcée 14

1. Excitation exponentielle complexe 14

2. Familles d'excitations impulsionnelles 15

VII. Les différents types de réponse caractérisant les systèmes LTI 17

1. Réponse indicielle 17

2. Réponse impulsionnelle 17

3. Réponse à une exponentielle complexe - fonction de transfert 20

VIII. Ce qu'il faut retenir 22

IX. Exercices 22

C. Transformée de Laplace 23

I. Dérivation d'une transformation d'une fonction temporelle du domaine
temporel 23

II. Transformée de Laplace 24

1. Définition 24

2. Conditions d'existence de la transformée de Laplace et stabilité des
systèmes 25

3. Pôles et zéros d'une fonction de transfert 26

III. Propriétés de la transformée de Laplace 26

1. Linéarité 27

2. Théorème du retard 27

3. Théorème du changement d'échelle 27

4. Dérivation dans le domaine temporel 27

5. Dérivation dans le domaine des fréquences complexes p 28

6. Intégration temporelle 28

7. Théorème de la valeur initiale 28

8. Théorème de la valeur finale 28

9. Multiplication et produit de convolution 28

10. Fonctions périodiques causales 28

11. Fonctions causales avec N répétitions 28

12. Tableau récapitulatif des propriétés 28

IV. Transformée de Laplace des fonctions courantes 28

V. Transformée de Laplace inverse 30

1. Méthode 30

2. Décomposition pôles-résidus 30

3. Exemple 31

VI. Applications 31

1. Résolution d'équations différentielles ordinaires 31

2. Calcul de la réponse transitoire d'un système linéaire 32

VII. Ce qu'il faut retenir 32

VIII. Exercices 32

D. Filtrage 34

I. Définition d'un filtre 34

1. Filtre à réponse impulsionnelle réelle 35

II. Analyse harmonique 35

III. Représentation - Diagramme de Bode - Gabarit 36

2. Diagramme de Bode 36

3. Exemples 37

4. Gabarit 39

IV. Les principales caractéristiques d'un filtre 40

V. Les différents types de filtres 41

VI. Méthodes simples d'analyse de fonction de transfert 42

VII. Aspects énergétiques 42

VIII. Ce qu'il faut retenir 42

IX. Exercices 43

E. Série de Fourier 45

F. Transformée de Fourier 46

G. Analyse des systèmes et des signaux dans le domaine temporel 47

I. Réponse impulsionnelle 47

II. Produit de convolution 48

1. Définition 48

2. Propriétés 48

3. Mise en œuvre du produit de convolution 48

III. Corrélation 50

1. Définition 50

2. Propriétés 52

3. Exemple 53

IV. Relation de Wiener-Khintchine 53

1. Densité spectrale de puissance 53

2. Théorème de Wiener-Khintchine 53

3. Application pour les filtres linéaires 54

V. Ce qu'il faut retenir 54

VI. Exercices 54

H. Récapitulatif 56

I. Annexe 1 - Nombres complexes 57

I. Propriétés de base 57

II. Représentation géométrique 57

III. Représentation des signaux (co)sinusoïdaux - Phaseur 58

J. Annexe 2 - Distribution de Dirac 59

Introduction

un positionnement de ce cours, définition du périmètre.

On pourra encadrer les choses à retenir !

Le signal

Fonction d'une ou plusieurs variables qui véhicule de l'information à
propos d'un phénomène physique.

Classification des signaux~

réel/complexe, déterministe/aléatoires, temps continu/discret,
périodique/apériodique, à valeurs continues/discrètes, mono ou
multidimensionnels.

Selon sa nature, le phénomène physique, dont le signal constitue la
signature, peut être exprimé par une fonction mathématique évoluant en
fonction d'une ou plusieurs variables. Les plus courantes sont des
variables spatio-temporelles. Dans ce cours, nous traiterons
principalement de signaux temporels, c'est-à-dire que leur évolution ne
dépend que du temps. Nous pourrions traiter de la même manière des
signaux dépendant d'une ou plusieurs variables spatiales, auxquelles on
pourrait aussi ajouter le temps.

Considérons que le signal dépend du temps. Nativement, il est exprimé
dans le domaine temporel, c'est-à-dire que nous pouvons décrire son
évolution par une fonction dépendante du temps.~ Nous verrons dans ce
cours qu'il est possible, via des transformations mathématiques,
l'exprimer dans un domaine dual, facilitant ainsi son étude ou l'étude
du système qu'il attaque ou dont il est issu.

Quelques exemples : \ldots{}

Nous nous focaliserons aussi uniquement sur des signaux déterministes,
des signaux à une dimension, et généralement exprimés en fonction du
temps. Bien que les concepts que nous verrons s'adaptent à tous les
types de signaux, nous ne considérerons que des signaux à une dimension,
dépendant du temps par souci de simplification.

Temps continu et temps discret

Nous nous focaliserons uniquement sur les signaux à temps continu. En
effet, les concepts traités cette année dans le cadre des signaux
continus formeront la base pour l'étude des signaux discrets,
indispensables pour traiter des fonctions numériques.

Le traitement du signal

Qu'appelle t-on le traitement du signal ?~

analyser et transformer les signaux en d'autres signaux ou paramètres
pour en extraire de l'information.

C'est une discipline en soit, mais elle forme une base nécessaire pour
de nombreuses disciplines et domaines d'ingénierie, tels que
l'électronique, l'automatique, les télécommunications, la
thermodynamique, \ldots{}.

Quelles sont ses finalités :

analyse (extraction des composantes essentielles portant l'information
utile)

mesure (estimation d'une grandeur caractéristique)

détection

filtrage (élimination des composantes ``parasites'' d'un signal

restauration

synthèse

codage

compression

reconnaissance

modulation

Cette discipline est en lien avec les mathématiques mais aussi avec les
moyens de calcul et de traitement numérique.

Les aspects signaux discrets n'étant pas traités dans ce cours, cette
partie traitement numérique ne pourra pas être abordé. Ce cours se
focaliseront donc sur la présentation des outils mathématiques de base
pour le traitement du signal, et l'illustration sur des exemples
pratiques. Cependant, une mise en œuvre concrète sur des cibles réelles
ne sera possible qu'une fois le traitement numérique abordé.

Les concepts seront néanmoins mis en œuvre à travers des outils de
simulation numérique (Matlab, Octave On Line).~

Les systèmes linéaires à temps invariant - Réponse d'un système à un
signal

Les concepts vus dans ce cours en traitement du signal ont aussi pour
but l'étude des systèmes, c'est-à-dire la manière dont ils répondent à
un signal d'entrée. Un système peut être vu comme une ``boite'' attaquée
sur son/ses entrées par un ou plusieurs signaux (excitations) et dont
les sorties dépendront du signal d'entrée et des propriétés du système
(excitation). Dans ce cours, nous ne considérerons que des systèmes à
une entrée et une sortie.~

On appelle système la modélisation mathématique d'un processus physique
qui produit un signal de sortie (la réponse du système) en réponse à un
signal d'entrée (la source ou excitation du système).

Soient respectivement x(t) et y(t) les signaux d'entrée et de sortie
d'un système. On dit que le système effectue une transformation de x(t)
en y(t) et on représente cela de façon symbolique par l'opérateur L.

Dans ce cours, nous traiterons d'une classe de systèmes particuliers,
appelés systèmes linéaires à temps invariants (LTI).~ Les signaux
d'entrée x(t) et de sortie y(t) seront reliés entre eux via un ou
plusieurs opérateurs linéaires notés L, dont les caractéristiques
n'évoluent pas au cours du temps. Si elles sont connues, il devient
alors possible de prédire la réponse en sortie du système quelle que
soit l'excitation appliquée en entrée. Pour cela, il nous faut
déterminer une représentation mathématique de l'effet de ce système,
quelle que soit la nature réelle du système. Celui-ci est vu comme une
boîte noire.

Dans ce cours, L'étude des systèmes~ prédire leur réponse à une
excitation donnée, ou retrouver le signal d'entrée à partir de la
sortie.~

Nous présenterons dans ce cours des concepts et des outils mathématiques
qui permettront l'étude des signaux et des systèmes linéaires (leur
réponse), indépendamment de leur nature. Ils s'appliqueront aussi bien à
des systèmes électroniques, mécaniques ou ???, tant que ceux-ci sont
linéaires. Ce sera la même chose pour les signaux. Quelle que soit la
nature des signaux, tant que ceux-ci seront continus, les concepts que
nous présenterons resteront valables.

Les outils

L'analyse des systèmes et le traitement du signal met en œuvre des
calculs potentiellement complexes, ne pouvant être pas résolus
efficacement à la main. Des outils numériques sont donc requis.

Dans ce cours, nous travaillerons avec Matlab, mais aussi Octave On
Line, pour valider les concepts et valider les calculs effectués.

Les objectifs pédagogiques

A l'issue du cours, vous devrez maîtriser :

\ldots{}

L'étude des systèmes linéaires à temps invariant

Après une définition, nous définirons les concepts requis pour répondre
à la question suivante :

comment déterminer efficacement la réponse d'un système linéaire
lorsqu'il est soumis à une excitation quelconque ? Par efficacement,
nous entendons :

méthode mathématiquement ``simple''

méthode indépendante de l'excitation et des propriétés du système LTI

Pour cela, nous allons commencer par nous interroger sur la manière de
représenter l'interaction entre l'entrée et la sortie d'un système, puis
sur les notions d'excitation et de réponse. Nous distinguerons les
notions de réponses naturelles et forcées, avant d'identifier deux
familles d'excitation adaptées à l'étude des systèmes. Nous introduirons
ensuite la notion de fréquence ainsi que la fonction de transfert qui
facilitera l'étude des systèmes.

\begin{itemize}
\itemsep1pt\parskip0pt\parsep0pt
\item
  Définition d'un système LTI
\end{itemize}

\begin{itemize}
\itemsep1pt\parskip0pt\parsep0pt
\item
  Linéarité
\end{itemize}

Qu'est-ce qu'un système linéaire à temps invariant ?~

y(t) = L{[}x(t){]} : comment vérifie t-on que L est un opérateur
linéaire.~

Soit y1 la réponse du système à une excitation donnée x1. Soit y2 la
réponse du système à une excitation donnée x2. Si le système est
linéaire, alors la réponse y à une excitation x correspondant à la somme
pondérée des excitations x1 et x2 sera la somme des réponses y1 et y2.

Une des conséquences de la linéarité est la possibilité d'appliquer le
principe de superposition.

Les trois opérateurs linéaires de base :

proportionnel

intégration

dérivée

Invariance temporelle

ou stationnarité, si la propriété suivante est respectée :

En d'autres termes, la réponse du système ne dépend pas de l'origine des
temps choisis.

Passivité

On considèrera aussi des systèmes passifs, dans le sens où il n'y a pas
d'énergie emmagasinée ou fournie par une autre entrée que celle sur
laquelle on applique le signal d'entrée étudié.

Causalité

Une dernière caractéristique importante qui permet de différencier les
systèmes physiquement réalisables de ceux qui ne le sont pas est la
notion de causalité. On entend par physiquement réalisable un système
pouvant fonctionner en temps réel, c'est-à-dire qu'il traite en temps
réel, sans traitement différé, les signaux d'entrée.

L'étude des systèmes linéaires passent par l'étude et la synthèse de
fonctions mathématiques. Tous les systèmes linéaires physiquement
réalisables peuvent être décrits par une fonction mathématique linéaire.
Par contre, l'inverse n'est pas vraie : un système décrit par une
fonction linéaire n'est pas nécessairement physiquement réalisable.

~Une des caractéristiques fondamentales des systèmes causaux est que
l'effet ne peut pas précéder la cause. Un système sera physiquement
réalisable s'il est causal, c'est-à-dire que sa réponse d'un système ne
dépend que des états présents et passés de l'excitation.

Les systèmes numériques peuvent être non causaux car ils sont capables
de stocker les valeurs du signal d'entrée pour un traitement différé. Ce
n'est pas le cas pour des systèmes dits analogiques.

Exemple d'un système non causal : enregistrement d'un son, puis
application d'un traitement dépendant d'échantillons futurs. Défloutage
d'image.

Equation générale d'un système linéaire

La réponse d'un système linéaire sera gouvernée par 3 type d'effet,
pouvant se superposer :

proportionnel : y = a*x

différentiateur : y = a*dx/dy

intégrateur : y = a*integ(x dt)~

Globalement, un système linéaire est gouverné par une équation
différentielle ordinaire du type :~

Représentation~

Un système sera considéré comme une boite noire (dans le sens où nous ne
connaissons pas nécessairement le contenu), mais nous disposons d'un
modèle mathématique représentant les interactions entre les sorties et
les entrées du système.~

On peut le représenter directement par une équation, ou sous une autre
représentation (schéma-bloc, représentation d'état ..)

Réponses d'un système

En reprenant l'équation générale, on peut voir que l'on peut distinguer
la réponse propre ou naturelle d'un système, notée
y\textsubscript{o}(t), de la réponse forcée y\textsubscript{f}(t). La
réponse du système sera la superposition de ces deux réponses (hormis si
le système n'est pas excité, auquel cas il n'y aura pas de réponse
forcée).

Nous nous intéressons à des systèmes causaux, donc nous considérerons
que les excitations sont nulles pour t \textgreater{} 0.

Réponse naturelle d'un système LTI

Il s'agit de la partie de la réponse indépendante de l'excitation. Elle
est donc intrinsèque aux caractéristiques du système.

Elle se détermine en résolvant l'équation différentielle suivante, cas
où l'excitation x(t) est nulle.

On pourrait s'interroger sur le sens de cette équation. Si le système
n'avait pas emmagasiné d'énergie au départ, la réponse serait nulle.
Comme il n'y a pas de source, la réponse sera transitoire.

Seule une fonction présentant la même forme avant/après dérivation
plusieurs fois peut être solution de cette équation. Il n'existe qu'une
seule forme de fonction avec une telle propriété : la forme
exponentielle complexe : yo(t) = A*exp(p*t), où A et p peuvent être
complexes.

La forme générale de la réponse naturelle sera donc :

Hormis la solution triviale yo(t) = 0, la solution est :

Ce polynôme est appelée l'équation caractéristique du système. En effet,
les caractéristiques de la réponse naturelle du système sont liées aux
solutions de cette équation. Elle présentent M solutions ou racines.~

La réponse naturelle sera donc du type :

où les termes Ai dépendent des conditions initiales connues en un temps
t\textsubscript{0}.~

\begin{itemize}
\itemsep1pt\parskip0pt\parsep0pt
\item
  Fréquences naturelles d'un système LTI
\end{itemize}

Ces n racines s'appellent les fréquences naturelles ou propres. Elles
sont complexes et de la forme :

La réponse naturelle pourra alors s'exprimer :

\textsubscript{i} caractérise la pulsation de la réponse associée à
cette racine, tandis que la partie réelle σi indique l'atténuation ou
l'amortissement de cette réponse. Leur connaissance est indispensable
car ce sont elles qui caractérisent la forme temporelle de la réponse du
système. La compréhension de l'évolution temporelle va passer par
l'analyse de ces racines.~

Plan p

Puisque les racines de l'équation caractéristique déterminent la réponse
naturelle d'un système, il est intéressant d'établir un système de
représentation graphique de ces racines pour analyser un système. C'est
le but de la représentation appelée plan complexe appelé plan P p. Il
s'agit d'un repère cartésien (quel nom pour ce plan Re-Im) dans lequel
les racines sont placées, permettant de visualiser leurs parties réelles
et imaginaires. Selon le positionnement des racines dans le plan P, les
caractéristiques du système seront différentes, notamment leur
stabilité.

Analyse du plan p - Stabilité du système

La stabilité d'un système est une notion qui sera abordée plus en détail
dans les cours d'automatique. Dans ce cours, nous utiliserons une
définition au sens large : on entend par stabilité le fait que la
réponse d'un système converge vers une valeur finie quelle que soit
l'excitation appliquée (à condition que cette excitation converge). En
ingénierie, c'est une propriété indispensable pour la conception de
système.

Un système est stable si et ssi à toute entrée bornée, il fait
correspondre une sortie bornée :

Remarque : un système oscillant vérife cette propriété, mais est à la
limite de la stabilité.

La position des racines de l'équation caractéristique détermine à la
fois la stabilité du système linéaire, mais aussi la forme temporelle de
réponse. Pour l'admettre, considérons les quatre cas suivants,
correspondant à cinq positions différentes d'une racine
p\textsubscript{i} dans le plan P, et déterminons le type de réponse
temporelle. On considère t \textgreater{} 0 :

p\textsubscript{i} est purement complexe (σ\textsubscript{i} = 0) : elle
est située sur l'axe des ordonnées du plan P. La réponse~ naturelle sera
donc un signal purement (co)sinusoïdale, dont l'oscillation non aura une
fréquence ou une pulsation de l'oscillation déterminée par
ω\textsubscript{i}. Le système est en limite de stabilité car sa réponse
oscille en permanence autour d'une valeur.

pi est purement réel et négatif (σ\textsubscript{i} \textless{} 0 et
ω\textsubscript{i} = 0) : la racine est située sur l'axe des abscisses à
gauche de l'origine. La réponse est une fonction exponentielle
décroissante, sans la moindre oscillation, indiquant un complètement
amorti. La réponse naturelle indique un système parfaitement stable.

pi est purement réel et positif (σ\textsubscript{i} \textgreater{} 0 et
ω\textsubscript{i} = 0) : la racine est située sur l'axe des abscisses à
droite de l'origine. La réponse est une fonction exponentielle
croissante, sans la moindre oscillation, indiquant un complètement
divergeant. La réponse naturelle indique un système instable.

pi est une valeur complexe quelconque dont la partie réelle est négative
(σ\textsubscript{i} \textless{} 0 et ω\textsubscript{i} ≠ 0) : la racine
est située dans le demi plan à gauche de l'axe es ordonnées. La réponse
va présenter une oscillation dont la fréquence est déterminée par ωi,
mais dans l'amplitude s'atténue plus ou moins rapidement au cours du
temps selon la valeur σ\textsubscript{i}. Cette réponse indique un
système stable.

pi est une valeur complexe quelconque dont la partie réelle est positive
(σ\textsubscript{i} \textgreater{} 0 et ω\textsubscript{i} ≠ 0) : la
racine est située dans le demi plan à droite de l'axe es ordonnées. La
réponse va présenter une oscillation dont la fréquence est déterminée
par ωi, mais dans l'amplitude s'accroit plus ou moins rapidement au
cours du temps selon la valeur σ\textsubscript{i}. Cette réponse indique
un système instable.

si σi \textless{} 0 : si sigma \textless{} w : oscillation sous amorti
\ldots{}~ pas forcément utile de rentrer dans ce genre de détail dans ce
cours. Ils verront ça en ecours d'automatique

La figure ci-dessous illustre les types de réponse naturelle associée à
une fréquence naturelle, selon sa position dans le plan P.

Remarque : domaine fréquentiel / domaine dual

On remarque que l'on est capable de représenter avec un point dans le
plan p une fonction temporelle. Il s'agit de la même fonction, mais vue
dans un autre domaine, que l'on appelle domaine fréquentiel. De manière
générale, les fréquences sont des grandeurs complexes. Néanmoins, dans
les cas pratiques que nous allons rencontrer pour analyser les systèmes
LTI, nous ne considérerons que des fréquences réelles.

Le domaine fréquentiel est le domaine dit dual du domaine temporel.

Ordre d'un système linéaire

Plus l'équation caractéristique présente de racines, plus sa réponse
devient complexe. Elle est en effet la superposition de plusieurs
fréquences complexes comme le montre l'équation xxx. On appelle l'ordre
d'un système le nombre de racines que présente son équation
caractéristique. Il s'agit aussi de l'ordre ou du degré de l'équation
différentielle caractérisant le système. Pour illustrer la notion
d'ordre, nous allons analyser la réponse naturelle de deux systèmes
électriques d'ordre 1 et 2.

Exemple système ordre 1/ordre 2

Circuits RC et RLC

Réponse forcée

Cette fois, nous considérons l'effet d'une excitation non nulle en
entrée du système, dont la réponse est appelée réponse forcée. La
réponse va dépendre de la forme de l'excitation, mais aussi des
propriétés du système. Puisque la réponse dépend de l'excitation, il est
préférable de déterminer des excitations dont les propriétés
faciliteront l'analyse de la réponse forcée d'un système. Pour cela, il
faut chercher des familles d'excitation dont la forme n'est pas modifiée
par l'effet du système LTI. Ainsi, en connaissant la forme de
l'excitation, on connaitra immédiatement la forme de la réponse. Seule
quelques coefficients (que nous préciserons) seront à calculer.

Nous allons nous intéresser à deux familles d'excitations : l'excitation
exponentielle complexe (que nous avons rencontré précédemment) et la
famille impulsionnelle.

\begin{itemize}
\itemsep1pt\parskip0pt\parsep0pt
\item
  Excitation exponentielle complexe
\end{itemize}

L'excitation exponentielle complexe présente la forme suivante :

où X =~ est l'amplitude complexe du signal et p\textsubscript{x} = α+jω
sa fréquence complexe. Dans le cas où l'on s'intéresse à des signaux
réels (cas rencontré en pratique), elle peut s'écrire :

Comme dans le cas de la fréquence naturelle, la fréquence complexe peut
être représentée dans le plan P. Sa position indique qualitativement la
forme temporelle de l'excitation. Si p\textsubscript{x} est purement
réelle, l'excitation est une fonction exponentielle pure. Si
p\textsubscript{x} est purement complexe, l'excitation est une fonction
cosinusïdale pure. L'excitation est dite monochromatique. Dans le cas
général, on pourra aussi écrire :

phase θ : d'où sert-elle ?

L'excitation exponentielle complexe a une propriété remarquable :
l'action d'un système LTI laisse sa forme inchangée. Seule son amplitude
et sa localisation sont modifiées. On peut donc en déduire une propriété
des systèmes linéaires : lorsqu'un système linéaire est excité par un
signal exponentiel complexe (en pratique, un signal sinusoïdal), la
réponse sera aussi une fonction sinusoïdale de même fréquence. Seule
l'amplitude et la phase changeront.

\begin{itemize}
\itemsep1pt\parskip0pt\parsep0pt
\item
  Phaseur
\end{itemize}

Considérons le cas d'une excitation sinusoïdale pure. La grandeur
complexe représente un vecteur, appelé vecteur de Fresnel ou phaseur. Il
s'agit d'un vecteur tournant dans le plan complexe à la fréquence w. La
projection de ce phaseur sur l'axe réel donne la valeur instantanée
prise par le signal.

Familles d'excitations impulsionnelles

Nous parlons ici de familles car nous allons parler de différentes
fonctions, reliées entre elles, et associées à une forme très importante
dans l'analyse des systèmes et des signaux : l'impulsion de Dirac.

\begin{itemize}
\itemsep1pt\parskip0pt\parsep0pt
\item
  Echelon unitaire ou fonction de Heaviside
\end{itemize}

Elle présente la forme suivante : + image

Elle représente le signal que l'on obtiendrait derrière un interrupteur
idéal que l'on fermerait à t = t0.

U(t-t0) = 1 si t \textgreater{} t0, 0 sinon. Dans la suite, on utilisera
principalement t0 = 0.~

Une opération de décalage dans le temps s'écrit\ldots{}.~

Cette fonction peut être utilisée pour représenter n'importe quelle
fonction en forme de marche d'escalier, comme l'illustre la figure
ci-dessous. A partir de trois fonction de Heaviside, il est possible de
générer une forme d'onde plus complexe.

Son intégration donne la fonction rampe. Si on l'intègre n fois :

On voit que l'on garde la même fonction de base. Le cas de la dérivation
est plus complexe car elle entraine l'apparition d'une discontinuité qui
ne peut se traiter que via la théorie des distributions.

Impulsion ou distribution de Dirac

L'impulsion de Dirac se trouve en différenciant l'échelon unitaire. On
voit immédiatement apparaître un problème : la dérivée est nulle sauf là
où l'échelon présente une discontinuité. Sa dérivée devient infinie.
L'impulsion de Dirac n'est pas une fonction au sens classique (même si
abusivement on rencontre le terme fonction de Dirac), mais une
distribution. Voir annexe.

Image 2.8 p 33 du livre Y. Mori.

Approche intuitive basée sur l'impulsion rectangulaire, qui a un sens
physique. C'est juste un passage à la limite, qui permet de définir une
excitation infiniment courte.

Une propriété intéressant de la distribution de Dirac est sa propriété
d'échantillonnage :

~et~

Revenons au problème initial de sélection de familles d'excitation
inchangées par l'effet d'un système LTI. C'est bien le cas pour la
famille impulsionnelle. L'action proportionnel, intégrateur, dérivative
du système résulte toujours dans une forme impulsionnelle, multipliée
éventuellement par un polynôme en t d'ordre n.

Les différents types de réponse caractérisant les systèmes LTI

Maintenant que nous avons défini plusieurs familles d'excitation
adaptées à l'étude des systèmes LTI, nous pouvons définir différents
types de réponse qui nous aideront à analyser les propriétés de ces
systèmes.

\begin{itemize}
\itemsep1pt\parskip0pt\parsep0pt
\item
  Réponse indicielle
\end{itemize}

Elle correspond à la réponse d'un système excité par une fonction
échelon. On la note a(t)

Cas d'un système d'ordre 1

Cas d'un système d'ordre 2 (voir p 37 livre Y. Mori)

Réponse impulsionnelle

\begin{itemize}
\itemsep1pt\parskip0pt\parsep0pt
\item
  Définition
\end{itemize}

Elle correspond à la réponse d'un système excité par une impulsion de
Dirac, que l'on note h(t). Même si l'amplitude de cette impulsion est
infinie, son énergie vaut 1. Au lieu de représenter cette valeur
infinie, nous utiliserons la valeur de 1. N'oublions pas qu'il ne s'agit
pas d'une fonction mais d'une distribution, qui n'a du sens qu'à
l'intérieur d'une intégrale.

Il apparait que quelque soit le système LTI, puisque l'excitation
devient nulle pour t \textgreater{} 0, on retrouve le calcul de la
réponse naturelle. L'application de l'impulsion ne fait que changer les
conditions initiales.

yf(t) = y0(t).γ(t) = h(t)

Remarque : autre manière de déterminer la réponse impulsionnelle

Sans démonstration, on peut affirmer que si on différencie l'excitation,
on peut déterminer la nouvelle réponse en différenciant la réponse à
l'excitation initiale :

si y(t) = L(x(t)), alors y't) = L(x'(t))

Ainsi, puisque l'impulsion de Dirac est la dérivée de l'échelon
unitaire, si on connait la réponse indicielle, on peut retrouver la
réponse impulsionnelle en dérivant la réponse indicielle.

La réponse impulsionnelle va nous fournir un moyen de calculer la
réponse d'un système LTI à n'importe quelle excitation, en effectuant
les calculs uniquement dans le domaine temporel. C'est ce que nous
allons démontrer. Tout signal peut être vu comme une somme infinie
d'impulsion de Dirac, pondérée et décalée dans le temps. Comme le montre
la figure ci-dessous, une excitation quelconque x(t) peut être
approximée par une suite de rectangles adjacents, de largeur Δτ, que
l'on peut remplacer par des impulsions de Dirac équivalentes de même
surface. Ainsi, elles transportent la même énergie. Si la largeur Δτ
tend vers zéro, cette approximation devient de plus en plus juste. Dans
l'exemple considéré, on suppose que x(t) = 0 pour t \textless{} 0 par
souci de lisibilité. Le même raisonnement resterait juste avec x(t) non
nul pour t \textless{} 0. L'excitation peut alors s'approximer par :

Supposons que cette excitation attaque l'entrée d'un système LTI dont la
réponse impulsionnelle h(t) est connue (par souci de lisibilité, on
suppose aussi que h(t) = 0 pour t \textless{} 0).

Chaque impulsion élémentaire composant x(t), apparaissant à l'instant
k×Δτ, contribue à la réponse en sortie. Chacune d'entre elles produit
une réponse égale à la réponse impulsionnelle du système, mais :

pondérée par l'amplitude de l'excitation d'entrée à l'instant k×Δτ

décalée dans le temps de k×Δτ~

La réponse globale du système est obtenue en sommant l'ensemble des
contributions de chaque impulsion élémentaire formant l'excitation. Elle
peut alors s'approximer par :~

En faisant tendre Δτ vers zéro, cette somme converge vers une intégrale
donnée par la relation ci-dessous. Ce calcul intégral particulier,
faisant intervenir le produit de deux fonctions dont les indices τ et
t-τ sont balayés dans des directions opposées, porte un nom : le produit
de convolution. Cette opération est symbolisée par le signe *.~

Cette relation basée sur le produit de convolution fournit donc un outil
de calcul de la réponse du système directement dans le domaine temporel.
Cependant, c'est un calcul complexe, passant par une opération longue et
fastidieuse si elle est effectuée à la main. Nous y reviendrons dans le
chapitre xxx, qui sera consacré au calcul des réponses des systèmes LTI
directement dans le domaine temporel. Cependant, avant de s'y consacrer,
nous allons d'abord considérer le cas d'une excitation exponentielle
complexe, et nous dériverons le concept de fonction de transfert défini
dans le domaine fréquentiel. Nous verrons que dans ce domaine, le calcul
de la réponse du système est beaucoup plus aisée !

Condition pour garantir la causalité

Causalité : voir p 351 livre Kudeki : un système est causal si sa
réponse ne dépend que des états passés et présents. A partir de la
réponse impulsionnelle, on peut en déduire une condition : il faut que
h(t) = 0 pour t \textless{} 0. Les systèmes causaux sont les seuls à
être physiquement réalisables. Le calcul de la réponse d'un système
causal à partir de sa réponse impulsionnelle peut s'écrire :

Dans la suite, nous traiterons uniquement de systèmes LTI causaux.

Faire une remarque sur la causalité (voir p 357 livre Kudeki)

Condition pour garantir la stabilité

Stabilité :

Réponse à une exponentielle complexe - fonction de transfert

On considère une excitation exponentielle complexe. La réponse forcée a
donc la même forme que l'excitation, et présente la même fréquence
complexe. La seule inconnue est le phaseur Y.

x(t) = Re(X exp(px.t)) et yf(t) = Re(Y exp(px.t)) pour t \textgreater{}
0

Si on reprend l'équation différentielle ordinaire générale d'un système
LTI : (ai*dyi/dti) = somme (bj*dxj/dtj), elle va s'écrire :

somme(ai*Re(px\^{}i*Y*exp(px.t)) = somme(bj*Re(px\^{}j*X*exp(px.t))

~somme(ai*Re(px\^{}i*Y) = somme(bj*Re(px\^{}j*X) Y*somme(ai*Re(px\^{}i))
= X * somme(bj*Re(px\^{}j)~

On peut donc facilement déterminer le phaseur Y associée à la réponse,
donc le module et le déphasage de la réponse.~

On peut alors caractériser l'effet du système à une excitation
exponentielle complexe quelconque sous la forme d'une fonction appelée
fonction de transfert, défini comme le rapport entre les phaseurs de
réponse et d'excitation, et défini pour toutes les fréquences complexes
:

H(px) = Y/X =~ somme(bj*Re(px\^{}j)) / somme(ai*Re(px\^{}i))

Avec une excitation exponentielle complexe, l'action du système LTI peut
donc se résumer à la multiplication de l'excitation avec un terme appelé
fonction de transfert.

On peut souligner la facilité de la méthode. Une fois la fonction de
transfert connue à une fréquence complexe donnée, la réponse forcée à
une exponentielle complexe de même fréquence est trouvée en multipliant
le phaseur d'entrée par la fonction de transfert.

Remarque : résolution d'équation différentielle ordinaire

on voit apparaitre un autre intérêt à travailler dans le domaine
fréquentiel, que nous détaillerons dans le chapitre suivant. L'effet
d'un système LTI est décrit comme une équation différentielle (voir
équation caractéristique du système), dont la résolution donne la
réponse du système. Selon l'ordre du système, elle peut être difficile à
résoudre. Or, en considérant une excitation exponentielle complexe et en
travaillant dans le domaine fréquentiel, cette équation est transformée
en un polynôme dépendant de la fréquence. Celui se résout bien plus
facilement.

Influence des pôles et des zéros :

Comme pour la réponse naturelle, les pôles et les zéros vont influencer
la réponse du système LTI. Ses propriétés vont donc dépendre uniquement
de la position des pôles et des zéros.

Réponse en fréquence en régime permanent : cas particulier où sigma = 0
: l'excitation est monochromatique ou harmonique. On parlera alors de
réponse~ en fréquence. Il s'agit d'un cas particulier, mais on verra
plus tard l'intérêt pour l'analyse du signal, via la transformée de
Fourier, où le signal sera décomposé en une série de composantes
harmoniques. Il s'agit d'un cas pratique largement utilisé pour
l'analyse des filtres. Dans ce mode d'utilisation, la réponse est
considérée en régime établie ou permanent. Autrement dit, toute la
réponse transitoire initiale est déjà écoulée. Nous reviendrons sur ces
notions dans le chapitre consacré à l'étude des filtres.

Passage de la fonction de transfert à la réponse en fréquence : dans le
cas général, l'excitation est complexe. Cas particulier où sigma est
nul. Il suffit donc de remplacer p par jw. w est reliée à la fréquence
par : w = 2pi*f.

v(t) =~ \textbar{}H(w)\textbar{}.exp(j.phi)*exp(jw))~ excitation
d'amplitude = 1, v(t) est complexe. Le module de la réponse en fréquence
est le gain en amplitude de la fonction de transfert du système ou du
filtre. Le déphasage du signal en sortie du filtre par rapport au signal
d'entrée est l'argument de la fonction de transfert. Le module et la
phase sont des fonction de la fréquence w.

Si on considère une excitation harmonique réelle, par exemple
cosinusoïdale : cos(wt) = 1/2(exp(jwt)+exp(-jwt))~ v(t) =
Re(H(jw)*exp(jwt))

Remarque : nous avons vu deux manières de représenter l'effet d'un
système LTI, selon que 'l'on considère une excitation impulsionnelle ou
exponentielle complexe. L'une est directement définie dans le domaine
temporel, l'autre dans le domaine fréquentiel. Bien qu'il semblerait
plus naturel de faire le calcul de la réponse dans le domaine temporel,
nous avons montré que le calcul dans le domaine fréquentiel était
beaucoup plus évident, car il se résumait à une simple multiplication.
En effet, quel que soit l'effet du système LTI (qu'il soit intégrateur,
différentiateur, proportionnel), le calcul se résumait à une
multiplication.

Comme il s'agit du même système, il y a nécessairement un lien entre la
réponse impulsionnelle et la fonction de transfert.~ C'est ce que nous
allons voir dans les prochains chapitres, où nous aborderons les
questions de transformée de Laplace, puis de Fourier. Nous allons ainsi
mettre en évidence des transformations mathématiques, permettant le
passage d'une fonction temporel au domaine fréquentiel (et inversement).

Quelque part, il faudrait s'assurer que l'utilisation de la notation
complexe pour traiter de signaux harmoniques est comprise (phaseur)~
simplification de l'analyse de superposition de signaux harmoniques et
de systèmes en régime permanents, de résolution d'équation
différentielles ordinaires.

Ce qu'il faut retenir

Exercices

Un exercice général permettant d'identifier si un filtre est LTI à
partir de son équation différentielle.

Un exercice avec un système d'ordre 1 (par exemple un circuit RL) avec
une condition initiale et que l'on soumet à une excitation harmonique.
On pourrait aussi proposer le même exercice avec un circuit RC.

Question bêbête : est-ce que les termes suivants sont justes : un
système linéaire à temps variant, un signal non linéaire, un signal non
causal.

Transformée de Laplace

Position du problème : dans le chapitre précédent, nous avons identifié
deux manières de calculer la réponse transitoire d'un système :

soit en considérant la réponse impulsionnelle et en calculant le produit
de convolution dans le domaine temporel

soit en considérant la fonction de transfert et en réalisant une simple
multiplication dans le domaine fréquentiel, nécessitant une excitation
exponentielle complexe.

Bien que cette deuxième méthode soit plus simple, elle suppose une
excitation d'un type donné. La question que l'on se pose est : peut-on
passer directement d'une fonction temporelle à sa forme dans le domaine
fréquentiel complexe, même si celle-ci n'est pas une exponentielle
complexe ? Nous allons voir que cela est possible, via une
transformation mathématique appelée transformée de Laplace. Elle va
permettre de passer d'un domaine à son domaine dual. Dans les chapitres
suivants, nous présenterons un autre type de transformation dérivée de
la transformée de Laplace : la transformée de Fourier, qui sera très
utile pour l'analyse des signaux, ainsi que pour l'étude des systèmes en
régime permanent..

On considère que des fonctions causales, avec un instant d'apparition
des signaux en t = 0.

\begin{itemize}
\itemsep1pt\parskip0pt\parsep0pt
\item
  Dérivation d'une transformation d'une fonction temporelle du domaine
  temporel
\end{itemize}

Dans le domaine temporel, l'entrée et la sortie du système sont reliés
via la réponse impulsionnelle h(t) selon le produit de convolution. Cela
est vrai quel que soit le type d'excitation appliquée en entrée du
système. Le passage dans le domaine fréquentiel suppose une excitation
exponentielle complexe. Considérons ce cas dans le domaine temporel : la
réponse sera donc égale au produit de convolution entre la réponse
impulsionnelle et l'excitation exponentielle complexe. Nous considérons
dans un premier temps un signal défini sur tout le domaine temporel.

Cette relation peut être modifiée selon la forme suivante : .

On remarque que la réponse du système est le produit entre l'excitation
et un terme intégrale dépendant de la réponse impulsionnelle. On
retrouve donc une forme très similaire à celle vue à l'équation xxx,
reliant réponse et excitation exponentielle complexe, via la fonction de
transfert. Ce terme intégrale n'est donc rien d'autre que la réponse du
système LTI dans le domaine fréquentiel complexe, c'est-à-dire sa
fonction de transfert.

Nous venons de mettre en évidence une transformation mathématique
permettant de passer de la représentation temporelle d'une fonction à sa
représentation fréquentielle complexe. Cette transformée s'appelle la
transformée de Laplace.~

La démonstration précédente reste valable dans le cas d'un signal
causal, défini pour t \textgreater{} 0. Dans ce cas :

Transformée de Laplace

\begin{itemize}
\itemsep1pt\parskip0pt\parsep0pt
\item
  Définition
\end{itemize}

La transformée de Laplace est une transformation mathématique,
transformant une fonction temporelle en une fonction définie dans le
domaine des fréquences complexes p. Nous ne noterons TL{[}{]} cette
opération. On distingue deux types de transformée de Laplace :
bilatérale ou unilatérale. Cette dernière s'impose naturellement dès que
l'on considère des systèmes causaux (h(t) = 0 pour t \textless{}0). Dans
la suite, comme nous ne traiterons que de systèmes causaux, nous ne
considérerons que la transformée de Laplace unilatérale.

Il est important de noter que la transformée de Laplace n'existe que si
l'intégrale converge. Ceci n'est pas le cas pour toutes les valeurs de
p.

L'objectif de ce cours n'est pas de faire ce calcul intégrale, mais
plutôt de l'utiliser comme outil pour l'étude des systèmes linéaires.
Nous utiliserons la plupart du temps des tables contenant les
transformées pour les fonctions les plus courantes. Néanmoins, nous
allons mettre en œuvre ce calcul à travers un exemple et indiquer les
conditions d'existence de la transformée de Laplace. Nous allons mettre
en évidence ce que cela signifie du point de vue de la stabilité des
systèmes LTI. Par souci de simplification, on dira qu'un système est
stable si il converge vers une valeur finie.

Conditions d'existence de la transformée de Laplace et stabilité des
systèmes

Exemple : nous considérons un système LTI dont la réponse impulsionnelle
est définie par la fonction :~ où α est un réel quelconque.

Calculons sa transformée de Laplace :

La transformée de Laplace existera si cette intégrale converge, donc
cela dépend de la partie réelle σ de la fréquence complexe :

Si on représente l'excitation en fréquence dans le plan p, il existe un
domaine d'existence de la transformée de Laplace appelé domaine de
convergence. Il s'agit d'un plan où Re(p) = σ \textgreater{} α. Nous le
représentons dans deux cas différents : α \textless{} 0 et α
\textgreater{} 0. Nous allons analyser la signification concrète de ce
domaine de convergence.

Le domaine de convergence indique, dans le cas d'une excitation
exponentielle complexe, l'ensemble des fréquences complexes pour lequel
le système convergera (que l'entrée converge ou pas !). Pour s'en
convaincre, considérons une excitation exponentielle, de fréquence
complexe po = β+jω. Reprenons la forme initiale dérivée de la réponse
impulsionnelle.

Cette fonction converge bien si la partie réelle β de la fréquence de
l'excitation est supérieure à α, autrement dit si la fréquence complexe
d'excitation appartient au domaine de convergence, mais aussi si β
\textless{} 0. Dans le cas contraire le signal d'entrée va diverger pour
t \textgreater{} 0. Cette double condition sur β n'est possible que si α
\textless{} 0, qui est donc une condition nécessaire pour que le système
soit stable.

Remarque : cas d'une excitation harmonique :

Cette situation correspond au cas où β = 0. Le système sera stable
uniquement si, dans le plan p, l'axe des imaginaires appartient au
domaine de convergence. Dans le chapitre consacré à la transformée de
Fourier, nous verrons qu'elle est équivalente à une transformée de
Laplace, mais pour une fréquence purement complexe β = 0. Si l'axe des
imaginaires appartient au domaine de convergence, alors la transformée
de Fourier de la fonction pourra exister.

Pôles et zéros d'une fonction de transfert

Reprenons l'exemple précédent. Comme nous l'avions introduit dans le
chapitre précédent, α correspond à un pôle de la réponse du système. Sa
stabilité est liée à sa partie réelle.~

Toutes les TL que l'on considère peuvent se mettre sous la forme d'une
fonction rationnelle, où le numérateur N(p) et dénominateur D(p) sont
des polynômes. Eux-mêmes peuvent être exprimés en fonction de leurs
racines appelés zéros et pôles respectivement. Ceux-ci correspondent à
des fréquences complexes particulières qui :

annulent la fonction de transfert dans le cas des zéros
p\textsubscript{zi}

font tendre la fonction de transfert vers l'infini dans le cas des pôles
p\textsubscript{pj}

où G est le gain de la fonction de transfert.

Comme nous l'avions déjà vu dans le chapitre A, la position des pôles
nous donne une indication sur l'allure temporelle de la fonction de
transfert (réponse naturelle).

Quel que soit le système linéaire considéré, la position de ces pôles va
déterminer la stabilité du système. Vous approfondirez ces concepts dans
le cours d'automatique.~

Propriétés de la transformée de Laplace

Dans cette partie, nous allons passer en revue plusieurs des propriétés
importantes de la transformée de Laplace, qui facilitent les calculs de
la transformée de Laplace et les applications associées. Nous ne
démontrerons pas l'ensemble de ces propriétés.

\begin{itemize}
\itemsep1pt\parskip0pt\parsep0pt
\item
  Linéarité
\end{itemize}

La transformée de Laplace est linéaire. La transformée de Laplace d'une
somme pondérée de fonction est la somme des transformées de Laplace
individuelle de ces fonctions, pondérées de la même manière. Elle
vérifie donc la relation suivante.

Théorème du retard

Supposons que l'on connaisse la transformée de Laplace F(p) de la
fonction causale f(t), qui est nulle pour t \textless{} 0. Si on a
retardé de t\textsubscript{0} cette fonction, elle devient
f(t-t\textsubscript{0}) celle-ci est nulle pour tout t \textless{}
t\textsubscript{0}. Calculons sa transformée de Laplace à partir de xxx
et appliquons le changement de variable w = t-t\textsubscript{0}. On
montre que la transformée de Laplace de la fonction retardée est la
transformée de Laplace non retardée multipliée par
e-\textsuperscript{pt0}.

Théorème du changement d'échelle

Supposons que l'on connaisse la transformée de Laplace F(p) de la
fonction causale f(t). Dilatons l'échelle de temps de cette fonction par
un facteur réel k. La fonction devient f(kt). Calculons la transformée
de Laplace de cette version dilatée de f(t), en appliquant le changement
de variable w = kt.

Dérivation dans le domaine temporel

Considérons le cas d'une fonction f(t) continue non nulle pour t
\textgreater{} 0. La transformée de Laplace de la dérivée de cette
fonction est donnée par :

On remarque que l'opérateur dérivée correspond à une multiplication de
la transformée de Laplace par p. Cependant, il est nécessaire de tenir
compte de la condition initiale f(0\textsuperscript{+}).

Il est possible de généraliser cette relation au cas où l'on dérive N
fois. L'expression devient :

Le calcul nécessite de connaître les conditions initiales non seulement
sur la valeur de la fonction mais sur ses (n-1) premières dérivées.

Dérivation dans le domaine des fréquences complexes p

Intégration temporelle

Si une fois et si N fois

Théorème de la valeur initiale

Théorème de la valeur finale

Multiplication et produit de convolution

Fonctions périodiques causales

Fonctions causales avec N répétitions

En lien avec la propriété précédente

Tableau récapitulatif des propriétés

Tableau 5.1 du livre Y. Mori.

Uniquement les plus importantes.

Transformée de Laplace des fonctions courantes

Le tableau ci-dessous donne la transformée de Laplace des fonctions
causales les plus courantes, et pour lesquelles des formes mathématiques
simples existent. Pour la plupart, aucune démonstration n'est donnée,
mais les transformées peuvent être retrouvées. Nous nous limiterons à
deux exemples de mise en forme de la transformée de Laplace pour deux
formes temporelles courantes, sachant que nous avons déjà vu celle d'une
fonction exponentielle.

Exemple 1 : échelon unitaire f(t) = u(t)

Exemple 2 : fonction cosinusoïdale

Tableau récapitulatif des transformées de Laplace

~~ ~ ~

\begin{longtable}[c]{@{}ll@{}}
\toprule\addlinespace
\begin{minipage}[t]{0.47\columnwidth}\raggedright
\textbf{Fonctions temporelles f(t)}

\textbf{Transformée de Laplace F(p)}
\end{minipage} & \begin{minipage}[t]{0.47\columnwidth}\raggedright
Echelon unitaire u(t)
\end{minipage}
\\\addlinespace
\bottomrule
\end{longtable}

A partir de cette table, il est possible de dériver les transformées de
Laplace pour d'autres fonctions temporelles. En effet, il suffit
d'identifier des formes dont la transformée de Laplace est connue et
utiliser les différentes propriétés de la transformée pour en déduire
l'expression de la transformée de Laplace.

Exemple : cas composé de plusieurs formes connues~ par exemple trois
marches d'escalier, avec transition en t = 0, 1, 2 (exo 3-a p 108 Y.
Mori) : solution : f(t) = F(p) = u(t) + u(t-1) + u(t-2) - 3u(t-3)~ F(p)
=
1/p+e\textsuperscript{-p}/p+e\textsuperscript{-2p}/p-3e\textsuperscript{-3p}/p

Transformée de Laplace inverse

\begin{itemize}
\itemsep1pt\parskip0pt\parsep0pt
\item
  Méthode
\end{itemize}

Du point de vue pratique, le but de la transformée de Laplace est de
fournir un outil simple pour le calcul des réponses temporelles des
systèmes linéaires. Elle permet de calculer la représentation
fréquentielle d'une fonction à partir de sa représentation dans le
domaine temporel. Dans le domaine fréquentiel, le calcul de la réponse
du système est facilité en passant par la fonction de transfert.
Cependant, si nous ne disposons pas d'une transformation mathématique
permettant de réaliser le passage inverse, cette approche perd de son
intérêt.~ Fort heureusement, cet outil existe. Il s'agit de la
transformée de Laplace inverse, que nous allons rapidement présenter.~

La transformée de Laplace inverse est donnée par la formule de
Bromwich-Mellin , où σ est choisi pour que l'intégrale converge.

Dans la pratique, cette formule est difficile à mettre en œuvre car elle
nécessite de calculer une intégrale dans le plan complexe. Ceci dépasse
le cadre de ce cours et nous n'effectuerons pas de calcul de cette
transformée. Dans ce cours, nous privilégierons la méthode utilisée en
pratique, basée sur les tables reliant des fonctions courantes et leur
transformée de Laplace, comme le tableau xx. Il s'agit de rechercher des
paires de transformées. A partir de la fonction F(p), on identifie des
fonctions connues et on déduit la fonction temporelle correspondante.
Dans le cas de fonctions compliquées, le calcul de la forme temporelle
passe par des approches basées sur le calcul numérique, non détaillées
ici.~

Décomposition pôles-résidus

Comme nous l'avons dans la partie II.3, la plupart des transformées de
Laplace peuvent être mis sous la forme d'une fonction rationnelle,
rapport de deux polynômes faisant apparaitre leurs racines (pôles et
zéros). Cependant, il n'existe sans doute pas une paire de transformées
simple pour toutes les fonctions rationnelles. Pour appliquer la méthode
précédente, il est nécessaire de décomposer cette fonction rationnelle
en éléments plus simple, dont on connait la paire de transformée.

Il est possible de montrer que toute fonction rationnelle de deux
polynômes peut s'écrire sous la forme d'une somme d'éléments simples,
appelées fractions partielles ne dépendant que des pôles et de
constantes appelées résidus. Cette décomposition s'appelle décomposition
pôles-résidus.~

Forme après décomposition pôles-résidus

On voit immédiatement l'intérêt de cette approche car on remarque qu'on
identifie immédiatement la paire de transformée simple. En effet, la
transformée de Laplace inverse de la fraction partielle est une fonction
exponentielle, dépendante du pôle (remarque : on retrouve l'expression
temporelle de la réponse naturelle vue dans le chapitre A).

Méthode : cas où n \textgreater{}= m et cas où n \textless{} m

~Méthode du cache~ vue en première année.

Exemple

Applications~

Nous allons présenter deux applications où l'utilisation de la
transformée de Laplace et de son inverse trouve tout leur intérêt : dans
la résolution d'équations différentielles et dans le calcul de la
réponse transitoire de systèmes linéaires.

\begin{itemize}
\itemsep1pt\parskip0pt\parsep0pt
\item
  Résolution d'équations différentielles ordinaires
\end{itemize}

Les équations différentielles ordinaires s'écrivent de la manière
générale suivante. Elles résultent de l'effet d'opérateurs
proportionnels, intégrateurs et différentiateurs. Dans le domaine des
fréquences complexes, ces deux derniers effets se résument à une
multiplication et une division par la fréquence p. On constate qu'en
passant dans le domaine fréquentiel, l'équation différentielle se
transforme en une simple simple équation algébrique. On retrouve bien
que les fonctions Y(p) et X(p) sont reliées par un terme H(p) qui est la
fonction de transfert et qui ne dépend que des coefficients
a\textsubscript{i} et a\textsubscript{j}. Une fois l'expression de la
fonction de transfert établie, on peut calculer la solution Y(p) de
l'équation différentielle pour toute excitation X(p) (à condition que
l'on puisse calculer la transformée de Laplace de x(t)). On pourra
déterminer ensuite la solution y(t) dans le domaine temporel à l'aide de
la transformée de Laplace inverse.

Exemple : soit l'équation différentielle \ldots{} avec comme conditions
initiales \ldots{}

Calcul de la réponse transitoire d'un système linéaire

Comme le comportement transitoire d'un système est dicté par une
équation différentielle, déterminer la réponse temporelle à une
excitation donnée revient à résoudre cette équation différentielle. Le
problème est donc équivalent au précédent. Supposons que nous disposions
de la fonction de transfert H(p) du système et que l'on recherche la
réponse y(t) à une excitation x(t) dont l'expression est donnée. La
première étape consiste à déterminer la transformée de Laplace X(p) de
l'excitation. La réponse du système dans le domaine fréquentiel Y(p) est
directement le produit de la fonction de transfert et de l'excitation
X(p). Ensuite, la transformée de Laplace inverse de Y(p) permet de
retrouver l'expression de la réponse temporelle y(t).

Il est bien entendu indispensable de prendre en compte les conditions
initiales du système, qui vont intervenir dans la réponse naturelle du
système. Si l'expression de la fonction de transfert ne les intègre pas,
il est indispensable de repartir de l'équation différentielle décrivant
le système et établir la fonction de transfert en intégrant l'effet des
conditions initiales.

Exemple : on considère un système dont la fonction de transfert est
donnée par \ldots{} On souhaite calculer sa réponse à l'excitation x(t)
= \ldots{} Les conditions initiales sont \ldots{}

Ce qu'il faut retenir

Exercices

Un exercice de calcul de transformée de Laplace qui exploite les
propriétés de la transformée.

Exemple :

sin t.u(t)~ on peut utiliser la table. sin 3t.u(t)~ utilisation du
théorème de changement d'échelle~ 3/(p²+9)

Fonction impulsion : A*{[}u(t-a)-u(t-b){]}

exp(kt)*cos(wt)~ théorème de l'amortissement : TL{[}exp(-kT).f(t){]}~
F(p+k). Ex : cos(2t)*exp(-t)~ (p+1)/((p+1)²+4)

Ajout d'un déphasage dans sin et cos :

Trois impulsions qui se suivent

Filtrage

Ici, uniquement dans le domaine fréquentiel. L'analyse dans le domaine
temporel sera vue dans le chapitre xxx, et utilisera la réponse
impulsionnelle.

analyse du comportement fréquentiel d'un système linéaire~ on considère
le cas d'une excitation purement harmonique. Définition des différents
types de filtre, gabarit fréquentiel et caractéristiques associées
(bande passante, ordre, fréquence de coupure).

On considère le régime harmonique établi~ p = jw. Ici, l'analyse
harmonique va jouer un rôle majeur dans l'analyse fréquentielle des
filtres.

A partir de la fonction de transfert dans le domaine de Laplace : H(p)~
H(jw) ou H(f).

Dans la suite, par souci de simplification, nous parlerons de fréquence
f au lieu de pulsation ω. La fonction de transfert ne sera pas notée
H(jω) mais H(f).

\begin{itemize}
\itemsep1pt\parskip0pt\parsep0pt
\item
  Définition d'un filtre
\end{itemize}

Un filtre est un système présentant une certaine sélectivité dans le
domaine des fréquences, et qui est donc utilisé pour limiter le spectre
d'un signal à une certaine bande de fréquences.

Un filtre atténue certaines des composantes fréquentielles du signal
d'entrée et en laisse passer d'autres, d'où son appellation de filtre !
C'est bien cette opération sélective d'atténuation ou d'amplification
des fréquences que modélise la multiplication de l'excitation par la
fonction de transfert.

La fonction de transfert peut s'écrire : , qui se décompose sous la
forme d'un module et d'un argument. Le module correspond au gain du
filtre, c'est-à-dire le rapport entre le module de la réponse du filtre
sur celui de son excitation. Si les signaux d'entrée et de sortie
représentent le même type de grandeur (par exemple, une tension dans le
cas de signaux électriques), le gain est sans unité. L'argument
correspond au déphasage apporté par le filtre.

retard : le déphasage traduit un retard τ, dépendant de la fréquence,
qui se calcule de la manière suivante : Vérifier la formule car une
phase \textless{} 0 va donner tau \textless{} 0. Or, si tau est un
retard, alors il devient une avance dans cette situation.

Un déphasage constant quelque soit la fréquence n'est pas réalisable en
pratique\ldots{}

On remarque une propriété intéressante : soit un signal est constitué de
la superposition de plusieurs signaux sinusoïdaux, qui traverse un
filtre présentant un déphasage variable en fonction de la fréquence.
Même si le gain est le même à ces différentes fréquence, si le retard
introduit par le filtre n'est pas le même, alors le signal présentera
une distorsion.

Cependant, si le déphasage varie linéairement avec la fréquence, on
remarque que le retard sera constant. Dans la situation précédente, la
distorsion disparaîtra. Il est courant de réaliser des filtres dits à
phase linéaire pour éviter ce type de problème.

\begin{itemize}
\itemsep1pt\parskip0pt\parsep0pt
\item
  Filtre à réponse impulsionnelle réelle
\end{itemize}

Considérons le cas particulier, mais pratique, où la réponse
impulsionnelle est une fonction réelle : .

En effet, une réponse temporelle prenant des valeurs complexes n'aurait
pas directement de sens physique ! On peut alors appliquer la propriété
suivante de la Transformée de Fourier : la Transformée de Fourier d'une
fonction réelle est une fonction en général complexe mais admettant une
symétrie conjuguée, i.e. dont le module est une fonction paire tandis
que l'argument est une fonction impaire. Appliquée à la réponse
impulsionnelle, cette propriété devient :

~~ et ~

Fréquence négative : comme nous l'avons vu dans les chapitres
précédents, la fréquence est une grandeur réelle pouvant être négative.
En raison de propriétés de symétries (que nous décrirons plus tard), les
représentations que nous utiliserons dans ce chapitre ne font apparaître
que les fréquences positives. Les fréquences négatives sont omises par
convention parce qu'elles n'apportent pas d'informations
supplémentaires.

la notion de basse ou de haute fréquence ne s'attache qu'à la valeur
absolue de la fréquence considérée : seules les fréquences positives f
appartenant à R+ ont un sens physique ; les fréquences négatives f
appartenant à R-* n'ont pas de signification physique directe, mais leur
prise en compte assure une utilisation correcte de l'analyse harmonique
comme outil d'étude des filtres.

La propriété de symétrie paire du gain explique pourquoi les notions de
fréquence de coupure et de bande passante ne sont définies que pour les
fréquences positives. De la même manière, le tracé du gabarit ou les
tracés du plan de Bode ne se font en général eux aussi que pour les
fréquences positives ; ces tracés seront ensuite complétés si nécessaire
par symétrie (symétrie paire ou symétrie impaire selon qu'il s'agit du
module ou de l'argument de la fonction de transfert).

Analyse harmonique

A partir de la connaissance du gain et du déphasage de la fonction de
transfert, lorsque le filtre est attaqué par un signal (co)sinusïdal, il
est très simple de déterminer l'expression de sa réponse. En considérant
une notation complexe, la réponse temporelle du filtre en régime
harmonique est donnée par :

Celle-ci peut aussi s'écrire :

Représentation - Diagramme de Bode - Gabarit

Diagramme de Bode

L'étude d'un filtre passe donc par une analyse de son gain et de son
déphasage. Il faut donc chercher une représentation graphique adaptée à
l'analyse du comportement fréquentiel du filtre. La méthode la plus
courante est le tracé du diagramme de Bode. Il contient deux tracés
graphiques décrivant :

l'évolution du gain en fonction de la fréquence

l'évolution du déphasage en fonction de la fréquence

Bien qu'il n'existe pas une forme de représentation unique du diagramme
de Bode, certains usages sont récurrents et il convient de les
expliquer. Pour des raisons pratiques, l'axe fréquentiel est
généralement tracé en échelle logarithmique. En général, on s'intéresse
au comportement fréquentiel sur de large gamme de fréquence, couvrant
plusieurs décades. Un tracé avec une échelle linéaire offre une
résolution constante, qui cache les détails dans les basses fréquences.
Avec un tracé en échelle logarithmique, la résolution varie en fonction
de la fréquence et s'adapte à chaque décade. Le même niveau de détail
peut être apprécié en basse ou en haute fréquence. C'est ce qu'illustre
la figure ci-dessous : le filtre étudié est de type passe-bas,
c'est-à-dire qu'il laisse passer les signaux basses fréquences. Le tracé
est réalisé entre 100 Hz et 1 MHz. Avec l'échelle linéaire, le tracé ne
permet pas d'évaluer précisément à partir de quelle fréquence le filtre
commence à atténuer significativement le signal d'entrée. Avec l'échelle
logarithmique, on peut mieux évaluer cette fréquence, appelée fréquence
de coupure, qui se situe aux alentours de 1 kHz.

Conseil pratique : comment tracer une échelle logarithmique ?

On définit tout d'abord le pas représentant une décade (par exemple 2 cm
par décade). On considère que l'origine sera la valeur 1 (log10(1) = 0).
La position de toute valeur par rapport à cette origine sera donnée par
pas*log10(valeur). Ainsi, la valeur 10 (une décade de plus que 1) sera
située à 2 cm à droite de l'origine, 100 (deux décades de plus) sera
située à 4 cm à droite de l'origine, tandis que 0.1 (une décade de
moins) sera située à 2 cm à gauche de l'origine. La valeur 50 sera
située à 2*log10(50) = 3.4 cm à droite de l'origine.

Une autre convention consiste à exprimer les gains en décibels, qui
n'est rien d'autre qu'une représentation des valeurs sur une échelle
logarithmique.. De part leur sélectivité en fréquence, les gains des
filtres vont présenter des valeurs très variables en fonction de la
fréquence. Pour les mêmes raisons que la représentation des fréquences,
une échelle logarithmique pour le gain permettra d'apprécier avec la
même résolution les faibles et les fortes valeurs de gain. La figure
ci-dessous l'illustre, en reprenant l'exemple du tracé du gain du filtre
précédent. Le tracé du gain sur une échelle linéaire ne permet pas de le
mesurer précisément lorsqu'il atteint de faibles valeurs. A contrario,
lorsqu'il est exprimé en dB, la mesure des faibles valeurs devient plus
précise et il est possible de mettre en évidence des tendances
particulières du gain. Par exemple, le gain de ce filtre décroit avec
une pente de -20 dB par décade au-dessus de la fréquence de coupure, ce
qui est typique d'un filtre passe-bas d'ordre 1. ~

\begin{longtable}[c]{@{}llll@{}}
\toprule\addlinespace
\begin{minipage}[t]{0.22\columnwidth}\raggedright
Il n'y a pas de convention particulière pour le déphasage, qui peut être
exprimé en radians ou en degrés.~

La figure ci-dessous présente le diagramme de Bode du filtre précédent,
montrant les tracés de l'évolution fréquentielle de son gain et de son
déphasage. Le diagramme de Bode indique un déphasage dont la valeur
diminue entre 0 et -π/2. Le déphasage étant négatif, le filtre introduit
un retard.
\end{minipage}
\\\addlinespace
\bottomrule
\end{longtable}

Exemples

J'en ai déjà fait un au-dessus.

Filtres d'ordres 1 et 2 : Diagramme de Bode usuels et diagramme
asymptotique

Ordre 1 : passe-bas et passe-haut

Ordre 2 : passe-bas, haut, passe-bande, coupe-bande (faire en TD ?)

Passe-bas ordre 1 :~

Passe-haut-ordre 1 :~

Passe-bas ordre 2 :~

Passe-bande d'ordre 2 :~

Mise en cascade filtre~ Chainage de fonction de transfert

Gabarit

~peut-être pas utile car on ne fera pas de synthèse de filtre.

remarque : normalisation des fréquences :

Un filtre de nature donnée présentera toujours la même forme de gabarit.
Seule la ou les fréquences de coupure changeront selon les propriétés du
filtre. Pour disposer d'une représentation indépendante de la fréquence
de coupure, il est courant de représenter l'axe des fréquences sous la
forme d'une fréquence normalisée. Cette normalisation se fait par
rapport à une fréquence de référence, généralement la fréquence coupure
du filtre. Ainsi, le même gabarit peut être utilisé pour plusieurs
filtres présentant différentes fréquences de coupure. Si l'axe
fréquentiel est exprimé en pulsation, la normalisation se fera de la
même manière.

Fnorm = f/fc

Les principales caractéristiques d'un filtre

Lorsqu'on analyse un filtre ou lorsqu'on cherche à le dimensionner,
plusieurs caractéristiques sont étudiées. Celles-ci définissent le
comportement fréquentiel global. Nous allons les définir ici.~

gain : le gain est dépendant de la fréquence. Néanmoins, on peut
spécifier une valeur unique, qui correspond généralement à la valeur
maximale du gain.

fréquence(s) de coupure : celles-ci correspondent à des fréquences
auxquelles le gain présente un changement de tendance. Dans la pratique,
on considère un critère d'atténuation du gain de -3 dB (soit une
division du gain par √2). On appelle fréquence de coupure (en toute
rigueur fréquence de coupure à -3dB) d'un filtre toute fréquence
positive ou nulle fc 2 vérifiant :

~~ ~ ~

Tout filtre admet alors une ou plusieurs fréquences de coupures : un
filtre de gabarit passe-bas ou de gabarit passe-haut n'a qu'une
fréquence de coupure, tandis qu'un filtre passe-bande ou coupe-bande
classique a deux fréquences de coupure.

bande passante : la sélectivité du filtre est donnée par sa bande
passante. Il s'agit de la plage de fréquence sur laquelle l'atténuation
est faible. Elle est délimitée par deux fréquences, définies selon un
critère d'atténuation par rapport au gain maximal obtenu dans la bande
passante. Un critère courant consiste à considérer une atténuation de -3
dB. La bande passante est donc la bande de fréquence où le gain reste
compris entre sa valeur maximale et sa valeur maximale atténuée de 3 dB.
On définit la bande passante d'un filtre comme l'ensemble des fréquences
f de R+ pour lesquelles le module jH(f)j de la fonction de transfert
demeure au moins égal à 1/racine(2) fois sa valeur maximale :

~~ ~

La bande passante spécifie donc le domaine de fréquences à l'intérieur
duquel le gain du filtre demeure plus ou moins constant, ou du moins ne
chute pas de plus de 3 décibels. Elle donne ainsi la plage de fréquences
que le filtre va laisser passer, d'où son nom de bande passante ! La
bande passante d'un filtre est constituée d'un ou plusieurs intervalles
de R+, les bornes de ces intervalles étant données par les fréquences de
coupure du filtre. Par exemple, la bande passante d'un filtre passe-bas
est de la forme {[}0; fc{]} où fc est l'unique fréquence de coupure,
tandis que la bande passante d'un filtre passe-haut est de la forme
{[}fc;+1{]} ; la bande passante d'un filtre passe-bande classique (i.e.
avec une seule bande passante) est de la forme {[}fc1; fc2{]} où fc1 et
fc2 sont les deux fréquences de coupure, tandis que la bande passante
d'un filtre coupe-bande classique (i.e. avec une seule bande coupée) est
de la forme {[}0; fc1{]} {[} {[}fc2;+1{]}.

Illustration de la bande passante ou de la fréq de coupure ?

équivalent bande passante pour un coupe bande

pente de variation : lié à l'ordre

fréquence(s) de résonance et d'antirésonance

Les différents types de filtres

On classifie les filtres en quatre types selon la forme de leur gabarit
: un filtre passe-bas laisse passer les basses fréquences (i.e. celles
proches de la fréquence nulle) et atténue les hautes fréquences (i.e.
celles grandes en valeur absolue), à l'inverse d'un filtre passe-haut ;
le rôle d'un filtre passe-bande est de laisser passer uniquement les
fréquences contenues dans un intervalle donné de fréquences, en
atténuant toutes les autres, tandis qu'à l'inverse le rôle d'un filtre
coupe-bande est de supprimer toutes les fréquences contenues dans un
intervalle donné en laissant passer toutes les autres. Les gabarits des
différents types de filtres idéaux :

Les filtres physiquement réalisables sont néanmoins caractérisés par :

des pentes de variation de gain finis, liées à l'ordre du filtre

Méthodes simples d'analyse de fonction de transfert

Dans la pratique, on s'intéresse souvent à la tendance du comportement,
ou au comportement autour de point spécifique.

Comportement asymptotique : les valeurs des pentes en fonction des
ordres. Ordre 1 : *10~ +20 dB. Ordre 2 : * 10~ + 40 dB. Ordre N : *10~
20*N dB. On parlera de pente en dB/dec.

Analyse aux résonances et anti-résonances

Aspects énergétiques

C'est sans doute trop tôt pour voir :~

Ce qu'il faut retenir

Une question qui se pose maintenant : nous avons vu que l'étude des
systèmes LTI, dont les filtres, est particulièrement simplifiée dans le
domaine fréquentiel, via l'analyse harmonique. La transformée de Laplace
constitue un outil permettant d'extraire la fonction de transfert des
systèmes LTI.

Notre objectif est de déterminer comment un filtre agit sur un signal
d'entrée quelconque. Les signaux sont généralement acquis dans le
domaine temporel. Pour analyser l'effet du filtre, il est donc
nécessaire de calculer la représentation fréquentielle du signal au
préalable. Pour cela, nous allons utiliser la transformée de Fourier qui
sera présentée dans les deux prochains chapitres. Comme nous le verrons,
cet outil est très proche de la transformée de Laplace, qui constitue
une généralisation.~

Exercices~

Un exercice sur un filtre déphaseur pur (passe-tout).~

Exo bêbête : on présente le contenu fréquentiel d'un signal bruité, en
indiquant la bande passante du signal utile. Quel type de filtre faut-il
mettre ? Quel ordre ?

Filtre anti-glitch

Soit une perturbation, que l'on considère d'abord comme une impulsion
rectangulaire de largeur tau, puis d'impulsion double exponentielle.

Dans chaque cas, calculez la transformée de Laplace des signaux.

Exprimez l'amplitude maximale atteinte par ces signaux.

On applique un filtre passe-bas d'ordre 1 d'expression \ldots{} Calculez
la réponse du filtre dans le domaine de Laplace pour les deux types
d'excitation.

Calculez les amplitudes maximales des signaux. Quelle est l'influence du
paramètre Fc.

Ce filtre est placée sur une ligne de signal électrique de fréquence
\ldots{} Proposez une valeur pour Fc.

Série de Fourier

Transformée de Fourier

Analyse des systèmes et des signaux dans le domaine temporel

Jusque-là, nous avons principalement calculé la réponse de systèmes LTI
en passant par le domaine fréquentiel, en utilisant les notions de
fonctions de transfert, de transformée de Fourier et de Laplace. Cette
approche permettait de calculer facilement cette réponse.

Tous les calculs auraient pu être effectué en restant dans le domaine
temporel à partir de la réponse impulsionnelle du système, que nous
avions introduite au chapitre B. Nous l'avions laissé de côté
momentanément car son utilisation passait par un calcul relativement
complexe appelé produit de convolution. Nous allons le détailler dans ce
chapitre.

Nous pourrions croire que nous sommes en train de présenter un nouvel
outil pour analyser l'effet d'un même système LTI. Nous allons voir au
contraire que les concepts de fonction de transfert et de réponse
impulsionnelle sont intimement liées.

L'analyse temporelle présente de nombreux intérêts que nous allons aussi
aborder. Par exemple, la notion de corrélation, permettant de mesurer le
degré de ressemblance entre deux signaux. Cette dernière notion ne
permettra d'aborder un des aspects que nous n'avons pas encore abordé :
la manière dont l'énergie est répartie dans le domaine fréquentiel et
l'impact du système sur le transfert d'énergie ou de puissance. A partir
des relations dans le domaine temporel et fréquentiel, nous pourrons
aborder ces questions.

\begin{itemize}
\itemsep1pt\parskip0pt\parsep0pt
\item
  Réponse impulsionnelle
\end{itemize}

Comme nous l'avons vu dans le chapitre B, il s'agit de la réponse d'un
système LTI à une excitation impulsionnelle élémentaire, modélisée par
une impulsion de Dirac. Sa connaissance permet de déterminer la réponse
du système quelle que soit l'excitation appliquée à l'aide de la
relation suivante.

Rappel pour un système causal :~

Comme nous l'avons vu dans les chapitres précédents, la fonction de
transfert n'est rien d'autre que la transformée de Laplace de la réponse
impulsionnelle d'un système. Plusieurs moyens permettent de le
démontrer. On peut exploiter le lien entre la multiplication et le
produit de convolution : un produit de convolution dans le domaine
temporel est équivalent à une multiplication dans le domaine
fréquentiel.

Dans la prochaine partie, nous allons détailler les propriétés du
produit de convolution et voir comment le mettre en œuvre.

Produit de convolution

\begin{itemize}
\itemsep1pt\parskip0pt\parsep0pt
\item
  Définition
\end{itemize}

Propriétés

Ajouter la propriété d'équivalence de multiplication dans le domaine
fréquentiel (transformée de Laplace et de Fourier).

Ajouter aussi le produit de convolution avec un Dirac~ élément neutre ou
décalage temporel.

Dans le produit de convolution, on peut aussi intervertir les arguments
τ et t-τ, ce qui revient à écrire x(t)*h(t) = h(t)*x(t)~ x*h(t) =
integ(x(t-tau).h(tau)dtau)

Mise en œuvre du produit de convolution

Voyons la démarche pour calculer manuellement un produit de convolution.
Nous considérons un système causal dont on connait la réponse
impulsionnelle h(t) ainsi que l'excitation x(t).

1. changement de variable (on remplace t par tau)

2. Pliage de la fonction : rotation à 180° autour de l'axe des ordonnées
de la fonction : en d'autres termes, f(tau) devient f(-tau)

3. Décalage de la fonction pliée : f(t-tau) est décalé à droite si t
\textgreater{} 0.

4. Produit des deux fonctions point à point

5. Intégration du produit des deux fonctions~ résultat du produit de
convolution en t. On répète l'opération pour les autres valeurs de t.

Le refaire avec Matlab

\textbf{Exemple :~}

Dans l'exemple ci-dessous, on considère un filtre d'ordre 1 dont on
connait la fonction de transfert : . Il est excité à son entrée par un
signal x(t) = u(t) - u(t-2). Calculez et tracez la réponse de ce filtre
en utilisant sa réponse impulsionnelle.

La réponse impulsionnelle du filtre peut être déterminée à partir de la
transformée de Laplace inverse de sa fonction de transfert. A partir de
la table des transformées de Laplace usuelles, on identifie la réponse
impulsionnelle :~ .

Nous pouvons donc calculer la réponse du filtre à l'excitation x(t) en
calculant le produit de convolution h*x(t).

On trace graphiquement, selon la procédure indiquée ci-dessus. On
considère les trois cas possibles : t \textless{} 0, t compris entre 0
et 2, t \textgreater{} 2.~

On met en équation :

On peut aussi réaliser le calcul dans le domaine de Laplace pour
vérifier la justesse des calculs précédents. On détermine d'abord
l'expression de la fonction x(t) dans le domaine fréquentiel, puis on
calcule la réponse Y(p) à partir de la fonction de transfert. Enfin, on
déduit sa forme temporelle à l'aide de la transformée de Laplace
inverse.

Les résultats concordent, confirmant ainsi la validité de ces calculs.

Corrélation

\begin{itemize}
\itemsep1pt\parskip0pt\parsep0pt
\item
  Définition
\end{itemize}

Comment comparer deux signaux ? Comment mesurer leur ressemblance ? Dans
le cadre de l'analyse des signaux, cette notion de ressemblance
s'appelle la corrélation.~

Une première tentative consiste à utiliser les outils dont nous
disposons, par exemple le produit scalaire. Dans un premier temps, nous
considérons des signaux bornés dans le temps sur un intervalle
{[}t\textsubscript{1};t\textsubscript{2}{]}. Celui-ci correspond à la
moyenne temporelle du produit des deux signaux à comparer
s\textsubscript{1} et s\textsubscript{2}, définie par la relation
ci-dessous.

Ci-dessous, nous présentons deux exemples nous permettant d'estimer
l'intérêt de cette définition. A gauche, deux fonctions portes de même
largeur apparaissent en même temps. La définition précédente renvoie une
valeur maximale, indiquant une très grande ressemblance entre les deux
signaux. A droite, on a décalé le signal s\textsubscript{2} par rapport
s\textsubscript{1}. Ils se ressemblent autant que dans le cas précédent,
seul un délai a été ajouté entre eux. Pourtant, l'indicateur de
ressemblance utilisé renvoie une valeur nulle que nous interprétons
comme une faible corrélation.

La définition que nous avons choisi pour la corrélation n'est pas
satisfaisante car elle prend en compte la position relative des deux
signaux. Une astuce pour ne plus la prendre en compte consiste à refaire
le même calcul, mais pour différente position relative entre les deux
signaux, que nous noterons τ. L'indicateur calculé correspond à la
corrélation entre ces deux signaux. Il n'est pas à valeur unique
puisqu'il dépend de τ. De manière générale, lorsqu'on travaille avec des
signaux complexes, il est nécessaire de prendre le conjugué du signal
s\textsubscript{2}.

Dans le cas où le signal est à support temporel infini, la relation
devient :

Dans le cas de fonctions périodiques de même période :~

Reprenons l'exemple précédent et calculons la corrélation des deux
signaux. On décale la position respective du signal s2 par rapport au
signal s1. Pour chaque position, on calcule la corrélation.

Selon les signaux considérés dans le calcul de corrélation, on
distinguera deux cas :

si le calcul s'effectue sur le même signal (s\textsubscript{1} =
s\textsubscript{2}), on parlera d'autocorrélation que l'on notera
R\textsubscript{11}(τ)

si le calcul s'effectue sur deux signaux différents, on parlera
d'intercorrélation que l'on notera R\textsubscript{12}(τ)

On remarque que le calcul de la corrélation est assez proche de celui du
produit de convolution. En effet, dans les deux cas, il s'agit d'un
calcul d'une intégrale appliqué à un produit de deux signaux que l'on
décale. La seule différence est qu'on ne ``retourne'' pas un des deux
signaux.~ ~

Comparaison des formules de produit de convolution et de corrélation
pour deux signaux à valeurs réels :

Remarque : lorsqu'on s'intéresse à ce que représente la corrélation, on
se rend compte qu'elle représente une puissance. En effet, il s'agit
d'une grandeur quadratique.

Propriétés

Changement de variable~

. En effet, retarder la première fonction revient à avancer la seconde.
Cette même propriété se retrouve pour la fonction d'intercorrélation.

Parité :

la fonction d'autocorrélation est une fonction paire. En effet, on peut
montrer que :

Par contre, ce n'est pas nécessairement le cas pour la fonction
d'intercorrélation.

Valeur maximale :

l'autocorrélation présente toujours un maximum pour τ = 0.
R\textsubscript{11}(0) représente la puissance moyenne du signal. On en
déduit donc R\textsubscript{11}(0) ≥ R\textsubscript{11}(τ). Si le
signal est périodique (de période T), alors la fonction
d'autocorrélation présente des maxima pour τ = k.T où k est un nombre
entier. Ce n'est pas forcément le cas pour la fonction
d'intercorrélation.

Exemple~

Relation de Wiener-Khintchine

\begin{itemize}
\itemsep1pt\parskip0pt\parsep0pt
\item
  Densité spectrale de puissance
\end{itemize}

Soit le signal x(t) dont le spectre est donné par X(f) via la
transformée de Fourier. Le spectre en amplitude est donné par
\textbar{}X(f)\textbar{}.

La densité spectrale de puissancedu signal, noté S\textsubscript{x}(f)
est le carré du spectre en amplitude de X(f) :
\textbar{}X(f)\textbar{}². On peut montrer que :~

Unité : W/Hz

Théorème de Wiener-Khintchine

Le théorème de Wiener-Khintchine relie la corrélation d'un signal avec
sa densité spectrale de puissance. Elle permet donc de faire le lien
entre une grandeur définie dans le~ domaine temporel et une autre dans
le domaine fréquentiel. Celui-ci peut facilement être démontré en
repartant de la formule de la corrélation. Considérons un signal f(t).

On peut exprimer les signaux f(t) et f(t-τ)\textsuperscript{*} à partir
de leur spectre. A partir de la définition de la transformée de Fourier
inverse, on peut montrer que :~

~~

En combinant ces différentes équations, on trouve :~

L'autocorrélation du signal f n'est rien d'autre que la transformée de
Fourier inverse de sa densité spectrale de puissance S\textsubscript{f}.
Inversement, on peut en déduire que la densité spectrale de puissance du
signal f est la transformée de Fourier de son autocorrélation.

Application pour les filtres linéaires

Les deux relations ci-dessous permettent de déterminer la manière dont
la puissance d'entrée du signal est propagée par le filtre. Ou, dit
autrement, comment le filtre modifie les propriétés du signal appliqué à
son entrée, par exemple sa fonction d'autocorrélation.

~~ ~

Ce qu'il faut retenir

Exercices

Un exercice d'application de calcul de produit de convolution

Même chose avec la corrélation

Calculer l'autocorrélation de A cos(w0*t) et Acos(w0*t+phi). En déduire
leur densité spectrale de puissance. comparer avec le transformée de
Fourier. Conclure. Faire la même chose avec le signal 2*A*cos(w0*t).

Question bêbête : un filtre peut-il présenter une réponse impulsionnelle
h(t) = 1 ? Un filtre passe-tout présente t-il une réponse impulsionnelle
h(t) = 1 ?

Récapitulatif

A l'issue du cours, les étudiants doivent être capables de connaitre
l'ensemble des transformations entre signaux.

Annexe 1 - Nombres complexes

On rappelle quelques notions de base avec les complexes, notamment la
représentation trigonométrique associée. Il faudrait éclaircir la
notation x(t) = Re{[}Xexp(jwt){]} pour un signal sinusoïdal.

\begin{itemize}
\itemsep1pt\parskip0pt\parsep0pt
\item
  Propriétés de base
\end{itemize}

soit x = a + j.b et y = c+j.d.

Re(x) = a, Im(x) = b.

Addition : x+y = (a+c) +j.(b+d)

Multiplication, division

Conjugué, module, argument

On peut montrer que : . Par contre :~

Représentation géométrique

Les réels peuvent être représentés comme une droite. Mais où se situent
les nombres complexes ? Dans un plan en 2D, il se situent partout
ailleurs. Les nombres réels deviennent ainsi des nombres complexes
particuliers, dont la partie réelle s'annule.

On peut représenter un nombre complexe comme un point P dans ce plan
appelé plan complexe, de coordonnées (a;b), où l'axe des abscisses porte
l'axe des réels et l'axe des ordonnées représente l'axe des nombres
imaginaires.

On peut aussi donner une interprétation géométrique aux différentes
opérations :

additionner une partie réelle revient à une translation parallèle à
l'axe des réels

additionner une partie imaginaire revient à une translation parallèle à
l'axe des imaginaires

Le module correspond à la distance entre le point P et le centre O du
repère

L'argument correspond à l'angle formé entre l'axe des réels et la droite
OP.

On peut en déduire une représentation polaire de nombres complexes,
ainsi qu'une relation trigonométrique :

Le conjugué est le symétrique du point par rapport à l'axe des réels.
Son module est conservé, son argument est inversé.

La multiplication par un réel k correspond à une homothétie de rapport
k. La multiplication par un nombre imaginaire induit en plus une
rotation de π/2.~

Représentation sous la forme d'une exponentielle, formule d'Euler.

Donner un exemple montrant l'intérêt d'utiliser la représentation
complexe plutôt que trigonométrique.

Représentation des signaux (co)sinusoïdaux - Phaseur

L'utilisation de la représentation par des nombres complexes de signaux
à valeurs réels peut présenter de nombreux avantages pratiques.

Prenons un signal cosinusoïdal à valeurs réels X(t) = a.cos(wt+phi). La
forme temporelle est connue et ne sera pas modifiée par l'effet d'un
système linéaire. Seules l'amplitude et la phase apporte une
information. On peut représenter ce signal comme la partie réelle du
nombre complexe :

Une interprétation géométrique consiste à voir le signal comme un
vecteur tournant dans un repère complexe, la vitesse de rotation étant
fixe et donnée par ω. L'information n'est donc portée que par le rayon a
de ce vecteur et sa phase à un instant donnée (par exemple t = 0). Une
manière de mettre en évidence cette partie est de réécrire la relation
précédente en :

Le premier terme s'appelle le phaseur, le second terme représente
l'évolution temporelle du signal. Il contient toute l'information sur ce
signal sinusoïdal : amplitude et phase. Il ne dépend pas du temps. Pour
l'étude des systèmes LTI, cette représentation est adaptée car :

les systèmes étant linéaires, si l'excitation est (co)sinusoïdale, alors
la réponse l'est aussi

les systèmes étant invariants dans le temps, on peut omettre la partie
décrivant l'évolution temporelle

On ne pourra donc s'intéresser qu'à l'effet du système sur le phaseur et
donc omettre le terme exp(jωt) pour simplifier l'écriture des
expressions. De même, on pourra omettre l'opérateur Re{[}{]} puisque
nous savons que des signaux réels sont considérés. On pourra effectuer
l'ensemble des calculs dans le domaine complexe puis ne conserver que la
partie réelle à l'issue des calculs.

Annexe 2 - Distribution de Dirac

Il faudrait montrer l'égalité : integ(A à B) (e\textsuperscript{jwt}) =
2*pi*dirac(t).

Cela peut servir pour démontrer la transformée de Fourier inverse, ainsi
que la transformé de Laplace inverse.

\end{document}
