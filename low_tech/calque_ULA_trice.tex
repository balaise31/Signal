\documentclass[a4paper, landscape]{article}

\usepackage{amsmath}
\usepackage{pgfplots}

\usetikzlibrary{shapes.misc, calc}

\pgfplotsset{compat=newest}

\def\L{29.7}
\def\H{21.0}
\def\ym{-9.5}
\def\yM{9.5}
\def\xm{-20.5}
\def\xM{7.5}




\newcommand{\millimetrique}[2]{
  \draw[step=1mm, line width=0.1mm, brown!25] (#1) grid (#2);
  \draw[step=5mm, line width=0.2mm, brown!50] (#1) grid (#2);
  \draw[step=1cm, line width=0.4mm, brown!75] (#1) grid (#2);
  \draw[step=5cm, line width=0.6mm, brown!100] (#1) grid (#2);
}

\newcommand{\logametrique}[6]{
  \coordinate (S) at ($(#3,#4)$) {};
  
  \foreach \i in {1,2} {
    \foreach \j in {1,...,9} {
      \coordinate (Q) at ($(0,0)!{log10(\i+(\j/10))}!(S)$) {};
      \draw[shift=(Q), xstep=#3cm,  ystep=#4cm, line width=0.2mm, #6!25]  ($(#1)+(0,2*#5)$) grid (#2);
    }
  }
  
  \foreach \i in {3,...,6} {
    \foreach \j in {2,4,...,8} {
      \coordinate (Q) at ($(0,0)!{log10(\i+(\j/10))}!(S)$) {};
      \draw[shift=(Q), xstep=#3cm, ystep=#4cm, line width=0.2mm, #6!25] ($(#1)+(0,1.5*#5)$) grid (#2);
    }
  }
  
  \foreach \i in {1,2,7,8,9} {
    \coordinate (P) at ($(0,0)!{log10(\i+0.5)}!(S)$) {};
    \draw[shift=(P), xstep=#3cm,  ystep=#4cm, line width=0.4mm, #6!50] ($(#1)+(0,#5)$) grid (#2);
  }
  
  \foreach \i in {2,...,9} {
    \coordinate (P) at ($(0,0)!{log10(\i)}!(S)$) {};
    \draw[shift=(P), xstep=#3cm, ystep=#4cm, line width=0.6mm, #6!75] ($(#1)+(0,0.5*#5)$) grid (#2);
  }
  \draw[xstep=#3cm, line width=0.8mm, #6!100] (#1) grid (#2);
  % \draw[step=1mm, line width=0.1mm, #6!25] (#1) grid (#2);
  % \draw[step=5mm, line width=0.2mm, #6!50] (#1) grid (#2);
}

\newcommand{\cercle}[1]{
  \draw[fill=none, line width=0.6mm, brown!100](#1) circle (5) ;
  
  % \foreach \i [count=\k from 0] in {0,11.4591559026,...,359} {
  \foreach \i [count=\k from 0] in {90, 101.4591559026,...,449} {
    \draw ($(#1)+(\i:5.6)$) node[brown] {\textbf{\k}};
  }
  
  \foreach \i in {90,101.4591559026,...,449} {
    \draw[line width=0.4mm, brown!100] ($(#1)+(\i:4.7)$) -- ($(#1)+(\i:5.)$);
  }
  
  \foreach \i  in {90,95.72957795131,...,449} {
    \draw[line width=0.4mm, brown!100] ($(#1)+(\i:4.8)$) -- ($(#1)+(\i:5.2)$);
  }
  \foreach \i  in {90,91.14591559026,...,449} {
    \draw[line width=0.2mm, brown!100] ($(#1)+(\i:4.9)$) -- ($(#1)+(\i:5.1)$);
  }
}

\begin{document}
\pagestyle{empty}

\begin{tikzpicture}[remember picture, overlay,  x=1cm, y=1cm]
  \coordinate (SW) at (current page.south west) ;
  \coordinate (NE) at (current page.north east) ;
  \coordinate (HD) at ($(NE) - (1.,1.)$);
  \coordinate (BG) at ($(SW) + (1.,1.)$);
  \coordinate (milieu) at ($(BG)!0.5!(HD)$);
  \coordinate (centre) at ($(milieu) - (7.,0.)$);
  \coordinate (NG) at ($(BG)+(0,1)$);
  \coordinate (ND) at (HD |- NG);
  \coordinate (MG) at ($(NG)+(0,1)$);
  \coordinate (MD) at (HD |- MG);
  
  
  
  
%  \begin{scope}[shift={($(BG)+(3,3)$)}]
  \begin{scope}[shift=(centre)]
    
    \millimetrique{MG}{HD}
    \logametrique{BG}{ND}{10}{0}{0.3}{black}
    
    \cercle{centre}
    \draw[->] ($(centre)-(0,7)$) -- ($(centre)+(0,7)$) node[above] {$\mathcal{R}$};
    \draw[->] ($(centre)+(6,0)$) -- ($(centre)-(6,0)$) node[left] {$\mathcal{I}$};
    
    \node[right] at ($(HD)+(-2,0.5)$){Calque ULAtrice};
    \node[right] at ($(BG)+(-0.5,-0.5)$){ULA: Uniform Log Arithmetic};
    \node[right] at ($(BG)+(23,-0.5)$){Utilise les  Logs Andouille ! (ULA)};
    
  \end{scope}
\end{tikzpicture}

\newpage
\pagestyle{empty}

\begin{tikzpicture}[remember picture, overlay,  x=1cm, y=1cm]
  \begin{scope}[shift={($(BG)+(1,3)$)}]
    
    \millimetrique{NG}{MD}{5}{5}
    \logametrique{MG}{HD}{5}{10}{0}{brown}  
    \logametrique{BG}{ND}{10}{0}{0}{black}
\end{scope}


\end{tikzpicture}

\end{document}