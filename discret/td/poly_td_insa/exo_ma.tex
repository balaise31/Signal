


 \section{MA : systèmes Mooving Average à moyenne glissante.}


 

\graphe{0.6\textwidth}{../../poly/tikz/ma_ab.png} 

Les cercles sont des multiplieurs, l'opérateur T est le retard unitaire, le cercle avec un + est un additionneur.



\question{Q1~: Bloc $\rightarrow$ Récurrence ou Équation différentielle} 
\refWeb{Vidéo MA1 : Bloc → récurrence}{https://youtu.be/QRiAfbWLMw8?t=107}


Passez du schéma bloc à l'équation aux différences (récurrence) que vous mettrez sous forme :  
\begin{itemize}
\item $y[k+1]=\ldots$  
\item $y[k]=\ldots$  
\end{itemize}
Simplifiez les expressions et dites de quel ordre est ce système.  



\question{Q2~: Récurrence $\rightarrow$ Fonction de transfert $G(z)$} 
\refWeb{Vidéo MA1 : Récurrence → G(z)}{https://youtu.be/QRiAfbWLMw8?t=479}

En se rappelant que $z^{-1}$ est le système/signal associé à
l'opérateur retard $T$, donnez la fonction de transfert en Z en
partant de chaque récurrence :
\begin{itemize}
\item $y[k]=\ldots \implies Y(z) = \overbrace{\ldots}^{G(z)} \;.\; X(z) $  
\item $y[k+1]=\ldots \implies z.Y(z) =  \overbrace{\ldots}^{G(z)} \;.\; z.X(z)$
\end{itemize}



\question{Q3~: Caractérisation et propriétés (LTI, causal,  \dots)}

Identifiez les trois types d'opérateurs utilisés et dites pour chacun
si ce sont des systèmes LTI (Linéaire à Temps Invariant), des systèmes
causaux.  Conclure si le système entier est lui-même LTI, causal.

Le système est-il~:
\begin{itemize}
\item auto-régressif ?
\item récurrent ?
\item à moyenne glissante (mooving
  average) ?
\item bouclé ?
\end{itemize}





\question{Q4~: G(z) $\rightarrow$ Réponse Impulsionnelle (RIp) }
\refWeb{Vidéo MA2 : Transformée inverse de H(z)}{https://youtu.be/sKBlBBVbRWw?t=252}

Utilisez la fonction de transfert $Y(z) = G(z) . X(z)$ pour trouver la
transformée de la réponse impulsionnelle
$H(z) = G(z) . \tZde{\vec{\delta_0}}$.

Utilisez la transformée inverse pour retrouver la réponse
impulsionnelle temporelle $\vec{h} = \tZiDe{H(z)}$

\begin{remarque}
  Rappelez-vous comment un système/opérateur $X(z)$ (à base des
  opérateurs cités en Q3) est associé à un signal unique
  $\vec{x}: k\mapsto x[k]$ temporel.

  Indice~: signal RIp $\to$ système. Gros indice~:
  $\vec{x}=X (z)\!\left\{\vec{\delta_0}\right\}$

  Utilisez la phrase \og{}Une impulsion unité en entrée d'un gain de 1
  produit une impulsion unité\fg{} pour en déduire $\tZde{\vec{\delta_0}}$~: la transformée en
  $z$ de l'impulsion unité $\delta_0$.

  Traduisez la définition de la fonction de transfert
  $Y(z) = G(z) . X(z)$ en un schéma bloc qui explique comment
  produire un signal de sortie temporel $\vec{y}: k \mapsto y[k]$ à
  partir d'une impulsion unité. Indice~:
  $\vec{y}=G(z)\left\{\overbrace{ X(z)\!\left\{ \vec{\delta_0} \right\}}^{\vec{x}} \right\}$

  L'association $X (z)\!\left\{\vec{\delta_0}\right\}=\vec{x}$ définit
  la transformée $X(z) = \tZde{\vec{x}}$. La transformée inverse $\tZiDe{H(z)}=\vec{h}$
  se trouve, sans calcul analytique à faire !, par l'association inverse.

  Indice~: Système  $\to$ RIp. Gros indice $\vec{h} = H(z)\!\left\{\vec{\delta_0}\right\}$

\end{remarque}

À partir de la réponse impulsionnelle concluez si le système est
FIR (Finite Impulse Response) ou IIR (Infinite Impulse response) ?


\question{Q5~: Schéma Bloc MA $\rightarrow$ RIp} \refWeb{Vidéo MA2 : RIp et Causalité}{https://youtu.be/sKBlBBVbRWw?t=523}

Donnez la réponse impulsionnelle des systèmes~: \og{} gain de a\fg{} ; \og{} gain de b\fg{} ; d'un retard unitaire $T$.

Rappelez la relation temporelle entre la sortie $\vec{y}$ d'un système de RIp $\vec{h}$ pour une entrée $\vec{x}$.

Dans le schéma bloc, remplacez l'entrée $x[k]$ par l'impulsion
$\vec{\delta_0}$ et propagez le signal dans le schéma bloc jusqu'à la
sortie (par exemple la sortie du gain $b$ donnerait
$\vec{\delta_0} \star (b.\vec{\delta_0}) = b.\vec{\delta_0}$).

Que devient la relation $Y(z) = G(z) . X(z)$ en schéma bloc puis en
temporel $\vec{y} = \dots$



\question{  Q6 }
Retrouvez la propriété de causalité de la Q1 mais à partir de la RIp $h$. 


\question{  Q7 }

Quelle propriété doit vérifier la RIp pour que le système soit stable
? Est-ce une condition nécessaire et suffisante ?

Concluez si ce système est BIBO stable à partir de sa RIp.

BONUS : démontrez que $\Sigma h$ ACV $\implies$ H stable

\question{  Q8 }

Donnez un schéma bloc en transformée en Z sous forme simplifiée Directe I (chaîne de retards puis combinaison linéaire). Retrouvez de quel ordre est ce système.

\question{  Q9 }
On appelle BIBO la propriété de stabilité : "Bounded Input Bounded Output"

A-t-on :

\begin{itemize}
\item  $MA \implies FIR$ ?  
\item $FIR \implies BIBO$ ? Ou $BIBO \implies FIR$ ? Ou $BIBO \iff FIR$ ?  
\item $MA \implies FIR \implies BIBO$ ?  
\item $BIBO \iff FIR \implies MA$ ?  
\end{itemize}

%%% Local Variables:
%%% mode: latex
%%% TeX-master: "poly_td"
%%% End:
