\section{ARMA~: Auto Regressive Mooving Average}


\begin{tikzpicture}[outer sep=auto]

	\matrix (graphe)[ row sep=1cm, column sep=1cm, inner sep=0.5cm]{ 	
	\node[dspnodeopen]  (x) {$x[k+1]$};																&
	\node[dspnodefull]  (tx) {};													 							&
	\node[dspsquare]  (Tx) {T};																				&
																																	&
	\node[dspsquare ]  (Ty) {T};																			&
	\node[dspmultiplier, dsp/label=  above]  (a) {$a$};								&
	\node (coin)	 {}		;																							&
																																	&
																																	\\
																																	&
																																 	&
																																	&
	 \node[dspadder ]  (S) {};																				    &
																																    &
																																    &
	\node[dspnodefull]  (ty) {};																				&
	\node[dspnodeopen,  dsp/label=above]  (y) {$y[k+1
]$};  							&\\
	};
	\draw[dspflow] (x) -- (tx) ;
	\draw[dspflow]  (tx)  -- (Tx);
	 \draw[dspflow, rounded corners=1cm] (Tx) -|  (S);
	\draw[dspflow, rounded corners=1cm] (tx)  |- (S);
	\draw[dspflow, rounded corners=1cm]  (ty) |- (a);
	 \draw[dspflow] (a) -- (Ty);
	\draw[dspflow, rounded corners=1cm] (Ty)  -| (S);
	\draw[dspflow] (S)  -- (ty);
	\draw[dspflow] (ty) -- (y);

      \end{tikzpicture}
      
\question{ Q1 - Manipulation de schéma bloc}
Repérez le(s) bouclages(s) (AR) et les moyennes glissantes (MA) et décomposez le schéma bloc en séparant clairement MA puis AR.

$\vec{x} \overset{MA}{\longrightarrow} \vec{w} \overset{AR}{\longrightarrow} \vec{y}$


\refWeb{Vidéo Schéma bloc MA-AR (s'arrêter avant AR-MA)}{https://youtu.be/TjNE5sKsKRM?si=83cs8c0a03OJ7mqv&t=805}

\question{ Q2 - Bloc -> Fonction de transfert}

On note l'opérateur retard $q=z^{-1}$. Donnez, en les exprimant d'abord avec des retards $q$ puis avec des avances $z$, les fonctions de transfert~:
\begin{itemize}
\item $MA(z)=\frac{W(z)}{X(z)}$, 
\item $AR(z)=\frac{Y(z)}{W(z)}$,
\item la fonction de transfert totale $G(z)=\frac{Y(z)}{X(z)} =  \underbrace{\frac{Y(z)}{W(z)}}_{AR(z)} . \underbrace{\frac{W(z)}{X(z)}}_{MA(z)} = \underbrace{\frac{W(z)}{X(z)}}_{MA(z)} . \underbrace{\frac{Y(z)}{W(z)}}_{AR(z)}$.   
\end{itemize}


\question{ Q3 - Stabilité - Pôles et zéros}
Identifiez les pôles et les zéros du système et concluez sur la stabilité en fonction de $a$.

\question{ Q4 - F. Transfert -> Equation aux Différences}
Donnez, à partir de $G(1/q)$, la récurrence  du système sous forme

$y[k+1]= \ldots y[k] + \ldots x[\ldots] \ldots x[\ldots]$

et vérifiez qu'elle correspond au schéma bloc.

\question{ Q5 - F.T. -> Réponse Impulsionnelle}
Trouvez la réponse impulsionnelle (on prendra $a=\frac{1}{2}$) en décomposant $G(z)$ en $MA(z)$ (numérateur FIR) puis $AR(z)$ (dénominateur IIR) :
\begin{itemize}
\item exprimez la réponse impulsionnelle du MA sous forme de deux
  impulsions
\item exprimez la réponse impulsionnelle du AR en réutilisant
  les résultat de l'exercice MA (1er ordre)
\item par linéarité du
  système, trouvez $h$ comme la somme des réponses AR à une somme
  finie d'impulsions issue du MA(FIR)
\end{itemize}

\begin{remarque}
N'oubliez pas que les systèmes sont causaux et que \emph{si l'on
  retarde un signal causal, il faut aussi retarder l'échelon de
  heaviside} : utiliser par exemple $a^{k-1}u[k-1]$ pour annuler la
réponse pour $k<1$
\end{remarque}

\question{ Q6 - Idem par division en puissances croissantes}
Vérifiez le résultat en faisant une division par les puissances croissantes de $q = z^{-1}$ de la fonction de transfert totale $G(z)$ prise avec une entrée impulsionnelle. 

Vous vérifierez ainsi sur les quelques premiers termes que la réponse de la Q5 fonctionne.

\question{ Q7 - RIp -> causalité et stabilité}
À partir de $h$ dites pourquoi le système est stable et causal. Vérifiez la cohérence avec la Q3.

\question{ Q8 - Manipulation de blocs -> forme canonique Directe II }

\refWeb{Vidéo ARMA: Forme Directe II}{https://youtu.be/TjNE5sKsKRM?t=844}

Nous avons $G(z) = MA(z).AR(z) = AR(z).MA(z)$.  
Redessinez le schéma block en séparant d'\textbf{abord le AR} puis le MA.  


$\vec{x} \overset{AR}{\longrightarrow} \vec{w} \overset{MA}{\longrightarrow} \vec{y}$


En permuttant le multiplieur avec le retard, regroupez les deux blocs mémoires pour n'en faire qu'un, dont l'entrée sera $w[k+1]$ et la sortie $w[k]$. 


Vous obtenez ainsi la \emph{forme Directe II} canonique (et optimale en termes de mémoire).

\question{ Q8 (Option) - Equa. Diff. -> Espace d'état}

Donnez l'ordre de ce système et la récurrence sous forme de
représentation d'état discrète avec pour vecteur d'état

${}^T\!W_k = {}^T\![w[k-N], \;w[k-(N-1)], \;\ldots, w[k]]$ de taille adéquate :  

\begin{equation*}
  \left\{
    \begin{array}{cccc}
      W_{k+1} &=& A . W_k &+ B. x[k] \\
      y[k] &=& C . W_k &+  D. x[k]
    \end{array}\right.
\end{equation*}


\question{ Q9 - F.T. -> Réponse harmonique (Bode)}

Nous prenons maintenant $a=1$. Donnez la réponse harmonique de ce
système. Pour cela passez de $AR(z)=AR(\frac{1}{q})$ à une
représentation en fréquence $AR(i\omega)$, puis pour $MA(z)$ et enfin
pour $G(z)$.
 

Dans chaque  expressions faites apparaître uniquement des termes en $e^{i\,2\pi\,\tilde{f}}$ où on note $\tilde{\omega}=\frac{\omega}{\omega_e}=\tilde{f}=\frac{f}{Fe}$ la fréquence nomalisée.


Dans chaque expression vérifiez:  
\begin{itemize}
 \item que la réponse hamonique en $f$ est $f_e$ périodique (et donc 1 périodique en $\tilde{f}$)
 \item que la réponse est celle d'un système réel, i.e., spectre de Hilbert : $G(-\tilde{f})=\overline{G(\tilde{f})}$
 \end{itemize}

 
 \begin{remarque}
 AStuce de \emph{l'arc moitiée} :
 Simplifiez bien les paires d'exponentielles ! C'est \emph{très fréquent en discret}:  

 $1 + e^{i.b} = e^{-i.b/2} (e^{i.b/2} + e^{-i.b/2}) = e^{-i.b/2}.2.\cos(b)$ et  

 $1 - e^{i.b} = e^{-i.b/2} (e^{i.b/2} - e^{-i.b/2}) = e^{-i.b/2}.2i.\cos(b)$ et  

 $e^{i.a} + e^{i.b} = e^{i.\frac{a+b}{2}} (e^{i.\frac{a-b}{2}} + e^{-i.\frac{a-b}{2}}) = e^{i.\frac{a+b}{2}}.2.\cos\left(\frac{a-b}{2}\right)$  

 $e^{i.a} - e^{i.b} = e^{i.\frac{a+b}{2}} (e^{i.\frac{a-b}{2}} - e^{-i.\frac{a-b}{2}}) = e^{i.\frac{a+b}{2}}.2i.\sin\left(\frac{a-b}{2}\right)$
 
 Come le dit M. Lebotlan : "Seul un Lycéen mémoriserait ces formules. 
  Alors qu'en maternelle ils savaient déjà factoriser par la moitié et puis faire comme Euler et puis c'est tout !"
\end{remarque}
Pour faire le tracé de la réponse harmonique faites des équivalents pour avoir les asymptotes :
\begin{itemize}
\item  un équivalent pour $\tilde{f} \to 0$ doit donner une constante
\item un équivalent pour $\tilde{f} \to \infty$ est impossible car la réponse est périodique ! 
\item la fréquence discrète la plus haute que l'on puisse imaginer est $\frac{F_e}{2}$ : faites donc un équivalent $\tilde{f} \to \frac{1}{2}$

\end{itemize}
  
   
%%% Local Variables:
%%% mode: latex
%%% TeX-master: "poly_td"
%%% End:
