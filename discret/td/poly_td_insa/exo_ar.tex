\section{AR : Système Auto-Régressif (AR)} 

La réponse impulsionnelle d'un système continu du premier ordre est~:

$$ Y(p) = \frac{1}{1+\tau. p} X(p) \quad \underset{\vec{x}=X(p)\left\{\vec{\delta_0}\right\}}{\longleftrightarrow} \quad \tau \vec{y} + \vec{\dot{y}} = \vec{x}\quad \underset{\vec{x}=\vec{\delta_0}}{\longleftrightarrow} \quad h(t) = e^{-t/\tau}.\Gamma(t)$$

Ce système de pôle $\widehat{p_c}=\frac{-1}{\tau}$ est stable s.s.i.
$h(t) \underset{t\to \infty}{\rightarrow} 0$ donc s.s.i. $\tau\succ 0$
(dans le cas général où $\tau \in \C$, $\tau\succ 0$ veut dire que la
partie réelle de $\tau$ est positive). Cela permet de retrouver le
fait qu'un pôle est stable s.s.i $\widehat{p_c}$ est dans le demi-plan gauche de
Laplace~: $\widehat{p_c}\prec 0$.


Nous voulons créer un système discret capable de
générer la même réponse impulsionnelle de durée infinie (asymptotique)~:

$$Y(z) = \frac{\dots}{\dots +\dots z} X(z) \quad \underset{\vec{x}=X(z)\left\{\vec{\delta_0}\right\}}{\longleftrightarrow} \quad  \dots \vec{y[\bullet]} + \dots. \vec{y[\bullet-1]} = \dots . \vec{x}\quad \underset{\vec{x}=\vec{\delta_0}}{\longleftrightarrow} \quad h[k] = h\left(k.T_e\right) = a^k.\Gamma[k]$$

Les pôles $\widehat{p_d}$ de cette fonction de transfert discrète pourraient, de même, donner un lieu géométrique des pôles discrets stables comme le demi-plan gauche du continu.

\question{Q1~: discrétisation de $h$} Discrétisez le signal $t\mapsto h(t)$, avec une période
d'échantillonnage temporelle $T_e$, pour obtenir le signal
$\vec{h} : k \mapsto h[k]=h(\overbrace{k\,T_e}^{t})$, et montrez qu'il
s'agit d'une suite géométrique $h[k]=a^k.u[k]$ où vous exprimerez $a$
en fonction de $\tau$ et $T_e$.

\question{Q2~: condition de stabilité en temporel avec $h$}
Représentez $h$ graphiquement pour $a=\frac{1}{2}$, puis $a=1$, puis
$2$, puis $a=-\frac{1}{2}$, puis $-1$, puis $-2$.

Déduisez une condition nécessaire et suffisante de stabilité BIBO du système H avec $a$ puis avec $\tau$ (Il y a une CNS de CVA des séries géométriques à utiliser) : 
\begin{itemize}
\item en discret, H BIBO stable $\iff \quad |a| \ldots $
\item de même en continu, H BIBO stable $\iff \quad \tau \ldots$  
\end{itemize}

Pour trouver l'équivalent du demi-plan gauche stable en discret, les 3 questions suivantes sont des
manières différentes de trouver la fonction de transfert du système
discret de réponse impulsionnelle du premier ordre $h[k]=a^k.\Gamma[k]$.

\question{Q3~: $h \quad \rightarrow  $ récurrence $\quad \rightarrow G(z) $}

\refWeb{Vidéo AR1 : Récurrence}{https://youtu.be/TjNE5sKsKRM?t=19}

Donnez l'expression sous forme de récurrence d'une suite géométrique
de raison $a$ et essayez d'identifier cette récurrence à celle d'une
récurrence d'un système discret d'entrée $x$ et de sortie $y$.


\begin{remarque}
  Une récurrence avec condition initiale est un \emph{système à
    mémoire} (avec dans notre cas $y[0]=1$ et $y[k<0]=0$) mais
\emph{sans entrée} (le système à entrée nulle est dit autonome)~:
 
 $y[k]=a.y[k-1] \quad \forall k>0$ avec condition initiale $y[0]=1$

 De manière équivalente, on construit un \emph{système sans mémoire}
 (à mémoire nulle $y[k\leq 0]=0$) mais \emph{avec entrée}
 $\vec{x}=\vec{\delta_0}$ pour établir la \emph{condition initiale
   (CI)} $y[0]=1$ :
  
 $y[k]=a.y[k-1] + x[k] \quad \forall k\in\mathbb{Z}$ avec C.I. nulles
 $y[k<0]=0$.
 
 
\end{remarque}


Si vous n'y arrivez-pas facilement, faites d'abord la question suivante.

Sinon, donnez la fonction de transfert $G_1(z)$ de cette récurrence et
faites le lien avec celle de la question suivante.


\question{Q4~[optionnelle]~: Schéma Bloc $\quad\rightarrow\quad$ récurrence $\quad\rightarrow\quad h[\bullet-1]$ } 

\graphe{0.8\textwidth}{../../poly/tikz/ar_o1.png}


Traduisez ce schéma bloc en récurrence, puis en fonction de transfert (notée $G_T(z)$).

Calculez-en la réponse impulsionnelle $h$ à partir de l'équation de récurrence.


\begin{remarque}
  
  \refWeb{Vidéo G(z) → RIp}{https://youtu.be/_Mrx9QQ-z5o?t=1366}
  Apprenez à faire une division par les puissances décroissante. Puis
  vérifiez les premiers termes de $h$ en faisant la division par les
  puissances décroissantes de~:
\begin{itemize}
\item $\frac{z}{z^2-az}$ (récurrence en $y[k+2]=\dots$)~; ou de
\item $\frac{1}{z-a}$ (récurrence en $y[k+1]=\dots$)~; ou encore de
\item $\frac{\zmu}{1-a\zmu}$ (récurrence en $y[k]=\dots$).
\end{itemize}
\end{remarque}
On doit obtenir notre suite géométrique avec un peu de retard ;-) ...


La fonction de transfert $G_1(z)$ de la Q3 donne la suite géométrique $h$ en RIp.

La fonction de transfert $G_T(z)$ donne cette suite avec un retard unitaire.

Donnez donc la relation $G_?(z) = z^{?}.G_?(z)$ qui exprime qu'un système est en avance sur l'autre.


\question{Q5~[Optionnelle]~: $h \quad \rightarrow \quad H(z) = G(z) $ }


La fonction de transfert $G_1(z)$ de notre système à pour réponse
impulsionnelle $h$ donc
$$H(z) = G_1(z).\underbrace{1}_{Z\{\delta_0\}}$$

Il en résulte que $G_1(z)=H(z)$ et donc $G_1(z)=\tZde{\vec{h}}$. On
peut donc trouver la fonction de transfert en calculant la transformée
de la RIp $\vec{h}$.

\begin{remarque}
  Tout signal se décompose dans le temps sous forme de somme d'impulsions unité~:

  $\vec{h}= h[0].\vec{\delta_0} + h[1].\vec{\delta_1} + h[2].\vec{\delta_2} + \dots$.

La transformée en Z de cette combinaison linéaire de retards d'impulsions unité est triviale~:  

$H(z)= h[0].\underbrace{1}_{\tZde{\vec{\delta_0}}} + h[1].\underbrace{\zmu}_{\tZde{\vec{\delta_1}}} + h[2].\underbrace{z^{-2}}_{\tZde{\vec{\delta_2}}} + \dots$.

Ce qui donne la formule analytique de la transformée en Z~:
$$\tZde{\vec{h}} = H(z) = \sum\limits_{j=-\infty}^{\infty} h[j] . z^{-j}$$
Dans la plupart des ouvrages, $z$ est un nombre complexe et le domaine
de convergence de cette série entière doit être donné~: valeur de
$z\in\C$ où la somme converge.

Dans ce cours, la variable $z$ est l'opérateur avance unitaire et on décrit simplement le système $H(z)$, qui produit $h$ en RIp, en  utilisant la décomposition de $\vec{h}$ dans le temps.

\end{remarque}

Dans notre cas, la suite géométrique temporelle $\left(h[k]\right)$ devient une suite géométrique de l'opérateur $z$.

Retrouvez la formule de la somme d'une série géométrique pour
retrouver $H(z)$ sous forme d'une fraction rationnelle en $z^{-1}$.

\begin{remarque}
  La somme partielle d'une suite géométrique se trouve facilement par effet télescopique~:

\begin{equation*}
  \begin{array}{cccccccc}
    S_n       &=&   q^0     &+q^1   & +\dots &  + q^{n} &  \\
    -q.S_n     &=&         &-q^1 & +\dots & - q^{n} &- q^{n+1}  \\\hline
    (1-q).S_n &=&   q^0  &       &         &       & -q^{n+1}
  \end{array}
\end{equation*}
Lorsque la raison $q$ de la suite est inférieure en module à 1, la
limite de $S_n$, lorsque $n\to \infty$, se simplifie car $q^{n+1}\to 0$.
\end{remarque}


Vous obtenez ainsi la fonction de transfert $H(z)=G_1(z)$ sous forme de fraction rationnelle. 
\begin{remarque}
  Apprenez à faire une division par les puissances croissantes et
  faites une division par les puissances croissante de $z^{-1}$ de
  $H(z)$ pour retrouver $H(z)$ sous forme d'une suite géométrique...


  \refWeb{Vidéo G(z) → RIp}{https://youtu.be/_Mrx9QQ-z5o?t=1366}  
\end{remarque}


\question{Q6~: conditions de stabilité avec $G(z)$}


Déterminez l'ordre, le pôle $\widehat{p_c}$ et les zéros de
$G(p)=\frac{1}{1+\tau.p}$ en fonction de $\tau$. 


Déterminez de même l'ordre, le pôle $\widehat{p_d}$ et les zéros de $G_1(z)$
en fonction de $a$ puis de $\tau$ (et de $T_e$ bien
sur). \textbf{Attention ! Il faut exprimer en fonction de $z$ et non
  $z^{-1}$}

On montre, en continu, qu'un pôle est stable $\iff h(t) \to 0 \iff \tau\succ 0 \iff \text{pôle continu} \prec 0$ et donc le fameux ``demi-plan gauche stable ''.

On a montré, en discret, que stable $\iff h[k] \to 0 \iff |a|<1$. Quelle est alors la région du plan complexe où les pôles discrets sont stables. 

Pour mémoriser ce résultat fondamental, dessinez à gauche le plan
complexe et la région des pôles stable en continu, en donnant la
fonction de transfert $g(p)$ et indiquant la valeur du pôle continu
$\widehat{p_c}$.  Puis indiquez à droite de ce schéma la fonction de transfert
$G_1(z)$, l'expression de son pôle discret $\widehat{p_d}$ et la région stable
des pôles en discret.  Quelle est la condition de stabilité en discret
en fonction des zéros et des pôles ?

\begin{remarque}
La relation $\widehat{p_d} = e^{-\widehat{p_c}.T_e}$ permet de faire le lien entre un
pôle oscillant amortis en continus et son équivalent (même RIp mais
échantillonnée) en discret. Cela est utile pour savoir où placer les
pôles lors du calcul d'une commande par placement de pôles.
\end{remarque}
\question{ Q5 }
Ce système est-il MA (moyenne pondérée glissante) ? AR (auto-régressif) ? Récursif ? Bouclé ?


\question{ Q7 }
Donnez un schéma bloc en transformée en Z sous forme Directe I (chaîne de retards et combinaison linéaire). 

Retrouvez de quel ordre est ce système.

\question{ Q8 }
On appelle BIBO la propriété de stabilité : "Bounded Input Bounded Output"

A-t-on :  
\begin{itemize}
\item $IIR \implies AR$ ?
\item $IIR \implies \overline{BIBO}$ ? Ou bien $ \overline{BIBO} \implies IIR$ ? Ou bien $\overline{BIBO} \iff IIR$ ?
\item $AR \iff IIR \implies \overline{BIBO}$ ?
\item $\overline{BIBO} \implies IIR \implies AR$ ?
\end{itemize}
%%% Local Variables:
%%% mode: latex
%%% TeX-master: "poly_td"
%%% End:
