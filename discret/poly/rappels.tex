\chapter{Les bases de \Fourier{}}
\label{chap:fourier}
Ce chapitre aborde les signaux et leurs transformées avec le point de
vue des espaces vectoriels. Les notions apprises pour les signaux
continus sont ainsi revues et étendues au signaux discrets.



\section{Signaux et espaces vectoriels}

La \figref{fig:espaces_vectoriels} illustre le cas d'un espace
vectoriel de dimension 3 et d'une relation possible avec les espaces
des fonctions continues et discrètes.

% TODO évoquer le 2D

\begin{figure}[htbp]
  \centering
  \graphe{0.9\textwidth}{espaces_vectoriels}
  \caption{Liens entre l'espace Euclidien $\R^3$ et les espaces de fonctions. La notions de sous espace vectoriel est illustrée par l'exemple des sous espaces pair (violet) et impair(rouge).}
  \label{fig:espaces_vectoriels}
\end{figure}


\begin{remarque}\remarqueTitre{Notation et rigueur}
  On apelle et notera indiféremment un signal signal $s$~; la fonction
  $s$ (ou dans le cas discret la suite numérique $(s_n)_{n\in\Z}$ sera
  simplement notée $s$)~; et le vecteur $\vec{s}$ de l'espace des
  signaux associé.

  On différencie, en notant $s[k]$, l'évaluation de la fonction
  discrète $s: \Z\to\C$ à l'instant $k$ de celle d'un signal continu
  $s: \R\to\C$ à un instant $t$ classiquement notée $s(t)$.
  
  Le segment discret est noté $\semiN{n}{m} \quad =\left\{k \in\Z \;|\; n\leq k < m\right\}$ différemment d'un segment continu $\semi{a}{b} \quad =\left\{t\in\R \;|\; a\leq t < b\right\}$

  La notation des espaces $\R^\R$ choisie est assez courante et
  s'explique par l'extension de la notation des produits carthésiens
  $\times$ car~:
  $$ \R^3 = \caCest{\R\times\R\times\R}{\text{3 fois}} = \left\{ (z_0, z_1, z_2) \quad|\quad a,b,c \in \R \right\}$$
  En indiquant que le nombre de produits est associé aux éléments $k$ d'un ensemble discret, et prenant cette fois-ci des valeurs complexes, on note~:
  
  $$ \C^3 = \C^{\left\{0, 1, 2\right\}} = \caCest{\C}{k=0}\caCest{\times\C}{k=1}\caCest{\times\C}{k=2} \quad\text{ avec } (\caCest{a_0, }{k=0}\caCest{a_1, }{k=1}\caCest{a_2}{k=2}) \in  \C^{\semiN{0}{3}} = \left\{ (z_0, z_1, z_2) \in \C^3 \right\}$$
  En prenant un ensemble de dimension infini dénombrable comme $\Z$  par exemple~:

  $$ \R^\Z = \ldots \times \caCest{\R}{k=-1} \caCest{\times\R}{k=0} \caCest{\times\R}{k=1} \times \ldots \quad\text{ avec } s=(\ldots, \caCest{s_{-1}, }{k=-1}\caCest{s_0, }{k=0}\caCest{s_1}{k=1}, \ldots) \in  \R^{\Z}= \Z\to\R$$
  Un élément est donc un n-uplet de dimension infini dénombrable qui
  est le graphe d'une application de $\Z\to\R$ et donc celui d'une
  suite numérique $\left(s_k\right)_{k\in\Z}$ que l'on appelle signal
  discret $s$. On note donc $s[1]$ l'évaluation de $s$ pour $k=1$ qui
  est la valeur $s_1$ représenté dans ce n-uplet. On assimilera donc
  ces deux notations $s_k=s[k]$, l'une indiquant le terme de rang $k$
  de la suite $s$ et l'autre l'évaluation de l'application $s$ avec
  l'argument $k$.

  La preprésentation d'un tel graphe se fait sous forme de graphe en
  \og{} batonnets\fg{} comme en bas à droite de la
  \figref{fig:espaces_vectoriels}

  
  La puissance du continu est obtenu en prenant l'ensemble infini
  indénombrable $\R$ ce qui donne~:
  $$ \C^\R = \ldots \caCest{\times\C}{t=0} \times \ldots \caCest{\times\C}{t=\pi} \times \ldots \quad\text{ avec } s=(\ldots, \caCest{z_0, }{t=0} \ldots, \caCest{z_\pi, }{t=\pi} \ldots) \in  \C^{\Z}=\R\to\C$$

Un élément de cet ensemble est donc un n-uplet de dimension infinie
indénombrable (remarquez qu'il y a une infinité indénombrable de
valeurs entre $z_0$ et $ z_\pi$ !). Ce n-uplet constitue le graphe
d'une fonction de $\R\to\C$ et donc celui du signal continu
$s$. Contrairement aux suites numériques du discret, il est rare de
voir noté $s_t$ une évaluation de cette fonction à l'instant $t$ et la
notation $s(t)$ sera toujours utilisée.
\end{remarque}


Ainsi que ce soit en continu avec le cas le plus général des fonctions
de $\R\to\C$, ou en discret celui des suites complexes de $\N\to\C$,
on manipulera les signaux comme des vecteurs et utiliserons les
notions de ~:
 \begin{description}
 \item[produit scalaire $\scal{\vec{f}}{\vec{g}}$]--- sera un scalaire
   complexe dont le module indiquera si deux fonctions $f$ et $g$ se
   ressemblent. Calculer le produit scalaire entre $f$ et $g$ avec
   différents retards $\tau$ appliqués à $g$ permet de définir la
   \emph{fonction de correlation}~;
 \item[norme de vecteur (au carré)
   $\|\vec{s}\|^2=\scal{\vec{s}}{\vec{s}}$]--- sera associé à l'énergie
   totale du signal, ou puissance moyenne dans le cas de signaux à
   support fini comme l'espace des fonctions périodiques etc.
 \item[bases vectorielles]--- permet de représenter des
   signaux par des signaux de base. Deux bases seront importantes~:
   une base temporelle faite de signaux localisés autour d'un instant
   mais diffus en fréquence~; et une base fréquentielle faites de
   signaux localisés en fréquence et diffus dans le temps.
 \item[projection sur un sous-espace]--- Revient à trouver un signal
   d'un sous espace vectoriel qui soit le plus proche d'u Porjeter sur
   un sous-espaces en enlevant des vecteurs d'une base correspond à un
   sous-échantillonnage du signal et ajouter des vecteurs à une base
   (avec des composantes nulles)
 \item[changement de base]--- correspond aux transformations de
   \Fourier{} (base temporelle vers fréquentielle) et aux transformée
   inverses. Dans le cas des signaux discrets bornés, la base étant
   finie on peut représenter la transformée \TFD{} (algorithme \FFT)
   par une matrice de passage. Le calcul de la transofrmée se fait par
   produit matriciel et l'inverse par le produit avec la matrice
   inverse.
\end{description}
 
% \begin{description}
% \item{signal vecteur} : un signal continu ou discret comme vecteur d'un espace vectoriel qui sera un espace de fonctions ou suites~;
% \item un produit scalaire sous forme d'intégrale (somme le discret)
%   d'un produit de fonctions ou suite ~;
% \item une norme issue de ce p.s. qui sera reliée à l'énergie ou
%   puissance moyenne du signal selon les espaces de fonction~;
% \item des bases temporelles constituées de signaux dont l'énergie est
%   concentrée autour d'un instant, et des bases fréquencielles où
%   l'énergie/puissance d'un vecteur sera concentrée autour d'une
%   fréquence~;
% \item une décompostion dans une base permettra d'obtenir les
%   coordonnées d'un signal qui seront les échantillons pour une base
%   temporelle, ou les coefficients de transformée pour le fréquenciel~;
% \item un changement de base passant d'une représentation à l'autre
%   sera une transformée de \Fourier{} ou son inverse~;
% \item une projection sur un sous espace vectoriel correspondra à
%   trouver dans un espace de signaux, le sgnal le plus proche d'un
%   autre en terme d'énergie/puissance (sorte d'optimisation aux
%   moindres carrés)~;
% \item \ldots
% \end{description}

\subsection{Transformations par changement de base}
La \figref{fig:espaces} mets en évidence l'analogie du changement de
base et d'une transformée.

\begin{figure}[ht]
  \centering
  \graphe{0.9\textwidth}{espaces}
  \caption{Changement de base d'un vecteur dans un espace Euclidien de
    $\R^3$ et transformation d'une fonction de $\R\rightarrow\C$ dans
    un espace de Hilbert.}
  \label{fig:espaces}
\end{figure}

Si l'on note $E$ un espace vectoriel des signaux qui pourrait être~:

\begin{itemize}
\item $L_2 = \left\{s \in \C^\R \quad|\quad \norme{s}_2 \in \R\right\}$ -- celui des
signaux continus à énergie finie pour les transformées de \Fourier{}
(TF)~;
\item $L_2^p$ les signaux continus et périodiques à puissance moyenne
  finie pour les séries de \Fourier{} (\sdf)~;
\item les signaux discret de carré sommable pour les transformées de \Fourier{}
  des signaux discrets (TFSD)~;
\item les signaux discrets périodiques pour les transformées de
  \Fourier{} discrètes (TFD) avec son algorithme de calcul optimisé
  \emph{Fast Fourier Transform} (FFT).
\end{itemize}

On peut obtenir des fonctions/suites de coordonnées dans deux bases incontournables~:
\begin{description}
\item[Espace temporel (souvent le primal) --- $\left(E, B_c\right)$
  ---] Un signal $\vec{s}\in E$ peut être décomposé dans une base
  temporelle canonique $B_c=\left(e_\tau\right)_{\tau\in\R}$ de
  dimension infinie indénombrable où chaque vecteur $e_\tau$ de la
  base est un signal continu. Souvent considéré en premier, cet espace
  est appelé \emph{l'espace primal}.

  Les coordonnées sont alors exprimées sous la forme d'une fonction
  $s:\R\to\C$ qui, pour chaque instant réel $\tau\in\R$ associé à un
  vecteur $\vec{e_\tau}$, indique l'amplitude $s(\tau)$ de sa
  composante.

  Si l'on a une base orthonormée, alors cette composante
  s'obtient en projetant $\vec{s}$ sur chaque vecteur $\vec{e_\tau}$
  de la base avec le produit scalaire~:
  $$ s : \application{\R}{\C}{\tau}{\scal{\vec{s}}{\vec{e_\tau}}=s(\tau)}$$

  Dans le cas du temps discret on a une base infinie dénombrable
  $B_c=\left(e_j\right)_{j\in\Z}$ de signaux discrets et les
  coordonnées sont~:
  $$ s : \application{\Z}{\C}{j}{\scal{\vec{s}}{\vec{e_j}}=s\b{j}}$$

\item[Espace fréquenciel (souvent le dual) --- $\left(E, B_w\right)$
  ---] Ce même signal $\vec{s}\in E$ peut être décomposé dans une
  base fréquencielle $Bw=\left(w_f\right)_{f\in\R}$ où chaque vecteur
  $\vec{w_f}$ de la base est un signal localisé en fréquence.

  Les coordonnées sont alors exprimées sous la forme d'une fonction
  $\hat{S}:\R\to\C$ qui, pour
  chaque fréquence réelle $f$ associée à un vecteur $\vec{w_f}$,
  indique l'amplitude $\hat{S}(f)$ de cette composante.

  On retrouve la formule de transformée en utilisant le produit
  scalaire ainsi~:
  $$ \hat{S} : \application{\R}{\C}{f}{\scal{\vec{s}}{\vec{w_f}}=\hat{S}(f)}$$
  Dans le cas des \emph{fréquences discrètes}, la base est infinie
  dénombrable $Bw=\left(w_n\right)_{n\in\Z}$ et les coordonnées sont~:
  $$ \hat{S} : \application{\Z}{\C}{n}{\scal{\vec{s}}{\vec{w_n}}=\hat{S}\b{n}}$$
  
\end{description}
Pour une physicienne, une fonction du temps n'est pas la même chose
qu'une fonction de la fréquence. Pour une mathématicienne ce sont
toutes les deux des fonctions de la variable réelle~!

Ainsi le signal $\vec{s}$ est associé à une fonction $s:\R\to\C$ dans
la base temporelle et à une fonction $\hat{S}:\R\to\C$ dans la base
fréquencielle. 

On peut s'abstraire et considérer $\hat{S}:\R\to\C$ dans la base
temporelle~: ce qui donnera un signal différent de $\vec{s}$. Et aussi
considérer $s:\R\to$ dans la base fréquentielle ce qui donnera encore
un autre signal différent.

\begin{remarque}\remarqueTitre{Notation et rigueur}
  On peut alors être tentée d'écrire $s(t)$ et $\hat{S}(t)$ pour
  indiquer que ces fonctions sont des coordonnées dans une base
  temporelle. Mais c'est une erreur de typage car $s(t)\in\C$ dans le
  cas général et on veut parler de fonctions de $\C^\R$. La fonction
  se note $s$, son évaluation en un point $s(t)$ et si l'on veut
  évoquer la fonction en insistant qu'elle est dans une base
  temporelle on peut noter $t\mapsto s(t)$.

  Cette notation est lourde, on utilisera la notation \emph{assez rare
    et scpécifique} suivante qui remplace la variable $t$ qui pourrait
  aussi bien être $x$ ou $f$ (on parle de variable muette) par le
  symbole $\bullet$. Ainsi on note :
  $$ s = t\mapsto s(t) = f \mapsto s(f) = s(\bullet)$$
\end{remarque}


Une transformée de \Fourier{} est donc l'opérateur identité dans
l'espace vectoriel des signaux E qui consiste juste à changer de
système de coordonnées.
$$\F := I_d :  \application{E}{E}{\vec{s}}{\vec{s}}
$$

En revanche pour ce qui est des fonctions
(coordonnées) associées, une transformée est une application linéaire
qui transforme une fonction en une fonction~:

$$\F : \application{(E,B_c)}{(E,B_w)}{s : t\mapsto \;\scal{\vec{s}}{\vec{e_t}}=s(t)}{\hat{S} : f \mapsto \; \scal{\vec{s}}{\vec{w_f}}=\hat{S}(f)}
$$

\subsection{Projection par échantillonnage}

La base fréquencielle de \Fourier{} permet au moins de décrire
l'espace des fonctions de carré sommable (noté $L_2$) dont la
dimension est infinie indénombrable et possède la puissance du
continu. La décomposition d'un signal dans cette base constitue la
\TF.

\begin{definition}
L'\emph{échantillonnage} (on dira \emph{decimation} dans le cas
d'espaces de puissance discrète) est un type de projection d'un espace
de dimention infinie dans un espace de dimention inférieure. Il
consiste à prélever un sous-ensemble de vecteurs d'une base de manière
régulière~: avec un espacement constant nommé \emph{période
  d'échantillonnage} et parfois \emph{résolution}.
\end{definition}

\begin{remarque}
  On peut échantillonner aussi bien~:
  \begin{itemize}
  \item des bases temporelles continues et on parlera
    d'échantillonnage de période $T_e\in\R$ pour obtenir un signal
    primal discret~;
  \item des bases temporelles discrètes où on nommera
    \emph{décimation} l'opération avec un paramètre $M\in\N$ nommé
    \emph{facteur de décimation} (on garde un point tout les M points)
    pour obtenir un dual discret~;
  \item des bases fréquentielles continues et on parlera
    d'\emph{échantillonnage en fréquence} avec une période notée $F_0$
    (qui évoque les \sdf{}) et parfois $\Delta_f$ pour évoquer la
    \emph{résolution fréquencielle}~;
  \item des bases fréquentielles discrètes et on parlera de
    \emph{décimation en fréquence}.
\end{itemize}
\end{remarque}

On peut discrétiser la base de \Fourier{}
$\p{t\mapsto e^{i\,2\pi\,f\,t}}_{f\in\R}$ qui est continue en temps et en fréquences selon le temps, la fréquence ou les deux~:
\begin{description}
\item[temps discret] où la variable $t\in\R$ sera remplacée par une
  variable discrète $k\in\N$ avec la relation $t=k\,T_e$ où $T_e$ est
  la valeur qui sépare deux échantillons en temps est nommé
  \emph{période d'échantillonnage} et parfois \emph{résolution
    temporelle} notée $\Delta_t$ à ce moment.
\item[frequences discrètes] où la variable $f\in\R$ sera remplacée par
  une variable discrète $n\in\N$ avec la relation $f=n\,\Delta_f$ où
  $\Delta_f$ est la valeur qui sépare deux échantillons en fréquence
  est nommé \emph{résolution fréquentielle}. On peut noter $F_0$ cette
  même \emph{période d'échantillonnage en fréquence} pour évoquer le
  cas des \sdf.
\end{description}

\begin{quizz} ---  On note la fonction $w(t,f)=e^{i2\pi\,f\,t}$
  \begin{description}
  \item[Q1 - Typage~:] la fonction $w$ est un vecteur de l'espace~:
    \begin{enumerate}
    \item $L_2$ : l'espace des fonctions à énergie finie
    \item $\C^\R$
    \item $\R^2\to\C$
    \item ${\R^2}^\C$ mais ça fait bizarre
    \end{enumerate}
  \item[Q2 - Discret temporel:]~: je dispose d'un enregistrement audio 256 kb/s en 8bit~:
    \begin{enumerate}
    \item je suis à temps discret car en 8bit la résolution temporelle est $T_e=1/2^8$~; 
    \item temps discret et ma période d'échantillonnage est
      plus rapide ou égale à $T_e=\frac{256}{8}\; kHz$~;
    \item une base fréquentielle peut être~: $\left( k\mapsto e^{i2\pi\,f\,k T_e}\right)_{f\in\R}$
    \item un vecteur de la base fréquencielle peut être ~: $w_n= k\mapsto e^{i2\pi\,n F_0\, k T_e}$
    \end{enumerate} 
  \item[Q2 - Discret frequenciel:] ~: un vecteur d'une base fréquencielle discrète pourrait s'écrire
    \begin{enumerate}
    \item $w_n\b{k} = \sin\p{2\pi\,n F_0\,k T_e},\; k\in\Z$ 
    \item $w_f = f\mapsto\sin\p{2\pi f\,k\,T_e} = \sin\p{2\pi \bullet \,k\,T_e}$ 
    \item $w_f = k\mapsto\sin\p{2\pi f\,k\,T_e} = \sin\p{2\pi f \,k\,T_e}$ 
    \item $w_n = t\mapsto \sin\p{2\pi\, \frac{n}{T_0}\,t}$
    \end{enumerate}
  \end{description}
\end{quizz}


Selon la dimension continue (infinie indénombrable) ou discrète
(infinie dénombrable ou finie) des espaces primal et dual, on utilisera
différents produis scalaires pour effectuer différentes transformées
entre primal et dual.

L'appelation primal/dual est utilisée car elle exprime le fait qu'il
existe une opération permettant de passer de primal au dual et que
cette \emph{même opération} appliquée cette fois au dual permet de
revenir au primal~: comme appliquer deux fois un conjugué à un nombre
complexe, ou deux fois la transposée à une matrice.

La section \secref{sec:dualite} montre que la transformée de \Fourier{} et sa
version discrète sont toutes deux des opérateurs bi-duaux car~:
$$\ldots \;\toOp{\F}\; s \;\toOp{\F}\; \hat{S} \;\toOp{\F}\; s\p{-\bullet} \;\toOp{\F}\; \hat{S}\p{-\bullet} \;\toOp{\F}\; s \;\toOp{\F}\; \ldots $$

et donc $\quad\ldots \;\toOp{\F\circ\F}\; s \;\toOp{\F\circ\F}\; s\p{-\bullet} \;\toOp{\F\circ\F}\; s \;\toOp{\F\circ\F}\; \ldots$

L'opérateur $\F^2$ transforme donc un signal $s$ en un signal inversé,
appelé \emph{antipodal}, $s\p{-\bullet}$ qui est le dual de $s$. De
même le dual de $\hat{S}$ est son antipodal $\hat{S}\p{-\bullet}$.

On dit que $s$ est le dual de $\hat{S}$ un peu par abus.

\begin{remarque}\remarqueTitre{Pourquoi des signaux complexes alors
    qu'ils sont toujours réels ?}
  Cela permet justement d'appliquer plusieurs fois un même changement
  de base, comme la transformée de \Fourier{} à un signal. Ainsi si
  l'on considère uniquement des signaux de $\RR$

  $$ \application{\R^\R}{\C^\R}{f}{\hat{F}=T\{f\}} \eempile{\not\subset}{\neq} \application{\R^\R}{\C^\R}{g}{\hat{G}}$$

  Car la transformée d'un signal réel est dans le cas général un signal à valeur complexe ($a(f)$ et $b(f)$). En considérant le cas général $\CR$ on peut composer les tansformées~:

  $$ \ldots\application{\C^\R}{\C^\R}{f}{\hat{F}=T\{f\}} \eempile{=}{=} \application{\C^\R}{\C^\R}{g=\hat{F}}{\hat{G}=s\p{-\bullet}} \ldots$$

  De plus il est pratique de considérer des bases imaginaires $B_w=\left(t\mapsto e^{i2\pi\,f\,t}\right)_{f\in\R}$, quitte à admettre des fréquences négatives, et avoir une seule fonction de coordonnées $f\mapsto c(f)\in\CR$.

  Alors qu'en restant en réel avec des fréquences positives il faut traiter une union de deux bases $B_w = \left(t\mapsto \cos\p{2\pi\,f\,t}\right)_{f\in\R^+} \quad \bigcup \quad \left(t\mapsto \sin\p{2\pi\,f\,t}\right)_{f\in\R^{\star +}}$ et deux fonctions de coordonnées $f\mapsto a(f)\in\R^{\R^+}$ et $f\mapsto b(f)\in\R^{\R^{\star +}}$ 

\end{remarque}
  



\section{Les bases temporelles et espaces}
Nous allons définir les différents espaces des signaux (fonctions) et
y associer une base vectorielle \og canonique\fg{} et parfois des
variantes. Comme ces vecteurs de bases seront choisi de manière à être
très localisés dans le temps, la majorité de l'énergie est localisée
autour d'un instant, les bases seront dites temporelles.

Nous cherchons donc une famille de fonctions $t\mapsto p_\tau\p{t}$
($p$ comme impulsion) localisées autour d'un instant $\tau$ en
continu~; et des suites $k\mapsto p_{j}\b{k}$ localisées autour de
l'échantillon de rang $j$ pour le temps discret.

Ainsi un vecteur $\vec{s}$ d'un espace vectoriel de signaux discrets $E$ sera décomposé dans la base temporelle par~:
$$ \vec{s} = \sum\limits_j s\b{j}\;.\;\vec{k \mapsto p\b{k-j}} \;=\; \sum\limits_j  s\b{j}\;.\;\vec{p\b{\bullet-j}} $$
où le coefficient $s\b{j}$ de la composante à l'instant $j$ est obtenu
par projection~:
$$ s\b{j} = \scald{\vec{s}}{\vec{p\b{\bullet-j}}}$$

Dans le cas continu nous avons~:
$$ \vec{s} = \int s\p{\tau}\;.\;\vec{t \mapsto p\p{t-\tau}}\; d\tau$$
où le coefficient $s\p{\tau}$ de la composante à l'instant $\tau$ est obtenu
par projection~:
$$ s\b{\tau} = \scal{\vec{s}}{\vec{t \mapsto p\p{t-\tau}}}$$

Le plus simple est de considérer une suite finie de nombres :
incontournable lorsque l'on traite le signal avec des processeurs
numériques.  La mémoire étant finie, on ne peut qu'utiliser des
mesures faites périodiquements dans le temps (temps discret) nommées
\emph{échantillons} et de plus sur une durée finie (nombre
d'échantillons fini).


\subsection{Espaces discrets de dimension infinie}
\label{sec:RN}
L'espace des signaux discrets réels (resp. complexes) à support infini
est l'ensemble des applications de $\Z$ dans $\R$ (resp. $\C$) noté
$\RN = \left\{s : \N \to \R \right\}$ (resp.
$\CN = \left\{s : \N \to \C \right\}$).

\begin{remarque}\remarqueTitre{Rappel sur les notationsaz}
Un élément de cet espace est donc une fonction d'une variable discrète
que l'on notera souvent $k$ lorsqu'il s'agit d'un indice du temps avec
la relation $t \leftrightarrow k\,T_e$, ou bien $n$ lorsque l'indice
est relié aux fréquences par $f \leftrightarrow n.F_0$. Une fonction
de la variable continue évaluée à un instant $t$ étant notée $f(t)$,
on utilisera des crochets et notera $f[k]$ une fonction de la variable
discrète évaluée en $k$.

Pour différencier la notation d'une fonction de celle de son
évaluation en un point car $f\neq f(t)$, on notera de manière allégée
$f\p{\bullet}$ la fonction continue $t\mapsto f\p{t}$ où $\bullet$
remplace la variable muette $t$. De même on différencie le scalaire
$f\b{k}$ de la fonction $k\mapsto f\b{k}$ en notant simplement une
fonction $f\b{\bullet}$.
\end{remarque}
On veut trouver une base permettant de décomposer un signal/vecteur
$\vec{f}$ quelconque avec des vecteurs/signaux $\vec{\omega_i}$ d'une
base qui sera dite temporelle si le signal est localisé autour d'un
instant. La fonction discrète la plus localisée dans le temps que l'on
puisse utiliser pour construire cette base est \emph{l'impulsion
  unitée} que l'on définit ci-dessous.
\begin{definition}
  \label{def:impulsion_unite}
  L'\textbf{impulsion unité}, notée $\delta_0$ ou simplement $\delta$,
  est le signal discret tel que :


  \begin{minipage}[l]{0.5\linewidth}
  $ \delta=\delta_0~:\application{\N}{\R \text{ ou } \C}{ k}{\delta_0\b{k}=\pparMorceaux{1}{\text{si } k=0}{0}{\text{sinon}}} $
\end{minipage}
\begin{minipage}[l]{0.5\linewidth}
\graphe{0.9\linewidth}{impulsion}  
\end{minipage}


L'\textbf{impulsion unité centrée en $a$} est notée $\delta_a$ et définie par~:


\begin{minipage}[l]{0.5\linewidth}
$\delta_a = \delta_0\b{\bullet-a}~: \application{\N}{\R \text{ ou } \C}{k}{\pparMorceaux{1}{\text{si } k=a}{0}{\text{sinon}}}$ 
\end{minipage}
\begin{minipage}[l]{0.5\linewidth}
\graphe{0.9\linewidth}{impulsion_a}
\end{minipage}
\end{definition}

\begin{remarque}
  Bien qu'utilisant le même symbole $\delta$, il ne faut pas confondre
l'impulsion unité discrète avec l'impulsion de \Dirac. L'impulsion
unité est un signal discret tout à fait classique d'amplitude égale à
$1$ ; alors que l'impulsion de \Dirac{} est une fonction généralisée ou
distribution, voir \chapref{sec:dirac}, d'amplitude infinie et de
poids unité.
\end{remarque}

Ainsi on peut définir l'espace vectoriel des fonctions discrètes en
donnant une base canonique.

\begin{definition}
  \label{def:signal_discret}
  --- L'\textbf{espace des signaux discrets $\RN$ ou $\CN$} ( \textbf{à support infini})

  Est l'espace vectoriel de
  dimension infinie dénombrable des suites réelles défini par la base
  vectorielle $B_{\Z}=(\delta_j)_{j\in\Z}$. Un vecteur/signal
  $\vec{s}$ de $\RN$ a donc des coordonnées avec une infinité de
  composantes :
  \begin{equation*}
    \vec{s}  = (\ldots,s\b{-1}, s\b{0}, s\b{1}, \ldots)_{B_{\Z}} = \quad \ldots + s\b{-1}.\delta_{-1} + s\b{0}.\delta_0 + s\b{1}.\delta_1 + \ldots =\quad \somme{j\in\Z}{}{s\b{j}.\delta_j}
  \end{equation*}

  \center{\graphe{0.8\linewidth}{signal_RN}}
\end{definition}

Cet espace est utile du point de vue théorique car il permet de
modéliser l'échantillonnage d'un signal continu. En revanche, en
pratique, il est impossible de stocker une quantitée infinie
d'échantillons dans la mémoire d'un système. On va donc utiliser une
fenêtre d'observation pour limiter cette quantité d'information.


\subsection{Espaces discrets à support fini ou périodique}
Ainsi un échnatillonnage de $N_0$ points réels (resp. complexes) est
un vecteur de $\R^{N_0}$ (resp. $\C^{N_0}$). On peut établir des bijections
entre ce vecteur et différents \emph{sous-espaces} de l'ensemble des signaux
discrets à support infini $\RN$.



\subsubsection{Signal discret borné :  $\RNborne$ ou $\CNborne$}
L'ensemble des signaux discrets bornés à valeurs réelles (resp. complexes) $\left\{s : \SN0 \to \R \right\}$ (resp. $\left\{s : \SN0 \to \R \right\}$) est défini uniquement pour $k\in\SN0$. On plonge ces fonctions dans $\RN$ (resp. $\CN$) avec un prolongement nul pour $k$ en dehors de l'intervale de définition (appelé parfois fenêtre d'observation naturelle).

\begin{definition}
  \label{def:signal_discret_borne}
  --- L' \textbf{espace des signaux discret borné $\RNborne$ ou $\CNborne$ (à support fini)} 

  Est un sous espace vectoriel noté $\RNborne$ (resp. $\CNborne$) des
  suites réelles $\RN$ (resp. suite complexes $\CN$) défini pas la
  base vectorielle $B_{N_0}^b=\left(\delta_j\right)_{j\in\SN0}$. Ce
  signal est donc, sans perte de généralité, nul pour tout $k<0$ et
  $ N_0\leq k$ et se décompose ainsi :
  \begin{equation*}
    (\ldots,0, s[0], \dots, s[N_0-1], 0, \ldots) \quad \leftrightarrow \quad s[0].\delta_0 + \dots s[N_0-1].\delta_{N_0-1} = \somme{0\leq j< N_0}{}{s[j].\delta_j}
  \end{equation*}

  \center{\graphe{0.8\linewidth}{signal_RNb}}
  
\end{definition}

Remarquons que l'indice $j$ est un indice indiquant un vecteur de la
base temporelle qui est différents de la variable muette $k$ que nous
utilisons pour désigner l'argument dune fonction discrète. Ainsi il
faut bien analyser la somme précédente décomposant un signal $s$ par~:
\begin{equation}
  s = \somme{0\leq j< N_0}{}{s[j].\delta_j}~: \application{\N}{\R \text{ ou } \C}{k}{\somme{0\leq j< N_0}{}{s[j].\delta_j\b{k}}={\somme{0\leq j< N_0}{}{s[j].\delta\b{k-j}}}}
\end{equation}

\begin{exemple}
  On associe un enregistrement de 3 échantillons discrets de valeur 1,
  2 puis 3 au vecteur $(1, 2, 3)\in\R^3$. On fait une bijection entre
  ce vecteur de $\R^3$ et le signal discret
  $s=1.\delta_0+2.\delta_1+3.\delta_2$ en échangeant la base canonique
  de $\R^3$ ayant par exemple pour vecteur $\vec{e_0}=(1,0,0)$ avec le
  signal/vecteur de la base canonique des signaux discrets borné
  $B_3^b=\left(\delta_j\right)_{0\leq j<3}$ correspondant :
  $\vec{w_0} = \delta_0$.

  Nous avons bien un sous-espace vectoriel de dimension 3 de l'espace
  des suites réelles (resp. complexes) avec
  $$\vec{s} = \vvvectBase{1}{2}{3}{B_3^b} = 1.\vec{w_0}+2\vec{w_1}+3\vec{w_2} = 1.\delta_0+2.\delta_1+3.\delta_2$$

   On obtient le signal :
  $$
  \begin{array}{llcccccccc}
     &1.\delta_0 & =(\ldots, &0 , &1, &0, &0, &0, &0, &\ldots)\\
    +&2.\delta_1 & =(\ldots, &0 , &0, &2, &0, &0, &0, &\ldots)\\
    +&3.\delta_2 & =(\ldots, &0 , &0, &0, &3, &0, &0, &\ldots)\\\hline
    =& s  & =(\ldots, &0 , &1, &2, &3, &0, &0, &\ldots)
  \end{array}
  $$
\end{exemple}

\begin{remarque}
  Si l'on veut décomposer un signal discret borné dans une base
  fréquentielle, avec des signaux étalés dans le temps mais localisés
  en fréquences, nous verons que toutes les fréquences $f$ du segment
  continu $\SFe$ sont nécessaires et que les composantes
  fréquentielles sont périodiques de période $F_e$ :

  nous obtiendrons \textbf{la transformée de \Fourier{} des signaux
    discrets (TFSD)} pour représenter un \textbf{signal discret borné}
\end{remarque}


\subsubsection{Signal discret périodique : $\RNper$ ou $\CNper$}
En prolongeant le signal de manière périodique, nous obtenons une
représentation fréquentielle plus légère qui sera aussi une fonction
discrète périodique (au lieu d'une fonction continue périodique dans
le cas des TFSD précédent).

Pour cela nous devons introduire la version périodique de l'impulsion unité.

\begin{definition}
  Un \textbf{peigne d'impulsions de période $N_0$} est le signal noté
  $\Sha$ (lettre cyrilique \emph{Sha}) définit par :
  
  \begin{equation*}
    \Sha_{N_0} = \sum\limits_{m\in\mathbb{Z}} \delta_{m.N_0}: \application{\Z}{\R \text{ ou } \C}{k}{\delta_0[k\mod{}N_0]=\pparMorceaux{1}{\text{ si } k = m.N_0,\quad m\in\Z}{0}{\text{ sinon}}}
\end{equation*}

\graphe{0.9\linewidth}{peigne}
\end{definition}

Nous pouvons utiliser les peignes d'impulsions comme base de signaux pour définir le sous-espace vectoriel des signaux discrets périodiques :
\begin{definition}
  \label{def:signal_discret_periodique}
  --- L' \textbf{espace des signaux discrets $N_0$-périodiques $\RNper$ ou $\CNper$}

  Est un sous espace vectoriel noté $\RNper$ (resp. $\CNper$) des
  suites réelles (resp. suite complexes) définit par la base
  vectorielle \\$B_{N_0}^p=\left(\Sha_{N_0}\b{\bullet-j}\right)_{0\leq
    j<N_0}$. Ce signal est donc de période
  $N_0$ et se décompose ainsi :
  \begin{equation*}
    s = s[0].\Sha_{N_0} + \dots + s[N_0-1].\Sha_{N_0}\b{\bullet-(N_0-1)} = \somme{0\leq j< N_0}{}{s[j].\Sha_{N_0}\b{\bullet-j}}
  \end{equation*}
 
  \graphe{0.9\linewidth}{signal_RNp}
\end{definition}

Ne pas confondre l'indice
$j$ de la base temporelle et la variable muette
$k$ que nous utilisons pour désigner l'argument dune fonction
discrète. Ainsi il faut bien analyser la somme précédente décomposant
un signal $s$ par~:
\begin{equation}
  s[k] = \somme{0\leq j< N_0}{}{s[j].\Sha_{N_0}\b{k-j}}
\end{equation}

\begin{exemple}
  
  On associe maintenant $(1, 2, 3)\in\R^3$ au signal discret
  périodique
  $s=1.\Sha_{3}+2.\Sha_{3}\b{\bullet-1}+3.\Sha_{3}\b{\bullet-2}$ qui
  est 3-périodique.

  Nous avons bien un sous-espace vectoriel de
  dimension 3 de l'espace des suites réelles (resp. complexes) avec
  $$\vec{s} = \vvvectBase{1}{2}{3}{B_3^p} = 1.\vec{w_0}+2\vec{w_1}+3\vec{w_2} = 1.\Sha_{3}+2.\Sha_{3}\b{\bullet-1}+3.\Sha_{3}\b{\bullet-2}$$

  On obtient le signal :
  $$
  \begin{array}{llcccccccc}
     &1.\Sha_{3}            & =(\ldots, &1 , &0, &0, &1, &0, &0, &\ldots)\\
    +&2.\Sha_{3}[\bullet-1] & =(\ldots, &0 , &2, &0, &0, &2, &0, &\ldots)\\
    +&3.\Sha_{3}[\bullet-2] & =(\ldots, &0 , &0, &3, &0, &0, &3, &\ldots)\\\hline
    =& s  & =(\ldots, &1 , &2, &3, &1, &2, &3, &\ldots)
  \end{array}
  $$
\end{exemple}

\begin{remarque}
  Si l'on veut décomposer un signal discret $N_0$-périodique dans une
  base fréquentielle, nous verons que seul $N_0$ fréquences discrètes
  du segment $\SFe$ sont nécessaires et que les composantes
  fréquentielles sont périodiques de période $F_e$ :


  Nous obtiendrons \textbf{la transformée de \Fourier{} discrète (TFD)}
  pour représenter un \textbf{signal discret périodique}. Un algorithme
  optimisé de calcul de la TFD est la \textbf{Fast Fourier Transform
    (FFT)}
\end{remarque}

\subsection{Signal discret périodique pair~: $\RNdctII$}
Deux inconvénients du  prolongement périodique précédant sont :
\begin{itemize}
  \item une parité quelconque d'un signal  réel fait que sa transformée est complexe,
  \item le signal n'est pas forcément continu au point de raccordements périodique
\end{itemize}
Il en résulte une transformée de signaux réels qui est imaginaire et l'introduction de composantes haute fréquences due à l'introduction d'une discontinuité périodique.

\paragraph{Prolongement pair en continu~: CT -- \emph{Cosine Transform}}
En continu, le développement en série de cosinus (CT pour Cosine
Transform) est facile à définir car :il y n'y a qu'un seul
prolongement pair de période double d'un signal à support borné
$\ST0$. C'est donc une version réelle des SdF $c(n)$ où l'on prend une
période double avec :
$$CT[n] = a(\frac{n}{2}) = \frac{2}{2\,T_0}\int\limits_0^{2\,T_0} f_{\text{paire}}(t) . \cos\left(2\pi n \frac{t}{2.T_0}\right) dt \overset{\text{parité}}{=} \frac{2}{T_0}\int\limits_0^{T_0} f(t) . \cos\left(2\pi n \frac{t}{2.T_0}\right) dt$$

\textbf{Il n'y a pas de $b(n)$} ! Et tout est réel, en revanche la
résolution fréquentielle est double :$\frac{F_0}{2}$

\paragraph{Prolongements pairs en discret~: DCT -- \emph{Discrete Cosine Transform}}

En discret pour prolonger la suite bornée (a, b, c, d) par exemple, il
y a deux choix à faire à chaque borne :
\begin{itemize}
  \item $a$ est à l'instant 0 et donc non répété : $\ldots b,a,b,\ldots{}$
  \item $a$ est répété et donc à l'instant $\frac{T_e}{2}$ : $\ldots, b, a, a, b, \dots$
  \item $d$ est non répété et donc à $(N-1).T_e$ de l'échantillon a : $\ldots, c, d, c, \dots$ 
  \item $d$ est répété et donc à l'instant $(N-1).T_e + \frac{T_e}{2}$ de a : $\ldots, c, d, d, c, \dots$    
\end{itemize}

La \figref{fig_DCT_II} illustre un prolongement pair dit de type II où toutes les valeurs sont répétées de manière paires.
\begin{figure}[htbp]
  \centering \graphe{0.8\linewidth}{signal_DCT_II}
  \caption{Un prolongement pair possible en discret dit de type II~:
    exemple donnant la transformée DCT-II. La parité s'obtient en
    décalalant l'axe temporel avec la reltation $t=k\,T_e$ qui devient
    $t=k\,Te + \frac{T_e}{2}$ }
  \label{fig:DCT_II}
\end{figure}


\begin{remarque}
En discret il y a 4 prolongements pairs donnant 4 \textbf{Direct Cosine
Transform} DCT car il y a 4 prolongements pairs ``aux deux bornes'' conduisant à 4
espaces pairs périodiques, et 4 prolongements pairs en ``a'' et impair en
``d''.

\begin{tabular}{lllll}
\hline
  Type de DCT & Prolongement (N=4) & Décalage de ``a'' & parité en ``d'' &
Période \\
\hline
Type-I & $(\ldots, a, b, c, d, c, b, \quad a, b, \ldots)$ & 0 & pair & 2.(N-1) \\
Type-II & $(\dots, a, b, c, d, d, c, b, a,\dots)$ & +1/2 & pair & 2.N \\
\hline
\end{tabular}

Le prolongement de type II est très utilisé dans les algorithmes de
compressions d'image et de son (mpeg, avi etc.) car la transformée
DCT-II se calcule de manière efficace avec le mêm algorithme \FFT{}
utilisé pour les \TFD{} et de plus le prolongement n'ajoute pas de
discontinuïté ce qui fait les composantes hautes fréquences
deviennnent négligeables. Ces dernières sont ignorées ou codées avec
une faible résolution au moment de la réduction de la taille du
stockage~: ce type de compression est appelé \emph{compression avec
  pertes} ou \emph{lossy compression}.
\end{remarque}




\section{Les espaces et produits scalaires associés}
Nous allons considérer des espaces de fonctions tantôt à variables
continues puis discrètes, et en même temps sur des supports infinis ou
bornés. Dans le cas de fonctions bornées (définies sur un intervalle $\semi{a}{a+T_0}$ en continu ou $\semiN{a}{a+N}$ en discret), on peut \textbf{toujours} prolonger cette fonction
en dehors du support de manière périodique, plutôt que par des zéros, car cela permet d'avoir une représentation en séries de \Fourier{} (\sdf{}) (cas continu) ou en Transformée de Fourier Discrète (\TFD{}) dans le cas discret.

Les produits scalaires pour différents espaces de fonctions sont définis et illustrés dans la \tabref{tab:scalaires} en prenant~:
\begin{itemize}
\item  en rangées du tableau les signaux de la \emph{variable continue} (intégrale continue) ou bien de la \emph{variable discrète} (somme discrète)~;
\item en colonnes du tableau les \emph{supports infinis} (de $-\infty$ à $\infty$) ou bien support \emph{périodique/borné} (de $0$ à $T_0$ ou $N$).
\end{itemize}

\begin{table}[!ht]
  \begin{tabular}{p{0.1\textwidth}|c|c}
    &Support infini    & Support fini ou périodique  \\\hline
    &  \begin{tabular}{c} \graphe{0.4\textwidth}{fonction_rc}\end{tabular}
                       &  \begin{tabular}{c} \graphe{0.5\textwidth}{fonction_rtc} \end{tabular}                           \\
    variable continue & $f:\R\rightarrow\C$    &  $f:\;[0,\,T_0[\, \mapsto \C $       \\
    &$\scal{f}{g}=\intDt{\R}{}{f(t).\conj{g(t)}}$ $\quad\deDim{\frac{V^2}{\Hz}}$ & $\scalp{f}{g}=\frac{1}{T_0}\intDt{0}{T_0}{f(t).\conj{g(t)}}$ $\quad\deDim{V^2}$ \\\hline
    &   \begin{tabular}{c} \graphe{0.4\textwidth}{fonction_nc}\end{tabular}
                       &  \begin{tabular}{c} \graphe{0.5\textwidth}{fonction_znc}    \end{tabular}                        \\
    variable discrète & $\N\rightarrow\C$    & $\;\semiN{0}{N_0}\;\rightarrow \C$        \\
    & $\scald{f}{g}=\somme{k\in\N}{}{f[k].\conj{g[k]}}$ $\quad\deDim{V^2}$& $\scaldp{f}{g}=\frac{1}{N_0}\somme{k=0}{N_0-1}{f[k].\conj{g[k]}}$  $\quad\deDim{V^2}$  
  \end{tabular}
  \caption{Les proxduits scalaires adaptés aux différents espaces de fonctions. Par clarté, on ne représente que le module de la fonction qui est dans le cas général complexe.}
  \label{tab:scalaires}
\end{table}
  
\begin{exercice}\exerciceTitre{Propriété de scalaire et norme dans le cas général}
  On aurait pu définir ces produits scalaires en ne prenant jamais le
  conjugué d'une fonction $g$ (ou en considérant des fonctions à
  valeurs réelles de manière à ignorer ce conjugué car $\conj{g}=g$).
  \begin{enumerate}
  \item Vérifiez dans le cas réel (sans conjugué) que le produit
    $\scal{f}{g}$ a les propriétés d'un produit scalaire, en déduire la
    norme induite $\|f\|^2=\scal{f}{f}$ et déterminer la dimension de
    $\|f\|^2$ : est-ce de la puissance ou de l'énergie, est-ce une
    valeur ou une densité ?
  \item Appliquez cette norme (toujours sans le conjugué) au signal imaginaire pur
    $f: t\mapsto i$. Quelle propriétée de la norme n'est pas respectée ?
  \item Refaites de même en prenant cette fois-ci les formules de
    \tabref{tab:scalaires} avec le conjugué de $g$ et vérifiez que
    cette propriété est vérifiée dans le cas général des fonctions à
    variables complexes.
  \item Vérifiez que $\scal{f}{g}=\conj{\scal{g}{f}}$ et que donc le produit scalaire est linéaire à gauche $\forall\lambda\in\C,\; \scal{\lambda\,f}{g}=\lambda\scal{f}{g}$ et \emph{à moitié linéaire} à droite $\forall \lambda\in\C,\; \scal{f}{\lambda\,g}=\conj{\lambda}\scal{f}{g}$
  \end{enumerate}
On comprend maintenant pourquoi, dans le cas général des fonctions à valeurs complexes, on utilise le conjugué dans l'expression des produits scalaires et pourquoi on parle de produit \emph{sesqui--linéaire} pour ces produits scalaires~: \emph{sesqui} en latin voulant dire \og{}un et demi\fg{} en latin.  
\end{exercice}

Le produit scalaire est très utile car il permet d'obtenir~:
\begin{itemize}
\item de mesurer des longueurs de signaux avec la norme induite par le produit scalaire $\norme{\vec{s}}=\scal{\vec{s}}{\vec{s}}$, et de mesurer des distances entre signaux avec la norme de la différence $\norme{\vec{u}-\vec{v}}$~;
\item de projeter un vecteur sur un autre ou sur un sous-espace
  vectoriel~: cela revient à minimiser une distance $P_v(u)=\min\limits_{x\in\vect{u}}\p{\norme{u-x}}$ par simple calcul direct~;
  
\item trouver les meilleures, au sens de la distance avec la norme
  engendrée, décompositions d'un signal $u$ sur une base de vecteurs
  données~: calculer des transformées de signaux.
\end{itemize}

La \tabref{tab:hilbert} montre le parallèle entre l'utilisation du produit scalaire sur des vecteurs et sur des signaux, chacune permet de retrouver des formules bien connues des \sdf{} et des \TF{}.

%\begin{landscape}
\begin{table}[!ht]
\renewcommand{\arraystretch}{1.4}
\begin{tabular}{p{0.1\textwidth}|p{0.32\textwidth}|p{0.6\textwidth}}
  & Euclidien fini & Espace de fonctions \\\hline
  Base   Ortho--normée & une base finie de vecteurs $$\B=\p{\vec{e_n}}_{n\in\ZN0}$$ normés $\|\vec{e_n}\|=1$ et orthogonaux $\scal{\vec{e_n}}{\vec{e_m}}=0$ & base dénombrable de fonctions $\p{\vec{w_n}}_{n\in\N}$ ou indénombrable $\p{\vec{w_f}}_{f\in\R}$ repérées par leur fréquences $f$ ou un indice $n$ associé~;  fonctions d'énergie unitaire $\|\vec{w_n}\|=1$ ou $\|\vec{w_f}\|=1$, et orthogonales $\scalp{\vec{w_n}}{\vec{w_m}}=0$ ou $\scal{\vec{w_{f}}}{\vec{w_{f'}}}=0$\\\hline
  
  Analyse &  décomposer un vecteursdans cette base en coefficients $V_n=\scal{\vec{v}}{\vec{e_n}}$ et en donner les coordonnées $$\caCest{\vecDs{V}{\B}}{\rightleftharpoons\vec{v}}=\vvvectBase{V_0=\scal{\vec{v}}{\vec{e_0}}}
            {\vdots}
            {V_{N-1}=\scal{\vec{v}}{\vec{e_{N-1}}}}{\B}$$
                   &  décomposer une fonction $\vec{u}$ en fréquentiel avec la transformée $U(f)$ ou avec les coéfficients $U(n)$ de la série~: $$U(f)=\scal{\vec{u}}{\vec{w_f}}=\scalint{u(t)}{w_f(t)}{t}$$ $$U(n)=\scalp{\vec{u}}{\vec{w_n}}=\scalpint{u(t)}{w_n(t)}{t}$$ \\\hline
  
  Synthèse &  recomposer un vecteur dans cette base $\vec{v}=\somme{k\in\ZN0}{}{\caCest{U_0}{\scal{\vec{v}}{\vec{e_0}}}}\,.\,\vec{e_0}$ \graphe{4cm}{projections} &  recomposer une fonction par transformation inverse de $U(f)$  ou recompostion de série $U(n)$~: $$\vec{u}(t) = \int\limits_{-\infty}^{\infty}{\caCest{U(f)}{\scal{\vec{u}}{\vec{w_f}}}\,.\,\vec{w_f}(t)\,\dt} $$ $$\vec{u}(t)=\sum\limits_{n\in\N}\caCest{U(n)}{\scalp{\vec{u}}{\vec{w_n}}}\,.\,\vec{w_n}(t) $$\\\hline

  Projeter avec Plancherel  &  calculer le produit scalaire de vecteurs par leurs composantes~: $$\scal{\vec{u}}{\vec{v}}=\scal{U_{\B}}{V_{\B}}=\Tr{\vecDs{U}{\B}}\,.\,\vecDs{V}{\B}$$ $$\caCest{\hhhect{U_0}{\ldots}{U_{N-1}}}{\Tr{\vecDs{U}{\B}}}\,.\,\vvvect{V_0}{\vdots}{V_{N-1}}$$
  % au lieu de calculer $$\scal{\vec{u}}{\vec{v}}=\|\vec{u}\|\,\|\vec{v}\|\,\cos\p{\alpha}$$
                   &  on peut calculer un produit scalaire (utile aux correlations et convolutions) à partir de sa transformée ou composantes de la série~: $$\scal{\vec{u}}{\vec{v}}=\scalint{u(t)}{v(t)}{t} = \scal{U}{V} =  \scalint{U(f)}{V(f)}{f}$$ $$\scalp{\vec{u}}{\vec{v}}=\scalpint{u(t)}{v(t)}{t} = \scaldp{U}{V} =  \scalpdint{U(k)}{V(k)}{k}$$
  \\\hline
  
  Normer avec Parseval  &  Calculer la norme en sommant les carrés des coordonnées~: $$\|\vec{u}\|^2=\|U_{\B}\|^2=\sum {U_n}^2$$ &  calculer la puissance moyenne par la transformée $u(f)$ ou en sommant celle des composantes fréquentielles $U(n)$~: $$\|\vec{u}\|^2=\|U\|^2=\int\limits_{-\infty}^{\infty}{|u(t)|^2\dt}=\int\limits_{-\infty}^{\infty}{|U(f)|^2\df} $$  $$\|\vec{u}\|^2=\|U\|_P^2=\frac{1}{T_0}\int\limits_{0}^{T_0}{|u(t)|^2\dt}=\sum\limits_{k\in\N}{|U(k)|^2} $$
\end{tabular}

\caption{Structure Euclidiene à structure de Hilbert}
\label{tab:hilbert}
\end{table}
%\end{landscape}

\section{Les bases fréquentielles : transformations}

En prenant la base des ondes complexes adaptée à chaque espace de
signaux (discrétisée ou non, sur un intervalle infini ou
borné/périodique), et en utilisant les produits scalaires adaptés, on
peut définir quatre types de transformations et leurs réciproques entre
un primal avec une base canonique purement localisée dans le temps (et
infiniment étendue en fréquence) et un dual composé d'une base d'ondes
purement localisées fréquentielles (et infiniment étendue en dans le
temps).

Le schéma ci-dessous résume ces transformées, leurs bases et les
produits scalaires associés à chaque transformation~:


\graphe{0.8\textwidth}{transformees.png}

\subsection{Base de la \TF{}}

La \TF{} (Transformée de \Fourier), ou \FT{} (\emph{Fourier
  Transform}) en anglais, s'applique aux fonctions continues et utilise
une base d'ondes complexes
$\Bf=\p{\caCest{t\mapsto e^{i2\pi\,f\,t}}{w_f}}_{f\in\R}$.

\begin{exercice}
Tentez de retrouver la formule de la transformée et son inverse et d'esquisser le schéma ci-dessous sans le regarder, en se rappelant juste que c'est une application de $$\R\rightarrow\C\;\;\operateur{\TF}\;\;\R\rightarrow\C$$ basée sur le produit scalaire continu noté $\scal{}{}$ avec la base continue $\Bf=\p{w_f}_{f\in\R}$
\end{exercice}

\graphe{\textwidth}{tf}

 On peut faire
l'analogie avec les espaces Euclidiens mais pas l'amalgame, car~:
\begin{itemize}
\item le produit scalaire $\scal{}{}$ est défini  dans le cas de fonctions de carré
intégrable, ou \emph{fonction à énergie finie}, que nous notons $\Leb{2}$,
\item les vecteurs de la base ne sont pas normés car de norme infinie~;
\item la base n'est pas finie, ni infinie dénombrable mais infinie indénombrable.
\end{itemize}

Mais lorsque l'on se place dans le cas de fonctions de carré
intégrable, ou fonction à énergie finie, que nous notons $\Leb{2}$,
l'espace est complet (les suites de Cauchy convergent) dont les sommes
infinies se comporte bien dans $\Leb{2}$~: c'est une espace de
Banach. De plus le produit scalaire associé à la norme 2 existe et on
a donc un espace de \Hilbert{} où la norme et le produit scalaire sont
des applications linéaires dans $\Leb{2}$. Comme il s'agit d'une espace
de dimension infinie, il ne suffit pas d'avoir une base de dimension
infinie pour couvrir tout l'espace, mais dans le cas de $\Leb{2}$ avec
la base $\Bf$ on montre que tout l'espace est engendré.

Bref ! Cela fonctionne tout comme un espace Euclidien sans en être un.

\begin{exercice}
Prendre la base $\Bf=\p{w_f}_{f\in\R}$ et utiliser la partie espace indénombrable de la \tabref{tab:hilbert} pour retrouver les formules de \Plancherel{} et \Parseval{}. 
\end{exercice}


\subsection{Base des \sdf{}}
Les \sdf{} (Séries de \Fourier{}), ou \FS{} (\emph{Fourier Series}) s'appliquent aux fonctions continues périodiques et   utilisent une base dénombrable $\Bf=\p{\caCest{t\mapsto W_{T_0}^{n\,t}=e^{i \frac{2 \pi}{T_0}\,n\,t}}{w_{T_0}^n}}_{n\in\N}$ avec $W_{T_0}=e^{i \frac{2 \pi}{T_0}}$.
\begin{exercice}
Tentez de retrouver la formule de la décomposition et recomposition en \sdf{} et d'esquisser le schéma ci-dessous sans le regarder, en se rappelant juste que c'est une application de $$\RT0\rightarrow\C\;\;\operateur{\sdf}\;\;\N\rightarrow\C$$  basée sur le produit scalaire continu périodique noté $\scalp{}{}$ et avec la base discrète $\Bf=\p{W_{T_0}^n}$.
\end{exercice}

\graphe{\textwidth}{sdf}

 On peut faire l'analogie avec les espaces Euclidiens mais pas l'amalgame, car~:
 \begin{itemize}
   \item le produit scalaire $\scalp{}{}$ est défini dans le cas de fonctions périodiques de carré intégrable, \emph{fonctions de puissance moyenne finie},  que nous notons $\LebP{2}$
 \item ce n'est pas un isomorphisme car, on passe d'un espace continu périodique à
   un espace discret~! La transformée inverse se fait avec le produit scalaire discret $\scald{}{}$
\item la base n'est pas finie, mais infinie dénombrable~;
\end{itemize}

Bref ! ela fonctionne un peu comme un espace Euclidien fini sans en être un...

\begin{exercice}
  Prendre la base  $$\Bf=\p{\caCest{t\mapsto \cos\p{\frac{2 \pi}{T_0}\,n\,t}}{\cos_n}}_{\!\!\!\!n\geq 1}\cup \p{\caCest{t\mapsto \sin\p{\frac{2 \pi}{T_0}\,n\,t}}{\sin_n}}_{\!\!\!\!n\geq 1}\cup \p{t\mapsto 1}$$ et voir que l'on retrouve les formules des coefficients $a\b{n}$, $b\b{n}$ et $a_0$ \textbf{à un facteur 2 près !}
  
  Et oui ! La base n'est pas normée car un rapide calcul montre que la norme des vecteurs vaut~$\frac{1}{2}$ (on peut se rappeler que la valeur efficace d'un cosinus d'amplitude 1 est $\frac{\sqrt{2}}{2}$~; sa puissance moyenne sur une période est donc le carré de $\frac{\sqrt{2}}{2}$) 

  En prenant la base normée $$\Bf'=\p{\caCest{\sqrt{2}\cos_n}{\cos_n'}}_{\!\!\!\!n\geq 1}\cup \p{\caCest{\sqrt{2}\sin_n}{\sin_n'}}_{\!\!\!\!n\in\Z^*} \cup \p{t\mapsto 1}$$
  on obtient une définition des \sdf{} chère aux physiciennes~:
  \begin{equation}
    \label{eq:sdf_normee}
    s\p{t} = \left(
    \begin{array}{c}
      {\caCest{\scalp{s}{t\mapsto 1}}{a_0}\;1} \\
      + \somme{n=1}{+\infty}{\caCest{\scalp{s}{\cos_n'}}{a'\b{n}}\;\cos_n'(t)}  \\
      + \somme{n=1}{+\infty}{\caCest{\scalp{s}{\sin_n'}}{b'\b{n}}\;\sin_n'(t)}     
    \end{array}\right)
  = \left(
    \begin{array}{c}
      a_0  \\
      + \somme{n=1}{+\infty}{\caCest{\scalp{s}{\sqrt{2}\cos_n}}{\sqrt{2}\,a\b{n}}\;\sqrt{2}\,\cos_n(t)}  \\
      + \somme{n=1}{+\infty}{\caCest{\scalp{s}{\sqrt{2}\sin_n}}{\sqrt{2}\,b\b{n}}\;\sqrt{2}\,\sin_n(t)}
    \end{array}\right)
\end{equation}
Avec $\norme{t\mapsto 1}^2=1$, $\norme{\cos_n'}^2=\norme{\sqrt{2}\cos_n}^2=1$ et $\norme{\sqrt{2}\sin_n}^2=1$

On peut ne pas normer les vecteurs, et c'est le plus fréquent, mais introduire un facteur $2$ dans la formule de calcul des coefficients $a\b{n}$ et $b\b{n}$ qui n'apparait pas dans les coefficients $c\b{n}$~:
  \begin{equation}
    \label{eq:sdf_non_normee}
    s\p{t} = \left(
    \begin{array}{c}
      {\caCest{2\,\scalp{s}{t\mapsto 1}}{a_0}\;1} \\
      + \somme{n=1}{+\infty}{\caCest{2\,\scalp{s}{\cos_n}}{a\b{n}}\;\cos_n(t)}  \\
      + \somme{n=1}{+\infty}{\caCest{2\,\scalp{s}{\sin_n'}}{b\b{n}}\;\sin_n(t)}     
    \end{array}\right)
  = \left(
    \begin{array}{c}
      \frac{a_0}{2}  \\
      + \somme{n=1}{+\infty}{\caCest{2\,\scalp{s}{\cos_n}}{a\b{n}}\;\cos_n(t)}  \\
      + \somme{n=1}{+\infty}{\caCest{2\,\scalp{s}{\sin_n}}{b\b{n}}\;\sin_n(t)}
    \end{array}\right)
\end{equation}
Avec $\norme{t\mapsto 1}^2=1$, $\norme{\cos_n}^2=\norme{\sin_n}^2= \frac{1}{2}$.

\end{exercice}


\subsection{Base de la \TFSD{}}

La \TFSD{} (Transformée de \Fourier{} des Signaux Discrets), ou \DTFT{} (\emph{Discrete Time Fourrier Transform} en anglais, s'applique aux fonctions à variable discrète et utilise une base d'ondes complexes indénombrable $\Bf=\p{\caCest{k\mapsto W_{T_e}^{f\,k}=e^{i 2 \pi\,T_e\,f\,k}}{w_{T_e}^f}}_{f\in\SFe{}}$ avec $W_{Te}=e^{i 2 \pi\,T_e}$.

\begin{exercice}
Tentez de trouver la formule de cette \TFSD{}  et son inverse, d'esquisser le schéma ci-dessous sans le regarder, en pensant que c'est la \og{}duale\fg{} de la \sdf{}. Il s'agit d'une application de $$\N\rightarrow\C\;\;\operateur{\TFSD}\;\;\RFe\rightarrow\C$$ basée sur le produit scalaire discret noté $\scald{}{}$ avec la base continue $\Bf=\p{w_{T_e}^f}_{f\in\SFe}$
\end{exercice}

\graphe{\textwidth}{tfsd}


On peut difficilement faire l'analogie avec les espaces Euclidiens  car~:
\begin{itemize}
   \item le produit scalaire $\scald{}{}$ fonctionne dans le cas de suites discretes absoluement convergentes~;
 \item ce n'est pas un isomorphisme car on passe d'un espace discret à un espace continu périodique~! La transformée inverse se fait avec le produit scalaire continu périodique $\scalp{}{}$ (Attention la période dans l'espace des fréquences est $F_e$)~;
\item la base n'est pas finie, ni dénombrable  mais infinie indénombrable.
\end{itemize}


\begin{exercice}
  On admet pour le moment que la \TFSD{} d'un signal $s[k]$ quelconque est une fonction $S(f)$ de période $F_e$. On peut donc voir $S(f)$ comme une fonction de période $F_e$ de la variable réelle $f$ et y appliquer une décomposition en séries de \Fourier{}!

  Faites-le et comparez avec la \TFSD{} inverse. Vous venez de basculer dans un dual ! D'ailleurs on peut voir $s[k]$ comme les coefficients de \Fourier{} d'une fonction de fréquence fondamentale $T_E$ et appliquer une recomposition de la série et trouver $S(f)$.

  Donc \TFSD{}=\sdf{}$^{-1}$ et inversement \sdf{}=\TFSD{}$^{-1}$
\end{exercice}

\subsection{Base de la \TFD{} et  \FFT{}}

La \TFD{} (Transformée de \Fourier{} Discrète), ou \DFT{} (\emph{Direct Fourier Transform}) en anglais, s'applique aux fonctions discrètes à support fini et utilisent une base d'ondes complexes discrète finie $\Bf=\p{\caCest{k\mapsto W_N^{nk}=e^{i \frac{2 \pi}{N}\,n\,k}}{\vec{W_N^{n}}}}_{n\in\SN0}$ avec $W_N=e^{i \frac{2 \pi}{N}}$.

La $\FFT{}$ (\emph{Fast Fourier Transform} en anglais uniquement) est un algorithme efficace de calcul de la \TFD{}~: c'est donc la même transformation avec les mêmes valeurs~!

\begin{exercice}
Tentez de trouver la formule de cette \TFD{}  et son inverse, d'esquisser le schéma ci-dessous sans le regarder. Il s'agit d'une application de $$\SN0\rightarrow\C\;\;\operateur{\TFD}\;\;\SN0\rightarrow\C$$ basée sur le p.s. discret périodique noté $\scaldp{}{}$ avec la base continue $\Bf=\p{\vec{w_N^{n}}}_{n\in\SN0}$
\end{exercice}

\graphe{\textwidth}{tfd}


On peut faire l'analogie avec les espaces Euclidiens finis et \textbf{on peut faire l'amalgame~!} car c'en est un mais dans $\C$ donc~:
\begin{itemize}
\item le produit scalaire n'est pas symétrique mais \og{}symétrique et demi\fg{} \cad \emph{sesquilinéaire} car $\scal{u}{v}=\conj{\scal{v}{u}}$.
\end{itemize}

\begin{remarque}\remarqueTitre{Base orthogonale non normée}
  Au  produit scalaire $\scalp{f}{g}=$
  Le terme $W_N= e^{-i \frac{2 \pi}{N}}$ est en fait une racine $N$ième de l'unité. Le calcul de la norme $\norme{w_n}$ du vecteur de la base $w_n=k\mapsto W_N^{-nk}$ devient donc~: $$\scaldp{w_n}{w_n}=\somme{k=0}{N-1}{e^{i \frac{2 \pi}{N}nk}\,.\,e^{-i \frac{2 \pi}{N}nk}} =\somme{k=0}{N-1}{1}=N $$

  La formulation normée et symétrique de la \TFD{} \eqref{eq:tfd_normee} fait donc intervenir un facteur $\frac{1}{\sqrt{N}}$ pour la \TFD{} et son inverse~:
\begin{eqnarray}
    \label{eq:tfd_normee}
    s\b{k} = \somme{n\in\N}{}{\caCest{\scaldp{s}{w_n'}}{S\b{n}}\; w_n'} &= \somme{n\in\N}{}{\scaldp{s}{\frac{w_n}{\sqrt{N}}}}\;\, \frac{w_n}{\sqrt{N}} &\text{avec } \norme{w_n'}^2=\norme{\frac{w_n}{\sqrt{N}}}^2=1
  \end{eqnarray}
  
  Pour des raisons de simplicité et de contrainte de calcul numérique, la formulation non normée \eqref{eq:tfd_non_normee} est largement utilisée. Cela fait donc apparaitre le terme $\frac{1}{N}$ dans la \TFD{} inverse.

  \begin{eqnarray}
    \label{eq:tfd_non_normee}
    s\b{k} = \somme{n\in\N}{}{\caCest{\scaldp{s}{w_n}}{S\b{n}}\;\frac{w_n}{N}} &= \somme{n\in\N}{}{\scaldp{s}{w_n}}\;\, \frac{w_n}{N}&\text{avec } \norme{w_n}^2=N
  \end{eqnarray}
\end{remarque}

La transformée s'écrit aisément sous forme matricielle soit en considérant la transformée comme une application linéaire effectuant le changement de coordonnées entre les vecteur de la base canonique et la base fréquentielle, soit en faisant le changement de repère sous forme vectorielle~:

\begin{equation}
  \label{eq:tfd_matrice}
  \begin{array}{ll}
    \mTFD\b{\vec{s}} = \hat{S}=s_{B_F}&=\vvvectBase{\scaldp{s}{w_0}}{\vdots}{\scaldp{s}{w_{N-1}}}{B_F}\\
    &= \vvvectBase{\Tr{s} \, .\, \conj{w_0}}{\vdots}{\Tr{s} \, .\, \conj{w_{N-1}}}{B_F} = \vvvectBase{\Tr{\conj{w_0}} \, .\, s}{\vdots}{\Tr{\conj{w_{N-1}}} \, .\, s}{B_F}  \\
                                      &=\caCest{\b{\mmmaaatrice{\conj{w_0[0]}}{\ldots}{\conj{w_0[N-1]}}{\vdots}{}{\vdots}{\conj{w_{N-1}[0]}}{\ldots}{\conj{w_{N-1}[N-1]}}}}{M_{\F}}\;\;.\;\; \caCest{\vvvectBase{s[0]}{\vdots}{s[N-1]}{B_C}}{s_{B_c}} \\
    &S_{B_F} = M_{\F}\,.\,s_{B_c} 
  \end{array}
\end{equation}

On peut donc calculer la \TFD{} d'un signal en multipliant le vecteur du signal temporel par une matrice $M_{\F}$ pour obtenir le vecteur des composantes fréquentielle~: la \TFD{} du signal.

Donc la matrice $M_{\F}$ représente un changement de base d'une base orthonormée canonique $B_c$ vers la  base orthogonale fréquentielle $B_F$, c'est donc la matrice de l'application identité $\Id$ du signal d'une base vers l'autre, ou plus simplement la matrice de passage $M_{\F} = P_{B_F\leftarrow B_c} = \mat\p{\Id,B_F,B_c}$
\section{Dualité des transformées}
\label{sec:dualite}

On remarque que la \TF{} et la \TFD{} sont des endomorphismes (de E
dans E) isomorphiques (il existe une réciproque), on parle
d'automorphisme~:
\begin{itemize}
\item la \TF{} transforme une fonction complexe du primal en fonction
  complexe du dual et la transformée inverse du dual vers le primal
  existe~;
\item la \TFD{} transforme une suite périodique complexe du primal en
  suite périodique complexe du Dual et la transformée inverse du
  primal vers le dual existe.
\end{itemize}

Contrairement aux \sdf{} et \TFSD{} qui sont des isomorphismes (de E
dans F et elles sont réciproques entre-elles) ~:
\begin{itemize}
\item la \sdf{} transforme une fonction complexe périodique du primal
  en suite complexe du dual, sa réciproque est la \TFD{} du dual~;
\item la \TFSD{} transforme une suite complexe du primal en fonction
  complexe périodique du Dual.
\end{itemize}

Sans rentrer dans la véritable définition d'un dual et de la dualité,
nous pouvons garder cette notion de transformation d'un espace primal
en son espace dual, et que si l'on retransforme le dual de la même
manière alors on obtient à nouveau le primal. Le dual est comme une
application qui est sa propre réciproque mais cela à l'échelle de la
transformation d'ensemble qui est sa propre réciproque.

L'idée d'appliquer la \TFSD{} à la \sdf{} $n\mapsto \hat{S}[n]$ de la
fonction primale $t\mapsto s(t)$, soit de faire le dual du dual, nous
laisse espérer retomber sur la fonction primale $s$. Et cela marche
car ces espaces sont duaux.

\begin{exercice}
  Prenons le signal constant périodique $s : t\mapsto 1$, toutes ses
  projections $\scalp{s}{t\mapsto e^{i2\pi\,t\,n\,F_0}}$ sont nulles
  pour $n\neq0$ sauf pour le vecteur $w_0 : t\mapsto 1$. Donc sa
  \sdf{} est la suite complexes nulle partout sauf pour n=0 soit
  l'impulsion unité en 0 : $n\mapsto\delta_0[n]$.

  Si l'on applique la \TFSD{} à la \sdf{} $\hat{S}$ on obtient la
  fonction périodique constante et égale $1$ soit la fonction de
  départ du primal $s$ !

  Essayez de faire cela pour un fonction périodique $s(t)$
  quelconque~: soit montrer que $\TFSD{}\b{\sdf{}\b{s}}=s$ et donc
  $\TFSD{} \circ \sdf{} = \Id$ où $\Id$ est l'application identité des
  fonctions périodiques.  Ou son dual : soit pour une suite complexe
  $s[k]$ montrer que $\sdf \circ \TFSD = \Id$ où $\Id$ est
  l'application identité des suite complexes.
  
\end{exercice}


La \sdf{} permet de créer l'espace dual des fonctions périodiques, en
utilisant le produit scalaire avec les signaux de la base de \Fourier,
qui est alors un espace des suites complexes. La \TFSD{} permet de
créer le dual des suites complexes, en utilisant le produit scalaire
avec la base de \Fourier{}, qui est un espace des fonctions
périodiques.

L'idée d'appliquer la \TF{} à la transformée $\hat{S}$ d'une fonction
$s$ primale, soit de faire le dual du dual, nous laisse espérer
retomber sur la même fonction du primal $s$~: cela est vrai au signe
près~!


\begin{exercice}
  Prenons un signal pair de transformée connue (comme la fonction
  porte et son sinus cardinal) et calculez la transformée de la
  transformée (aussi connue à tout les coup). Vous vérifierez que
  \textbf{pour les fonction paires réelles} $\F{}\circ\F{}=\Id$ des
  fonctions réelles paires.

  On pourrait faire de même avec des fonctions imaginaires impaires et
  montrer que le dual du dual est le primal.

  Dans le cas général ce n'est pas vrai, nous allons voir que le
  signal est retourné dans le temps.
  
\end{exercice}

Si on calcule la transformée de $\Sy\b{s} : t\mapsto s(-t)$,
\textbf{où $\Sy{}$ est l'opérateur qui retourne une fonction dans le
  temps} (défini ci-après dans def.~\ref{def:Sy}) , on obtient par
changement de variable $x=-t$ la transformée inverse~:

\begin{eqnarray}
  \label{eq:Transformee_de_Sy}
  \F\b{\Sy\b{s}}(f)=\F\b{t\mapsto s(-t)}(f)&=&\quad\intDt{-\infty}{\infty}{s(-t)\,e^{-i2\pi\,f\,t}} \nonumber\\
                                           & \underset{x=-t}{=}& \quad\intDx{\infty}{-\infty}{s(x)\,e^{i2\pi\,f\,x}}\quad =\quad \Finv\b{s}(f) \nonumber\\
  \F\circ\Sy &=& \quad \Finv
\end{eqnarray}

\begin{definition}{Opérateur symétrie ou \og{} retournement \fg{}}
  \label{def:Sy}
  
  L'opérateur symétrie noté $\Sy$ transforme une fonction discrète de
  $L_e$ (resp. fonction continue de $L_c$) en fonction discrète de
  $L_e$ (resp. fonction continue de $L_c$) retournée par changement du
  signe de sa variable, soit~:
  \begin{align}
    \label{eq:Sy}
    &Sy\quad:\quad\application{L_e}{L_e}{k \mapsto s\ded{k}}{k \mapsto s\ded{-k}}\\
    &Sy\quad:\quad\application{L_c}{L_c}{t \mapsto s\de{t}}{t \mapsto s\de{-t}}\nonumber
  \end{align}
\end{definition}

En appliquant sucessivement cette relation au signal $s$~; à son
symétrique $Sy\circ s$~; à la transformée $\hat{S}$ et à son
symétrique $\Sy\circ\hat{S}$, on trouve le graphique
\figref{fig:dualite_4_f}.
\begin{figure}[ht!]
  \centering \graphe{0.6\textwidth}{dualite_4}
  \caption{Dualité de la transformation symétrie temporelle $\Sy$ car
    le symétrique du symétrique est lui même et bi--dualité de la
    transformée de \Fourier{} car transformer deux fois un signal
    c'est le retourner donc $\F{}^2=\Sy{}$ est duale et $\F{}$ est
    dite biduale.}
  \label{fig:dualite_4_f}
\end{figure}

\begin{exercice}
  Si on applique \eqref{eq:Transformee_de_Sy} au primal (flèches
  vertes), puis au dual (flèches oranges) et à leurs symétriques on
  obtient les 4 relations~:
  \begin{itemize}
  \item $\F\circ\Sy\b{s} = \Finv\b{s} = ?$
  \item
    $\F\circ\Sy\b{\hat{S}} = \Finv\b{\hat{S}} = \Finv\b{\F\b{s}}= ?$
  \item $\F\circ\Sy\b{t\mapsto s(-t)} = \Finv\b{t\mapsto s(-t)} = ?$
  \item
    $\F\circ\Sy\b{f\mapsto \hat{S}(-f)} = \Finv\b{f\mapsto
      \hat{S}(-f)} = ?$
  \end{itemize}

  Associez ces 4 relations aux quatre couleurs de flèches (vert,
  orange, cyan, rose) du diagramme et remplacez les \og{}?\fg{}

  Prenez en primal le signal porte
  $s(t)=\Pi_T(t)=u(t+\frac{T}{2})-u(t-\frac{T}{2})$ (signal nul
  partout sauf sur un intervalle de largeur $T$ autour de 0) dont on
  connaît la transformée sous forme de sinus cardinal
  $\hat{S}(f)=T\,\sinc\p{\pi\,f\,T}$ (utiliser la transformée de
  \Laplace, le théorème du retard et la passage à la transformée avec
  $p=i2\pi\,f$ pour retrouver cette formule ou par calcul
  direct). Appliquez ces formules pour trouver la transformée d'un
  sinus cardinal (fonction paire !)
\end{exercice}


On obtient ainsi des relations intéressantes du point de vue des
opérateurs $\F$ et $\Sy$, notamment sur le fait qu'ils commutent
bien~:
\begin{description}
\item $\F^2\circ\F^2 = \Sy\circ\Sy =\; \Id \; = \Sy^2 = \F^4$
\item $\Finv = \caCest{\F\circ\Sy}{\text{vert}} = \caCest{\Sy\circ\F}{\hat{S}\p{-\bullet}\text{ ou orange}}$
\item $\F = \caCest{\Finv\circ\Sy}{\text{rose}} = \caCest{\Sy\circ\Finv}{\hat{S}\p{\bullet}\text{ ou orange}}$
\end{description}
\begin{remarque} \remarqueTitre{De même pour la \TFD{}}
  
  On obtient le même type de diagramme avec la \TFD{} mais avec un
  retournement temporel discret périodique
  $\Sy : k \mapsto s[-k]= s[N-k]$ ce qui permet de conserver la
  relation $\mTFD\circ\Sy=\mTFD^{-1}$ mais avec une base normalisée~!
  Il faut donc faire attention et préférer la \TFD{} normalisée en la
  divisant par $\sqrt{N}$.
  
  Il suffit de partir de la TFD de $s[-k]$ et retrouver la formule de
  la \TFD{} inverse. On a de plus pour les signaux discrets une
  représentation matricielle. Ce qui donne pour le retournement
  temporel d'un signal à 4 points~:
  \begin{equation}
    \label{eq:sy_matriciel}
    \Sy\b{s_4} = \caCest{
      \left[
        \begin{array}{cccc}
          1 & 0 & 0 & 0\\
          0 & 0 & 0 & 1\\
          0 & 0 & 1 & 0\\
          0 & 1 & 0 & 0
        \end{array}\right]}{M_{\Sy}}\;.\;s_4
  \end{equation}
  avec $M_{\Sy}^{-1}=M_{\Sy}$ car $\Sy=Sy^{-1}$
      
  Et pour la \TFD{} normalisée à 4 points~:
  
  \begin{equation}
    \label{eq:tfd4_matriciel}
    \mTFD\b{s_4} = \widehat{S_4} = \caCest{
      \frac{1}{2}\left[
        \begin{array}{cccc}
          1 & 1 & 1 & 1\\
          1 & -i & -1 & i\\
          1 & -1 & 1 & -1\\
          1 & i & -1 & -i
        \end{array}
      \right]}{M_{\F}}\;.\;s_4
  \end{equation}
  avec $M_{\F}^{-1}=\Tr{M_{\F}}$ la matrice transposée. Car lors d'un
  changement entre base orthonormées, la matrice de passage est
  symétrique et orthogonale~: la transposée est aussi l'inverse.
  
  
  L'équation \eqref{eq:Transformee_de_Sy} devient alors
  $\TFD\circ\Sy = \TFD{}^{-1}$ ce qui donne en matriciel
  $M_{\F}\,M_{\Sy} = M_{\F}^{-1} = \Tr{M_{\F}} $
\end{remarque}

\subsection{Filtrage et fenêtrage~: opérations duales}

\subsection{Interpolation et gavage de zéro~: opérations duales}

\subsection{Sur-échantillonnage et multiplication de période~: opérations duales}

\subsection{Décimation et division de période~: opérations duales}

%%% Local Variables:
%%% mode: latex
%%% TeX-master: "main"
%%% End:
