\chapter{Les bases de Fourier}
\label{chap:fourier}
Ce chapitre présente les transformées, séries de \Fourier{} et introduit la transformée de\Fourier{} Discrète en se basant sur une analogie entre les espaces vectoriels Euclidiens et les espaces de Hilbert où~:
\begin{itemize}
\item un vecteur est un signal sous forme d'une fonction de $\R\rightarrow\C$~;
\item le produit scalaire de deux vecteurs est sous forme d'intégrale (ou somme) du produit~;
\item les coefficients peuvent être complexe $\C$-espace vectoriel et non $\R$-espace vectoriel comme  pour les Euclidiens, ce qui change la propriété de symétrie du produit scalaire.
\end{itemize}


La \figref{fig:espaces} mets en évidence cette analogie et souligne le fait que transformer un signal avec un produit scalaire revient à a effectuer un changement de base.

\begin{figure}[ht]
  \centering
  \graphe{0.9\textwidth}{espaces}
  \caption{Changement de base d'un vecteur dans un espace Euclidien de $\R^3$ et transformation d'une fonction de $\R\rightarrow\C$ dans un espace de Hilbert. }
  \label{fig:espaces}
\end{figure}

On utilise par habitude la variable $t$ car l'espace de départ est souvent des \emph{fonctions du temps}. On nomme \emph{espace primal} ce premier espace considéré.

En décomposant un signal vecteur dans une base, on va pouvoir représenter ce signal par ses coordonnées dans la base d'arrivée, on dit plutôt composantes pour un signal. On obtient alors un \og{}signal des composantes\fg{} dans un espace que l'on appelle \emph{espace dual}. Comme l'on utilise la plupart des temps des  bases dites fréquentielles, la variable de l'espace dual est notée $f$ et correspond aux \emph{fonctions de la fréquence}.

\begin{remarque}\remarqueTitre{Pourquoi du complexe pour du réel ?}
  On considère le cas général des fonctions à valeurs dans $\C$ car cela permet~:
  \begin{itemize}
  \item de représenter des signaux de type phaseur (vecteurs en rotation)
  \item de représenter facilement des signaux réels dont la bande de
    fréquence est finie (signaux dits \emph{bande étroite})
  \item de pouvoir appliquer la trasnformée de \Fourier{} à une
    transformée de \Fourier{} (qui est complexe même pour des signaux
    réels) et de voir l'opération de transformée comme un isomorphisme
    et de jouer avec la notion de dual.
    \end{itemize}
  \end{remarque}
  

La base de vecteurs utilisée dans les transformées de \Fourier{}
$\p{t\mapsto e^{2\pi\,f\,t}}_{f\in\R}$ va être déclinée en différentes
bases selon que l'on va discrétiser une des variables~:

\begin{description}
\item[temps discret] où la variable $t\in\R$ sera remplacée par une variable discrète $k\in\N$ avec la relation $t=k\,T_e$ où $T_e$ est la valeur qui sépare deux échantillons en temps est nommé \emph{période d'échantillonnage}.
\item[frequences discrètes] où la variable $f\in\R$ sera remplacée par une variable discrète $n\in\N$ avec la relation $f=n\,\Delta_f$ où $\Delta_f$ est la valeur qui sépare deux échantillons en fréquence est nommé \emph{résolution fréquentielle}
\end{description}

Selon la dimention continue (infinie indénombrable) ou discrète
(infinie dénombrable ou finie) des espaces primal et dual, on utiliser
différents produis scalaires pour effectuer différente transformées
entre primal et dual.

\section{Les espaces et produits scalaires associés}
Nous allons considérer des espaces de fonctions tantôt à variables
continues puis discrètes, et en même temps sur des supports infinis ou
bornés. Dans le cas de fonctions bornées (définies sur un intervalle $\semi{a}{a+T_0}$ en continu ou $\semiN{a}{a+N}$ en discret), on peut \textbf{toujours} prolonger cette fonction
en dehors du support de manière périodique, plutôt que par des zéros, car cela permet d'avoir une représentation en séries de \Fourier{} (\sdf{}) (cas continu) ou en Transformée de Fourier Discrète (\TFD{}) dans le cas discret.

Les produits scalaires pour différents espaces de fonctions sont définis et illustrés dans la \tabref{tab:scalaires} en prenant~:
\begin{itemize}
\item  en rangées du tableau les signaux de la \emph{variable continue} (intégrale continue) ou bien de la \emph{variable discrète} (somme discrète)~;
\item en colonnes du tableau les \emph{supports infinis} (de $-\infty$ à $\infty$) ou bien support \emph{périodiques/borné} (de $0$ à $T_0$ ou $N$).
\end{itemize}

\begin{table}[h!]
  \begin{tabular}{p{0.1\textwidth}|c|c}
    &Support infini    & Support fini ou périodique  \\\hline
    &  \begin{tabular}{c} \graphe{0.4\textwidth}{fonction_rc}\end{tabular}
                       &  \begin{tabular}{c} \graphe{0.5\textwidth}{fonction_rtc} \end{tabular}                           \\
    variable continue & $f:\R\mapsto\C$    &  $f:\RT0{}\mapsto\C\;$  ou $f:\;[0,\,T_0[\, \mapsto \C $       \\
    &$\scal{f}{g}=\intDt{\R}{}{f(t).\conj{g(t)}}$ $\quad\deDim{\frac{V^2}{\Hz}}$ & $\scalp{f}{g}=\frac{1}{T_0}\intDt{0}{T_0}{f(t).\conj{g(t)}}$ $\quad\deDim{V^2}$ \\\hline
    &   \begin{tabular}{c} \graphe{0.4\textwidth}{fonction_nc}\end{tabular}
                       &  \begin{tabular}{c} \graphe{0.5\textwidth}{fonction_znc}    \end{tabular}                        \\
    variable discrète & $\N\mapsto\C$    &  $\ZN0\mapsto\C\;$  ou $\;\semiN{0}{N}\;\mapsto \C$        \\
    & $\scald{f}{g}=\somme{k\in\N}{}{f[k].\conj{g[k]}}$ $\quad\deDim{V^2}$& $\scaldp{f}{g}=\somme{k=0}{N-1}{f[k].\conj{g[k]}}$  $\quad\deDim{V^2}$  
  \end{tabular}
  \caption{Les produits scalaires adaptés aux différents espaces de fonctions. Par clarté, on ne représente que le module de la fonction qui est dans la cas général complexe.}
  \label{tab:scalaires}
\end{table}
  
\begin{exercice}\exerciceTitre{Propriété de scalaire et norme dans le cas général}
  On aurait pu définir ces produits scalaires en ne prenant jamais le
  conjugué d'une fonction $g$ (ou en considérant des fonction à
  valeurs réelles de manière à ignorer ce conjugué car $\conj{g}=g$).
  \begin{enumerate}
  \item Vérifiez dans le cas réel (sans conjugué) que le produit
    $\scal{f}{g}$ à les propriété d'un produit scalaire, en déduire la
    norme induite $\|f\|^2=\scal{f}{f}$ et déterminer la dimention de
    $\|f\|^2$ : est-ce de la puissance ou de l'énergie, est-ce une
    valeur ou une densité ?
  \item Appliquez cette norme (toujours sans le conjugué) au signal imaginaire pur
    $f: t\mapsto i$. Quelle propriétée de la norme n'est pas respectée ?
  \item Refaites de même en prenant cette fois-ci les formules de
    \tabref{tab:scalaires} avec le conjugué de $g$ et vérifiez que
    cette propriété est vérifiée dans le cas général des fonctions à
    variables complexes.
  \item Vérifiez que $\scal{f}{g}=\conj{\scal{g}{f}}$ et que donc le produit scalaire est linéaire à gauche $\forall\lambda\in\C,\; \scal{\lambda\,f}{g}=\lambda\scal{f}{g}$ et \emph{à moitié linéaire} à droite $\forall \lambda\in\C,\; \scal{f}{\lambda\,g}=\conj{\lambda}\scal{f}{g}$
  \end{enumerate}
On comprend maintenant pourquoi, dans le cas général des fonctions à valeurs complexes, on utilise le conjugué dans l'expression des produit scalaires et pourquoi on parle de produit \emph{sesqui--linéaire} pour ces produits scalaires~: \emph{sesqui} en latin voulant dire \og{}un et demi\fg{} en latin.  
\end{exercice}

Le produit scalaire est très utile car il permet d'obtenir~:
\begin{itemize}
\item de mesurer des longueurs de signaux avec la norme induite par le produit scalaire $\norme{\vec{s}}=\scal{\vec{s}}{\vec{s}}$, et de mesurer des distances entre signaux avec la norme de la différence $\norme{\vec{u}-\vec{v}}$~;
\item de projeter un vecteur sur un autre ou sur un sous-espace
  vectoriel~: cela revient à minimiser une distance $P_v(u)=\min\limits_{x\in\vect{u}}\p{\norme{u-x}}$ par simple calcul direct~;
  
\item trouver les meilleures, au sens de la distance avec la norme
  engendrée, décompositions d'un signal $u$ sur une base de vecteurs
  données~: calculer des transformées de signaux.
\end{itemize}

La \tabref{tab:hilbert} montre le parallèle entre l'utilisation du produit scalaire sur des vecteurs et sur des signaux, chacune permet de retrouver des formules bien connues des \sdf{} et des \TF{}.

%\begin{landscape}
\begin{table}[h!]
\renewcommand{\arraystretch}{1.4}
\begin{tabular}{p{0.1\textwidth}|p{0.32\textwidth}|p{0.6\textwidth}}
  & Euclidien fini & Espace de fonctions \\\hline
  Base   Ortho--normée & une base finie de vecteurs $$\B=\p{\vec{e_n}}_{n\in\ZN0}$$ normés $\|\vec{e_n}\|=1$ et orthogonaux $\scal{\vec{e_n}}{\vec{e_m}}=0$ & base dénombrable de fonctions $\p{\vec{w_n}}_{n\in\N}$ ou indénombrable $\p{\vec{w_f}}_{f\in\R}$ repérées par leur fréquences $f$ ou un indice $n$ associé~;  fonctions d'énergie unitaire $\|\vec{w_n}\|=1$ ou $\|\vec{w_f}\|=1$, et orthogonales $\scalp{\vec{w_n}}{\vec{w_m}}=0$ ou $\scal{\vec{w_{f}}}{\vec{w_{f'}}}=0$\\\hline
  
  Analyse &  décomposer un vecteursdans cette base en coefficients $V_n=\scal{\vec{v}}{\vec{e_n}}$ et en donner les coordonnées $$\caCest{\vecDs{V}{\B}}{\rightleftharpoons\vec{v}}=\vvvectBase{V_0=\scal{\vec{v}}{\vec{e_0}}}
            {\vdots}
            {V_{N-1}=\scal{\vec{v}}{\vec{e_{N-1}}}}{\B}$$
                   &  décomposer une fonction $\vec{u}$ en fréquentiel avec la transformée $U(f)$ ou avec les coéfficients $U(n)$ de la série~: $$U(f)=\scal{\vec{u}}{\vec{w_f}}=\scalint{u(t)}{w_f(t)}{t}$$ $$U(n)=\scalp{\vec{u}}{\vec{w_n}}=\scalpint{u(t)}{w_n(t)}{t}$$ \\\hline
  
  Synthèse &  recomposer un vecteur dans cette base $\vec{v}=\somme{k\in\ZN0}{}{\caCest{U_0}{\scal{\vec{v}}{\vec{e_0}}}}\,.\,\vec{e_0}$ \graphe{4cm}{projections} &  recomposer une fonction par transformation inverse de $U(f)$  ou recompostion de série $U(n)$~: $$\vec{u}(t) = \int\limits_{-\infty}^{\infty}{\caCest{U(f)}{\scal{\vec{u}}{\vec{w_f}}}\,.\,\vec{w_f}(t)\,\dt} $$ $$\vec{u}(t)=\sum\limits_{n\in\N}\caCest{U(n)}{\scalp{\vec{u}}{\vec{w_n}}}\,.\,\vec{w_n}(t) $$\\\hline

  Projeter avec Plancherel  &  calculer le produit scalaire de vecteurs par leurs composantes~: $$\scal{\vec{u}}{\vec{v}}=\scal{U_{\B}}{V_{\B}}=\Tr{\vecDs{U}{\B}}\,.\,\vecDs{V}{\B}$$ $$\caCest{\hhhect{U_0}{\ldots}{U_{N-1}}}{\Tr{\vecDs{U}{\B}}}\,.\,\vvvect{V_0}{\vdots}{V_{N-1}}$$
  % au lieu de calculer $$\scal{\vec{u}}{\vec{v}}=\|\vec{u}\|\,\|\vec{v}\|\,\cos\p{\alpha}$$
                   &  on peut calculer un produit scalaire (utile aux correlations et convolutions) à partir de sa transformée ou composantes de la série~: $$\scal{\vec{u}}{\vec{v}}=\scalint{u(t)}{v(t)}{t} = \scal{U}{V} =  \scalint{U(f)}{V(f)}{f}$$ $$\scalp{\vec{u}}{\vec{v}}=\scalpint{u(t)}{v(t)}{t} = \scaldp{U}{V} =  \scalpdint{U(k)}{V(k)}{k}$$
  \\\hline
  
  Normer avec Parseval  &  Calculer la norme en sommant les carrés des coordonnées~: $$\|\vec{u}\|^2=\|U_{\B}\|^2=\sum {U_n}^2$$ &  calculer la puissance moyenne par la transformée $u(f)$ ou en sommant celle des composantes fréquentielles $U(n)$~: $$\|\vec{u}\|^2=\|U\|^2=\int\limits_{-\infty}^{\infty}{|u(t)|^2\dt}=\int\limits_{-\infty}^{\infty}{|U(f)|^2\df} $$  $$\|\vec{u}\|^2=\|U\|_P^2=\frac{1}{T_0}\int\limits_{0}^{T_0}{|u(t)|^2\dt}=\sum\limits_{k\in\N}{|U(k)|^2} $$
\end{tabular}

\caption{Structure Euclidiene à structure de Hilbert}
\label{tab:hilbert}
\end{table}
%\end{landscape}

\section{Les transformations}

En prenant la base des ondes complexes adaptée à chaque espaces de
signaux (discrétisée ou non, sur un intervalle infini ou
borné/périodique), et en utilisant les produits scalaires adaptés, on
peut définir quatre types de transformations et leur réciproques entre
un primal avec une base canonique purement localisée dans le temps (et
infiniment étendue en fréquence) et un dual composé d'une base d'ondes
purement localisées fréquentielles (et infiniment étendue en dans le
temps).

Le schéma ci dessous résume ces transformées, leurs base et les
produits scalaires associés à chaque transformation~:


\graphe{0.8\textwidth}{transformees.png}

\subsection{Base de la \TF{}}

La \TF{} (Transformée de \Fourier), ou \FT{} (\emph{Fourier
  Transform}) en anglais, s'applique aux fonction continues et utilise
une base d'ondes complexes
$\Bf=\p{\caCest{t\mapsto e^{i2\pi\,f\,t}}{w_f}}_{f\in\R}$.

\begin{exercice}
Tentez de retrouver la formule de la transformée et son inverse et d'esquisser le schéma ci-dessous sans le regarder, en se rappelant juste que c'est une application de $$\R\rightarrow\C\;\;\operateur{\TF}\;\;\R\rightarrow\C$$ basée sur le produit scalaire continu noté $\scal{}{}$ avec la base continue $\Bf=\p{w_f}_{f\in\R}$
\end{exercice}

\graphe{\textwidth}{tf}

 On peut faire
l'analogie avec les espaces Euclidiens mais pas l'amalgame, car~:
\begin{itemize}
\item le produit scalaire $\scal{}{}$ est défini  dans le cas de fonctions de carré
intégrable, ou \emph{fonction à énergie finie}, que nous notons $\Leb{2}$,
\item les vecteurs de la base ne sont pas normés car de norme infinie~;
\item la base n'est pas finie, ni infinie dénombrable mais infinie indénombrable.
\end{itemize}

Mais lorsque l'on se place dans le cas de fonctions de carré
intégrable, ou fonction à énergie finie, que nous notons $\Leb{2}$,
l'espace est complet (les suites de Cauchy convergent) dont les sommes
infinies se comporte bien dans $\Leb{2}$~: c'est une espace de
Banach. De plus le produit scalaire associé à la norme 2 existe et on
a donc un espace de \Hilbert{} où la norme et le produit scalaire sont
des application linéaire dans $\Leb{2}$. Comme il s'agit d'une espace
de dimention infinie, il ne suffit pas d'avoir une base de dimention
infinie pour couvrir tout l'espace, mais dans le cas de $\Leb{2}$ avec
la base $\Bf$ on montre que tout l'espace est engendré.

Bref ! ça fonctionne tout comme un espace Euclidien sans en être un.

\begin{exercice}
Prendre la base $\Bf=\p{w_f}_{f\in\R}$ et utiliser la partie espace indénombrable de la \tabref{tab:hilbert} pour retrouver les formules de \Plancherel{} et \Parseval{}. 
\end{exercice}


\subsection{Base des \sdf{}}
Les \sdf{} (Séries de \Fourier{}), ou \FS{} (\emph{Fourier Series}) s'appliquent aux fonctions continues périodiques et   utilisent une base dénombrable $\Bf=\p{\caCest{t\mapsto W_{T_0}^{n\,t}=e^{i \frac{2 \pi}{T_0}\,n\,t}}{w_{T_0}^n}}_{n\in\N}$ avec $W_{T_0}=e^{i \frac{2 \pi}{T_0}}$.
\begin{exercice}
Tentez de retrouver la formule de la décomposition et recomposition en \sdf{} et d'esquisser le schéma ci-dessous sans le regarder, en se rappelant juste que c'est une application de $$\RT0\rightarrow\C\;\;\operateur{\sdf}\;\;\N\rightarrow\C$$  basée sur le produit scalaire continu périodique noté $\scalp{}{}$ et avec la base discrète $\Bf=\p{W_{T_0}^n}$.
\end{exercice}

\graphe{\textwidth}{sdf}

 On peut faire l'analogie avec les espaces Euclidiens mais pas l'amalgame, car~:
 \begin{itemize}
   \item le produit scalaire $\scalp{}{}$ est défini dans le cas de fonctions périodiques de carré intégrable, \emph{fonctions de puissance moyenne finie},  que nous notons $\LebP{2}$
 \item ce n'est pas un isomorphisme car on passe d'un espace continu périodique à
   un espace discret~! La transformée inverse se fait avec le produit scalaire discret $\scald{}{}$
\item la base n'est pas finie, mais infinie dénombrable~;
\end{itemize}

Bref ! cela fonctionne un peu comme un espace Euclidien fini sans en être un...

\begin{exercice}
  Prendre la base  $$\Bf=\p{\caCest{t\mapsto \cos\p{\frac{2 \pi}{T_0}\,n\,t}}{\cos_n}}_{\!\!\!\!n\geq 1}\cup \p{\caCest{t\mapsto \sin\p{\frac{2 \pi}{T_0}\,n\,t}}{\sin_n}}_{\!\!\!\!n\geq 1}\cup \p{t\mapsto 1}$$ et voir que l'on retrouve les formules des coefficients $a\b{n}$, $b\b{n}$ et $a_0$ \textbf{à un facteur 2 près !}
  
  Et oui ! La base n'est pas normée car un rapide calcul montre que la norme des vecteurs vaut~$\frac{1}{2}$ (on peut se rappeler que la valeur efficace d'un cosinus d'amplitude 1 est $\frac{\sqrt{2}}{2}$~; sa puissance moyenne sur une période est donc le carré de $\frac{\sqrt{2}}{2}$) 

  En prenant la base normée $$\Bf'=\p{\caCest{\sqrt{2}\cos_n}{\cos_n'}}_{\!\!\!\!n\geq 1}\cup \p{\caCest{\sqrt{2}\sin_n}{\sin_n'}}_{\!\!\!\!n\in\Z^*} \cup \p{t\mapsto 1}$$
  on obtient une définition des \sdf{} chère aux physiciennes~:
  \begin{equation}
    \label{eq:sdf_normee}
    s\p{t} = \left(
    \begin{array}{c}
      {\caCest{\scalp{s}{t\mapsto 1}}{a_0}\;1} \\
      + \somme{n=1}{+\infty}{\caCest{\scalp{s}{\cos_n'}}{a'\b{n}}\;\cos_n'(t)}  \\
      + \somme{n=1}{+\infty}{\caCest{\scalp{s}{\sin_n'}}{b'\b{n}}\;\sin_n'(t)}     
    \end{array}\right)
  = \left(
    \begin{array}{c}
      a_0  \\
      + \somme{n=1}{+\infty}{\caCest{\scalp{s}{\sqrt{2}\cos_n}}{\sqrt{2}\,a\b{n}}\;\sqrt{2}\,\cos_n(t)}  \\
      + \somme{n=1}{+\infty}{\caCest{\scalp{s}{\sqrt{2}\sin_n}}{\sqrt{2}\,b\b{n}}\;\sqrt{2}\,\sin_n(t)}
    \end{array}\right)
\end{equation}
Avec $\norme{t\mapsto 1}^2=1$, $\norme{\cos_n'}^2=\norme{\sqrt{2}\cos_n}^2=1$ et $\norme{\sqrt{2}\sin_n}^2=1$

On peut ne pas normer les vecteurs, et c'est le plus fréquent, mais introduire un facteur $2$ dans la formule de calcul des coefficients $a\b{n}$ et $b\b{n}$ qui n'apparait pas dans les coefficients $c\b{n}$~:
  \begin{equation}
    \label{eq:sdf_non_normee}
    s\p{t} = \left(
    \begin{array}{c}
      {\caCest{2\,\scalp{s}{t\mapsto 1}}{a_0}\;1} \\
      + \somme{n=1}{+\infty}{\caCest{2\,\scalp{s}{\cos_n}}{a\b{n}}\;\cos_n(t)}  \\
      + \somme{n=1}{+\infty}{\caCest{2\,\scalp{s}{\sin_n'}}{b\b{n}}\;\sin_n(t)}     
    \end{array}\right)
  = \left(
    \begin{array}{c}
      \frac{a_0}{2}  \\
      + \somme{n=1}{+\infty}{\caCest{2\,\scalp{s}{\cos_n}}{a\b{n}}\;\cos_n(t)}  \\
      + \somme{n=1}{+\infty}{\caCest{2\,\scalp{s}{\sin_n}}{b\b{n}}\;\sin_n(t)}
    \end{array}\right)
\end{equation}
Avec $\norme{t\mapsto 1}^2=1$, $\norme{\cos_n}^2=\norme{\sin_n}^2= \frac{1}{2}$.

\end{exercice}


\subsection{Base de la \TFSD{}}

La \TFSD{} (Transformée de \Fourier{} des Signaux Discrets), ou \DTFT{} (\emph{Discrete Time Fourrier Transform} en anglais, s'applique aux fonctions à variable discrète et utilise une base d'ondes complexes indénombrable $\Bf=\p{\caCest{k\mapsto W_{T_e}^{f\,k}=e^{i 2 \pi\,T_e\,f\,k}}{w_{T_e}^f}}_{f\in\SFe{}}$ avec $W_{Te}=e^{i 2 \pi\,T_e}$.

\begin{exercice}
Tentez de trouver la formule de cette \TFSD{}  et son inverse, d'esquisser le schéma ci-dessous sans le regarder, en pensant que c'est la \og{}duale\fg{} de la \sdf{}. Il s'agit d'une application de $$\N\rightarrow\C\;\;\operateur{\TFSD}\;\;\RFe\rightarrow\C$$ basée sur le produit scalaire discret noté $\scald{}{}$ avec la base continue $\Bf=\p{w_{T_e}^f}_{f\in\SFe}$
\end{exercice}

\graphe{\textwidth}{tfsd}


On peut difficilement faire l'analogie avec les espaces Euclidiens  car~:
\begin{itemize}
   \item le produit scalaire $\scald{}{}$ fonctionne dans le cas de suites discretes absoluement convergentes~;
 \item ce n'est pas un isomorphisme car on passe d'un espace discret à un espace continu périodique~! La transformée inverse se fait avec le produit scalaire continu périodique $\scalp{}{}$ (Attention la période dans l'espace des fréquences est $F_e$)~;
\item la base n'est pas finie, ni dénombrable  mais infinie indénombrable.
\end{itemize}


\begin{exercice}
  On admet pour le moment que la \TFSD{} d'un signal $s[k]$ quelconque est une fonction $S(f)$ de période $F_e$. On peut donc voir $S(f)$ comme une fonction de période $F_e$ de la variable réelle $f$ et y appliquer une décomposition en séries de \Fourier{}!

  Faites-le et comparez avec la \TFSD{} inverse. Vous venez de basculer dans un dual ! D'ailleurs on peut voir $s[k]$ comme les coefficients de \Fourier{} d'une fonction de fréquence fondamentale $T_E$ et appliquer une recomposition de la série et trouver $S(f)$.

  Donc \TFSD{}=\sdf{}$^{-1}$ et inversement \sdf{}=\TFSD{}$^{-1}$
\end{exercice}

\subsection{Base de la \TFD{} et  \FFT{}}

La \TFD{} (Transformée de \Fourier{} Discrète), ou \DFT{} (\emph{Direct Fourier Transform}) en anglais, s'applique aux fonctions discrètes à support fini et utilisent une base d'ondes complexes discrète finie $\Bf=\p{\caCest{k\mapsto W_N^{nk}=e^{i \frac{2 \pi}{N}\,n\,k}}{W_N^{n}}}_{n\in\SN0}$ avec $W_N=e^{i \frac{2 \pi}{N}}$.

La $\FFT{}$ (\emph{Fast Fourier Transform} en anglais uniquement) est un algorithme efficace de calcul de la \TFD{}~: c'est donc la même transformation avec les mêmes valeurs~!

\begin{exercice}
Tentez de trouver la formule de cette \TFD{}  et son inverse, d'esquisser le schéma ci-dessous sans le regarder. Il s'agit d'une application de $$\SN0\rightarrow\C\;\;\operateur{\TFD}\;\;\SN0\rightarrow\C$$ basée sur le p.s. discret périodique noté $\scaldp{}{}$ avec la base continue $\Bf=\p{w_N^{n}}_{n\in\SN0}$
\end{exercice}

\graphe{\textwidth}{tfd}


On peut faire l'analogie avec les espaces Euclidiens finis et \textbf{on peut faire l'amalgame~!} car c'en est un mais dans $\C$ donc~:
\begin{itemize}
\item le produit scalaire n'est pas symétrique mais \og{}symétrique et demi\fg{} \cad \emph{sesquilinéaire} car $\scal{u}{v}=\conj{\scal{v}{u}}$.
\end{itemize}

\begin{remarque}\remarqueTitre{Base pas normée}
  Le terme $W_N= e^{-i \frac{2 \pi}{N}}$ est en fait une racine $N$ième de l'unité. Le calcul de la norme $\norme{w_n}$ du vecteur de la base $w_n=k\mapsto W_N^{-nk}$ devient donc~: $$\scaldp{w_n}{w_n}=\somme{k=0}{N-1}{e^{i \frac{2 \pi}{N}nk}\,.\,e^{-i \frac{2 \pi}{N}nk}} =\somme{k=0}{N-1}{1}=N $$

  La formulation normée et symétrique de la \TFD{} \eqref{eq:tfd_normee} fait donc intervenir un facteur $\frac{1}{\sqrt{N}}$ pour la \TFD{} et son inverse~:
\begin{eqnarray}
    \label{eq:tfd_normee}
    s\b{k} = \somme{n\in\N}{}{\caCest{\scaldp{s}{w_n'}}{S\b{n}}\; w_n'} &= \somme{n\in\N}{}{\scaldp{s}{\frac{w_n}{\sqrt{N}}}}\;\, \frac{w_n}{\sqrt{N}} &\text{avec } \norme{w_n'}^2=\norme{\frac{w_n}{\sqrt{N}}}^2=1
  \end{eqnarray}
  
  Pour des raisons de simplicité et de contrainte de calcul numérique, la formulation non normée \eqref{eq:tfd_non_normee} est largement utilisée. Cela fait donc apparaitre le terme $\frac{1}{N}$ dans la \TFD{} inverse.

  \begin{eqnarray}
    \label{eq:tfd_non_normee}
    s\b{k} = \somme{n\in\N}{}{\caCest{\scaldp{s}{w_n}}{S\b{n}}\;\frac{w_n}{N}} &= \somme{n\in\N}{}{\scaldp{s}{w_n}}\;\, \frac{w_n}{N}&\text{avec } \norme{w_n}^2=N
  \end{eqnarray}
\end{remarque}











\begin{exercice}
  Tout comme la \TF{}, la \TFD{} est un isomorphisme bidual. Pour un
  signal $s$ quelconque, calculez la transformée de la transformée et
  montrez que
  $ \caCest{\F\circ \F}{\F^2}\b{s}=\F\b{\F\b{t\mapsto s(t)}} =
  t\mapsto s(-t)$. Donc si on recalcule deux fois la transformée on obtient le signal d'origine et on montre que $\F^2$ est sa propre réciproque. Déduisez une technique pour calculer la transformée inverse $\F^{-1}$ en utilisant la transformée $\F$.

  Faites de même pour la \TFD{} 
\end{exercice}




%%% Local Variables:
%%% mode: latex
%%% TeX-master: "main"
%%% End:
