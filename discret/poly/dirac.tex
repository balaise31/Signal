\chapter{La distribution de Dirac}
\label{sec:dirac}

L'impulsion de \Dirac{} est incontournable en traitement du signal et système~; nous allons progressivement lever le voile~!

\section{La notion de densité}

Retournons sur 3 notions de densité avec phénomène de localisation intense~:
\begin{description}
\item [la densité de masse] -- est une notion physique que l'on peut comprendre aisément avec l'exemple d'une mousse au chocolat dont la \emph{densité de masse} est plus ou moins aérée \emph{selon une position réelle} sur un axe de découpe du gateau. On considère la pépite pure de chocolat de $1$g \emph{concentrée à un endroit} infinitésimallement bon du gateau.
\item[la densité de probabilité] -- on considère la \emph{densité de probabilité} d'un nombre tiré au hasard entre $1$ et $10$\emph{ selon la valeur réelle} de ce nombre. Le tirage d'un dé à 6 faces sera la \emph{concentration infinie autour de valeurs précises} de la densité de probabilité.
\item[la densité d'amplitude (ou de puissance)] -- dans le cas d'une transformation de \Fourier{}, on considère la \emph{densité d'amplitude} des composantes d'un signal \emph{selon une fréquence réelle}. La décomposition en série de \Fourier{} sera la \emph{concentration infinie autour de fréquences} harmoniques de cette densité d'amplitude.    
\end{description}


Dans les année 20, \Dirac{} a eu besoin de représenter la concentration de densité de probabilité de particules élémentaires autour de valeurs précises et discrètes pour développer la mécanique quantique.

Une densité $f$ n'as de sens, ou est utile, uniquement en l'intègrant pour avoir une \og{} mesure\fg{} de la masse, ou de la probabilité ou de l'amplitude sur un segment de valeurs $[a, b]$~:

\begin{equation}
  \label{eq:densite_mesure}
  M_{[a, b]} = \int_a^b f(x)\;\derivDe{x}
\end{equation}

On voit que dans le cas discret, il faut dériver une fonction
cumulative \og{} en escalier \fg{} et donc dériver des discontinuités
pour obtenir une densitée continue d'une variable discrète. Or la dérivée d'ubne discontinuité est mal définie ! Nous allons le voir dans la section suivante.

\clearpage

\begin{figure}[ht!]\centering
    \graphe{\textwidth}{densites}
    \caption{A gauche une variable aléatoire continue $\seg{0}{6}$. A droite une variable aléatoire discrète $\segN{1}{6}$. En bas une mesure de probabilité de tirer une valeur autour de 3 dans $]2, 4[$~; au milieu la fonction cumulative $F_x$ ou probabilité d'avoir une valeur inférieure à $t$, en haut une tentative de densité de probabilité.}
    \label{fig:densites}
  \end{figure}
  
\begin{quizz}

  
  En regardant la \figref{fig:densites} répondez à ces questions~:
  \begin{description}
  \item[Q1 -] la probabilité d'avoir un tirage unique à exactement 3 est de 
    \begin{enumerate}
    \item 0 pour t=3 en continu et 1/6 pour k=3 en discret
    \item 1/6 pour t=3 en continu et 1/6 pour k=3 en discret
    \item 1/6 pour t=3 en continu et $\infty$ pour k=3 en discret
    \item c'est pas 3 que je veux mais 20/20 !
    \end{enumerate}
  \item[Q2 -] la probabilité d'avoir un tirage unique $ <3$ est de 
    \begin{enumerate}
    \item 1/2 pour <t=3 en continu et 1/2 pour <k=3 en discret
    \item 1/2 pour <t=3 en continu et 1/3 pour <k=3 en discret
    \item pareil que pour $t\leq3$ en continus et pareil que pour $k\leq3$ en discret
    \item si on met la bonne réponse toujours au début c'est débile~!
    \end{enumerate}
   \item[Q3 -] la densité de probabilité pour la valeur exacte 3 est de 
    \begin{enumerate}
    \item 0 pour t=3 en continu et 1/6 pour k=3 en discret
    \item 1/6 pour t=3 en continu et 1/6 pour k=3 en discret
    \item 1/6 pour t=3 en continu et $\infty$ pour k=3 en discret
    \item je préfère rester discrète sur la question...
    \end{enumerate}
  \item[Q4] Parmi les transformées représentées sur la figure suivante, lesquelles sont des densitées et dans quelles unités (on suppose un signal primal temporel en Volts)~:
    \begin{enumerate}
    \item $\hat{S}(f)$ de la \TF{} est une densité continue en $V$~;
    \item $\hat{S}(f)$ de la \TF{} est une densité continue en $V/Hz$~;
    \item $\hat{S}(n)$ (impulsions unités) de la \sdf{} est une densité discrète $V/Hz$~;
    \item $\hat{S}(f)$ (diracs) de la \sdf{} est une densité en $V/Hz$~;
    \item au moins avec un choix unique, je fais qu'une erreur par question...
    \end{enumerate}

      \graphe{0.7\textwidth}{transformees}
\end{description}
\end{quizz}

\section{Dérivée de fonctions discontinues}
\label{sec:deriv_discontinues}
La densité de fonction discrètes nous pousse à dériver des fonctions
discontinues de première espèce (pas d'infinité à droite ni à gauche). 

Prenon la fonction de \Heaviside{} ou échelon unité noté $u(t)$ pour
modéliser toute discontinuité dans un signal, le but est de définir
une dérivée de cette fonction, voir~\figref{fig:derivee_echelon} et
d'obtenir à nouveau $u(t)$ par intégration.

\begin{figure}[ht!]
  \centering
  \graphe{0.8\textwidth}{derivee_echelon}  
  \caption{Dérivées de l'échelon, de fonctions constantes et d'une fonction continue convergeant vers l'échelons pour $T\to0$}
  \label{fig:derivee_echelon}
\end{figure}

Or la dérivée $u'(t)$ n'est pas définie en $0$ (elle vaudrait $+\infty$), il faut donc trouver un prolongement pour la définir partout~:
\begin{itemize}
\item $u'(0)=0$ par continuité, on obtient alors la fonction nulle. Sa primite est la fonction nulle et donc on ne retrouve pas $u(t)$ en intégrant la dérivée~;
\item $u'(0)=a, a\in \R$, la fonction est discontinue en $0$, on doit intégrer en deux fois $$u(t>0)=\intDx{-\infty}{t>0}{u'(x)}=\lim\limits_{\epsilon\to 0}\intDx{-\infty}{-\epsilon}{\caCest{u'(x)}{=0}}+\lim\limits_{\epsilon\to 0}\intDx{\epsilon}{t}{\caCest{u'(x)}{=0}}=0$$ 
\end{itemize}

La dérivée de l'échelon est donc la fonction nulle \og{} presque partout \fg{} ( $\norme{u'}_2=0$).

On peut essayer d'aprocher l'échelon avec deux fonctions rampes comme
illustré sur la figure, et obtenir une dérivée définie partout (sauf
en $-\frac{T}{2}$ et $\frac{T}{2}$) sous la forme de la fonction porte
$\porteDe{-\frac{T}{2}}{\frac{T}{2}}\p{t}$. Mais la limite de cette
fonction porte reste la fonction nulle presque partout !


\section{Opérateur dérivée et sa réciproque}
\label{sec:dirac_derivee}
Le fait que plusieurs fonctions ont pour image la même fonction
nulle, rend l'opérateur dérivée (noté $p$ en calcul opérationnel ou
$D$) font que l'application dérivée qui transforme une fonction de
$L_1$ en fonction de $L_1$ n'est pas injective.

\begin{equation}
  \label{eq:application_derivee}
  p=D=\dDtDe{} : \application{L_1}{L_1}{t\mapsto u(t)}{t\mapsto u'(t)}
\end{equation}


Elle est donc non bijective et une application réciproque unique
n'existe pas (l'opérateur primitive qui s'annule en 0). Pourtant
l'intuition montre que l'on a besoin d'établir un objet mathématique
qui soit la dérivée de l'échelon et dont une primitive soit
$u(t)$. Cette fonction généralisée ou fonction imaginaire est
l'impulsion de \Dirac{} introduite et utilisée par le scientifique
éponyme dans les année 1920 et établie mathématiquement par Schwartz
dans les année 1950.

Comme l'indique la \figref{fig:fonctions_et_distributions} on crée une
fonction imaginaire étant la dérivée de discontinuité et dont
l'intégrale réciproque donne une discontinuité.
\begin{figure}[ht!]
  \centering
  \graphe{\textwidth}{fonctions_et_distributions}
  \caption{L'espace des fonctions usuelles où l'opérateur dérivée n'est pas injectif, et l'espace des distributions où l'opérateur est bijectif. }
  \label{fig:fonctions_et_distributions}
\end{figure}


Cela permet de définir un opérateur inversible et d'introduire le
calcul opérationnel où l'opérateur $D$ possède une réciproque $D^{-1}$
tel que $D\circ D^{-1}=\Id$. La notation $D^{-1}$ n'est pas anodine
car on assimile une réciproque d'opérateur à un inverse algébrique (du
nombre $D$). C'est l'approche par \emph{calcul opérationnel} vu dans
la \secref{sec:calcul_operationnel} pour les systèmes discrets.

\subsection{Calcul opérationnel en continu}
On obtient donc dans l'espace des distribution, noté $\Dis$, un
opérateur $\oder$ qui admet un réciproque $\oint=\oder\pmu$ capable de
commuter, s'associer se distribuer et s'inverser avec l'addition~: une
algèbre d'anneau commutatif comme celles de $+$ et $\times$ sur les
réels. L'approche opérationnelle assimile c'est opérateur $\oder$ à un
nombre complexe $p$ que l'on manipule algébriquement~: ce nombre peut
être assimilé à la variable de $\Laplace$ notée $s$ lorsque le signal
de référence choisi est l'impulsion de $\Dirac$.

Tout système linéaire invariant continu est représenté par une
équation différentielle
$$
a_n\,y^{\b{n}}+\ldots+a_1\,y'+a_0\,y \;=\; b_m\,e^{\b{m}} + \ldots +
b_1\,e' + b_0\,e
$$

que l'on peut noter sous forme opérationnelle :
$$
a_n\,\caCest{D\circ\ldots\circ D}{\text{n fois}}\b{y} + \ldots +a_1\,
D\b{y} + a_0\b{y} = b_m\,\caCest{D\circ\ldots\circ D}{\text{m fois}}\b{e} + \ldots +b_1\,
D\b{e} + b_0\b{e}
$$

Pour pouvoir manipuler les signaux comme des opérateurs et pouvoir
écrire $\oder\circ y= y\circ\oder$ permettant d'avoir une algèbre
d'anneau commutatif. On assimile un signal $y$ au système $Y$
permettant de le générer à partir de sa réponse impulsionnelle
(réponse à une impulsion de \Dirac{} $\delta_0$~:
$$
y \quad\leftrightarrow\quad Y \text{ tel que } \; Y\b{\delta_0}=y
$$

Nous avons donc les opérateurs élémentaires gain $a_n$, dérivée $\oder$
et l'opérateur $Y$ associé au signal $y$ qui se composent et
commutent~: $a_n\circ\oder\circ Y = Y\circ\oder\circ a_n =
\dots$. Comme avec ces opérateurs élémentaires les opération $\circ$
et $+$ ont une structure d'anneau commutatif, on adopte la notation
algébrique classique pour ces opérateurs~:
$$
a_n\,D^n\,Y + \ldots +a_1\, D \, Y + a_0 \,Y = b_m\,D^m.E + \ldots +b_1\, D\, E + b_0 \,E
$$

Comme l'opérateur $D$ possède une réciproque notée $D^{-1}$ puisque
$D\circ D^{-1}= Id$ et que ces opérateurs commutent, on peut associer
$\oder$ à la variable scalaire $p$, l'opérateur gain $a_n$ au nombre
$a_n$ (réel ou complexe) et obtenir une équation algébrique~:

\begin{equation}
  \p{a_n\,p^n + \ldots +a_1\, p + a_0 }\, Y = \p{b_m\,p^m + \ldots +b_1\, p + b_0 }\,E
  \label{eq:equation_algebrique_p}
\end{equation}

Comme le système identité est $\Id=\oder\,\oder\pmu$ nous obtenons
algébriquement que le système identité est le système de gain $1$ et
donc associé au nombre $1$ qui représente donc l'impulsion de \Dirac{}
de poids $1$.

Si l'on borne les signaux $y$ et $e$ à des signaux générés par des
systèmes linéaire invariants $Y$ et $E$ alors nous obtenons une
représentation du signal $y$ par une fraction rationnelle $Y(p)$ et
$E(p)$ en appliquant l'équation \eqref{eq:equation_algebrique} pour
une entrée impulsionnelle qui vaut $1$. Donc on associe le système $Y$
à une fonction algébrique de $p$ notée $Y\de{p}$. De même pour le
système $E$ associé à la fonction $E\de{p}$

C'est ainsi que \Heaviside{} à remarqué que la composition des
dérivées de fonctions et leurs associations linéaires dans les
équations différentielle se comportait comme une algèbre
classique. L'idée est venue de manipuler $D$ comme un nombre ce qui
l'a conduit à l'utilisation du calcul opérationnel comme celui
effectué avec la transformée de \Laplace{} où $p$ joue le rôle de
l'opérateur dérivé mais est une nombre complexe.

Et donc passer le pas et de calculer la solution en faisant une
fraction rationnelle d'opérateur dérivés et une décomposition en
éléments simples d'opérateurs dérives comme pour des nombres avec la fameuse \emph{fonction de transfert} du système~:
$$
\frac{Y\de{p}}{E\de{p}} = \frac{\p{b_m\,p^m + \ldots +b_1\, p + b_0}}{\p{a_n\,p^n + \ldots +a_1\, p + a_0}}=
\frac{\beta_n}{p + \alpha_n} + \ldots + \frac{\beta_0}{p +
  \alpha_0}
$$

Le tout sans avoir défini la transformée de \Laplace{} qui est venue
plus tard avec la formule de Carson.


\section{Propriété sous l'intégrale de l'impulsion de \Dirac{} et de l'impulsion unité}
\label{sec:dirac_sous_integrale}
Nous avons donc une fonction de densité $\delta_0$ qui une fois
intégrée peut enfin donner une fonction de répartition discontinue en
$0$. On peut appliquer l'opérateur de retard sur cet objet et définir
$\delta_a$ la densité d'une fonction $u(t-a)$ discontinue en $a$. On peut définir par abus de notation~:
$$
\delta_a(t)= \delta_0(t-a)
$$

Cela permet aussi de représenter l'opérateur d'échantillonnage idéal d'un signal continu à un instant $a$ donné en multipliant le signal $s(t)$ par la densité de mesure en $a$ ce qui est représenté par la~\figref{fig:delta_mesure}
\begin{figure}[ht!]
  \centering
  \graphe{\textwidth}{delta_mesure}
  \caption{A droite l'impulsion unitaire discrète qui effectue une mesure "sous la somme" ou à travers le produit scalaire. A gauche l'impulsion de \Dirac{} qui effectue la mesure idéale d'une fonction continue.}
  \label{fig:delta_mesure}
\end{figure}

Nous avons donc les propriétés~:
$$
\scal{s}{\delta_a}=s(a) \text{ en continu et en discret } \scald{s}{\delta_a}=s[a] 
$$

\section{Propriété de convolution avec le \Dirac{} et impulsion unité}

On peut définir l'opérateur de convolution, noté $\conv$, à l'aide du produit scalaire puisque~:
\begin{equation}
  \label{eq:convolution}
  u\conv v(t) = \intDx{-\infty}{\infty}{u(x)\,v(t-x)} = \scal{u}{\conj{x\mapsto v(t-x)}\,}
\end{equation}

De même pour les signaux discret à support infini, on définit la convolution~:
\begin{equation}
  \label{eq:convolution_discrete}
  u\conv v[k] = \somme{-\infty}{\infty}{u[l]\,v[k-l]} = \scald{u}{\conj{l\mapsto v(k-l)}}
\end{equation}

Et la convolution cyclique pour les signaux discrets périodiques de $N$ points~:
\begin{equation}
  \label{eq:convolution_discrete}
  u\conv v[k] = \somme{0}{N-1}{u[l]\,v[k-l]} = \scaldp{u}{\conj{l\mapsto v(k-l)}}
\end{equation}


%%%Local Variables:
%%% mode: latex
%%% TeX-master: "main"
%%% End:
