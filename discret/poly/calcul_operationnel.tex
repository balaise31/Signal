\section{Calcul opérationnel~: Transformée en Z}

La \teZ{} joue le même rôle pour les systèmes discrets que la
transformée de \Laplace{} pour les systèmes continus. Elles permettent
de définir une \emph{fonction de transfert} du système et de manipuler
les systèmes algébriquement sous formes de fonctions de la variable
complexe (comme des fractions rationnelles polynomiales pour les
systèmes linéaires invariants).

Plutôt que de présenter sans justifier ces formules de transformées,
nous allons introduire la vision par calcul opérationnel qui consiste
justement à manipuler les opérateurs comme des nombres
algébriques. Cette méthode a été développée et fortement utilisée par
\Heaviside{} pour résoudre notamment l'équation des
télégraphistes. Les justifications théoriques de cette approche et la
relation entre l'opérateur $p$ de \Heaviside{} et la variable $s$ de
la transofrmée de \Laplace{} n'est apparue que plus tard.

\begin{citations} Mathematics is an experimental science, and definitions do not come
  first, but later on.

  \hfill Olivier \Heaviside{} (1850--1925)
\end{citations}

\subsection{Principe du calcul opérationnel}
\label{sec:calcul_operationnel}
Tout système discret linéaire invariant possédant une seule entrée $x$ et
une seule sortie $y$ se représente par une équation aux différences du
type suivant~:
\begin{eqnarray}
  \label{eq:systeme_recurrence}
  a_n\,y\b{k-n} \,+\, \ldots  \,+\,  a_1\,y\b{k-1} \,+\, a_0.y\b{k} \;=\; b_m\,x\b{k-m} \,+\, \ldots \,+\, b_0 x\b{k} \quad,\quad \forall k\in\Z
\end{eqnarray}
Les opérateurs discrets de base, retard unitaire, avance unitaire,
gain commutent entre-eux (pour les systèmes invariants) et se combinent
linéairement (pour les systèmes linéaires). On peut donc représenter la
relation entrée/sortie par une combinaison linéaire de ces opérateurs
de base~:
\begin{eqnarray}
  \label{eq:systeme_operationnel}
  \p{a_n\,\circ\,\caCest{\oret\circ...\circ \oret}{\text{n fois}}}\b{y}\; +\; \ldots \;+\; \caCest{\p{a_1\circ\oret}\b{y}}{k\mapsto a_1.y\b{k-1}} \;+\; \caCest{\p{a_0 \,\circ\, \Id}\b{y}}{a_0.y} \\
  =\quad \p{b_m\,\circ\,\caCest{\oret\circ...\circ \oret}{\text{m fois}}}\b{x} \;+\; \ldots \;+\; \p{b_0 \,\circ\,\Id}\b{x}
\end{eqnarray}

\begin{remarque}
  Remarquons bien que dans \eqref{eq:systeme_recurrence} les termes
  sont des scalaires réels ou complexe~; alors que dans l'écriture
  opérationnelle \eqref{eq:systeme_operationnel} les termes sont des
  fonctions (ou signaux ou plutôt des suites réelles ou complexes). Au
  lieu de prendre une égalité valable pour tout entier $k$~; nous
  passons à une équation de systèmes (ou opérateurs) prenant en
  argument des signaux.
\end{remarque}


Comme l'opérateur gain est invariant et qu'il commute avec l'opérateur
retard on peut noter la composition $\circ$ comme un simple produit
car elle possède les mêmes propriétés de commutativité, associativité
etc. La récurrence devient ainsi~:
\begin{eqnarray}
 \caCest{\p{a_n.\oret^n}}{\text{opérateur}}\caCest{\b{y}}{\text{de fonction}}\; +\; \ldots \;+\; \p{a_1.\oret}\b{y} \;+\; \caCest{a_0\b{y}}{k\mapsto a_0\,y\b{k}}  = \quad \p{b_m.\oret^m}\b{x} \;+\; \ldots \;+\; b_0\b{x}
\end{eqnarray}

\begin{exemple}
  Reprenons l'exemple du \emph{différentiateur} \ref{exemple:differentiateur_lineaire} d'écrit par l'opérateur ~: $$L : x \mapsto y=\frac{x-\oret\b{x}}{T_e}$$
  
  Nous obtenons avec la notation algébrique la relation entrée/sortie~:
  $$
  T_e.y = x - \oret\b{x}
  $$
  Qui correspond à l'équation aux différences~:
  $$
  y\b{k} = \caCest{\frac{1}{T_e}}{b_0}\,x\b{k}-\caCest{\frac{1}{T_e}}{b_1}\,x\b{k-1}
  $$
  
\end{exemple}

\begin{remarque}
  On ne peut pas noter des produits du type $a_n.T^n.y$, car cela
  signifierait la composition $a_n\circ T^n\circ y$. Alors que les
  systèmes gain et retard se composent et commutent (car se sont des
  opérateurs qui transforment des signaux discrets de $L_E$ en signaux
  discrets de $L_E$), la fonction $y$ est un signal discret de $L_E$
  qui transforme un entier $\Z$ en un scalaire complexe $\C$ et ne
  peut se composer avec un opérateur~: la composition (à droite ou à
  gauche) serait mal définie car~:
  \begin{equation*}
    \begin{array}{ccccc}
      a_n   & \circ &   T^n  & \circ &  y \\
      L_E \overset{a_n}{\longleftarrow} L_E & =  & L_E \overset{\oret^n}{\longleftarrow} L_E & \neq  & \C \leftarrow \Z 
    \end{array}
  \end{equation*}

  Rappelons que les opérateurs $a_n$ et $\oret$ sont des
  \og{}fonctions de fonctions discrètes donnant des fonctions
  discrètes\fg{}.

  La composition $a_n\circ y$ est donc mal définie et sa notation
  $a_n.y$ est fausse quand $a_n$ désigne l'opérateur (ou système)
  gain. En revanche la notation $a_n.y$ où $a_n$ est un scalaire est
  ambiguë mais correcte car elle peut désigner le produit d'un
  scalaire par une fonction. A part le système gain qui est noté comme
  un scalaire, les autres opérateurs comme $T^k$ ne se composent pas
  avec une fonction et les notation $T^k.y$ et $T^k\circ y$ restent
  fausse sans ambiguïté~!
\end{remarque}



Pour mener une approche par calcul opérationnel, il faut transformer
la fonction $y$ en un opérateur qui puisse commuter avec les autres.
Or, comme le montre la remarque précédente, une fonction discrète prend
un entier en argument pour donner un complexe, alors qu'un opérateur
(ou système) prend une fonction pour la transformer en fonction.

Pour contourner ce problème, on remplace le signal $y$ par un système
noté $Y$ dont la réponse à une excitation unitaire est le signal $y$
lui-même. Dans le cas de systèmes discrets, on choisi comme signal
unitaire l'impulsion unité $\delta_{0}$.

\begin{definition}
  \label{def:impulsion_unite}
  L'impulsion unité, notée $\delta_0$ ou simplement $\delta$, est le signal discret tel que :
  $$
  \delta_0\b{k}=\pparMorceaux{1}{\text{si } k=0}{0}{\text{sinon}} \quad k\in\Z
  $$

  L'impulsion unités centrée en $a$ est notée $\delta_a$ et définie par~:
  $$
  \delta_a\b{k}=\delta_0\b{k-a} = \pparMorceaux{1}{\text{si } k=a}{0}{\text{sinon}}
  $$

  Bien qu'utilisant le même symbole $\delta$, il ne faut pas confondre
  l'impulsion unité discrète avec l'impulsion de \Dirac. L'impulsion
  unité est un signal discret tout à fait classique d'amplitude égale
  à $1$ alors que l'impulsion de \Dirac{} est une fonction généralisée
  ou distribution, voir \chapref{sec:dirac}, d'amplitude infinie et de
  poids unité.
\end{definition}

Ainsi au lieu de considérer un signal $y$, on considère le système discret $Y$ dont la réponse impulsionnelle est~:
\begin{equation}
  y = Y\!\b{\delta_0}
\end{equation}



On exprime ainsi l'équation aux différences sous la forme pure
d'opérateurs, ou systèmes, qui commutent entre-eux et se distribuent
avec l'addition tout comme une multiplication classique~:

\begin{eqnarray}
  a_n.\oret^n.Y\; +\; \ldots \;+\; a_1.\oret.Y \;+\; a_0.Y \quad  = \quad b_m.\oret^m.X \;+\; \ldots \;+\; b_0.X
\end{eqnarray}

\begin{remarque}
  Dans le cas des systèmes continus, on exprime les équations
  différentielles sous forme opérationnelle en remplaçant l'opérateur
  discret de retard $\oret$ par l'opérateur de dérivation $\oder$. Un
  signal $y$ est de même remplacé par un système $Y$ dont la réponse
  impulsionnelle (à une impulsion de \Dirac{} cette fois-ci) est le
  signal $y$.

  Initialement, \Heaviside{} avait introduit l'échelon unité, ou
  échelon éponyme, comme signal d'excitation de référence à la place
  de l'impulsion de \Dirac{} qui n'était pas encore définie à
  l'époque. Voir le \secref{sec:dirac_derivee} pour une définition de
  l'opérateur réciproque de la dérivée nécessitant l'impulsion de
  \Dirac{}.
\end{remarque}

Nous obtenons avec cette notation une écriture de l'équation aux
différences qui ressemble à une équation algébrique polynomiale
classique. Dans le calcul opérationnel, l'opérateur d'avance
$\oavance$ (resp. $\oret$) est assimilé à un nombre que l'on notera
$z$ (resp. $\zmu$), les signaux $x$ et $y$ sont remplacés par leurs
systèmes générateurs $X$ et $Y$ à partir de leur réponse
impulsionnelle. Les systèmes générateurs $X$ et $Y$ pouvant être
eux-même exprimés en fonction de l'opérateur $z$, ils sont représentés
comme des fonctions de $z$ soit $X\p{z}$ et $Y\p{z}$. Nous verrons
dans la suite que les fonctions $X\p{z}$ et $Y\p{z}$ sont les
transformées en $\TZ$ des signaux (ou systèmes) $x$ et $y$.

Nous obtenons finalement l'équation algébrique associée à la récurrence
\eqref{eq:systeme_recurrence}~:
\begin{eqnarray}
  \label{eq:systeme_algebrique}
  a_n\,z^{-n}\,Y\p{z}\; +\; \ldots \;+\; a_1\,\zmu\;Y\p{z} \;+\; a_0\,Y\p{z} \quad  = \quad b_m\,z^{-m}\,X\p{z} \;+\; \ldots \;+\; b_0\,X\p{z}
\end{eqnarray}

Les opérateurs réciproques $\oret$ et $\oavance$ sont associés aux
nombres $z$ et $\zmu$ car la division et la multiplication sont
réciproques~: comme la composition d'une avance et d'un retard
$\oavance \circ \oret = \Id$ donne le système identité, le produit
algébrique $z\,\zmu=z\,\frac{1}{z}=1$ donne l'unité. L'unité
algébrique $1$ est donc associée au \og{}système identité \fg{} (qui
ne change pas le signal) dont la réponse impulsionnelle est
l'impulsion unité $\delta_0$.

La résolution de l'équation aux différences peut alors se faire en
traitant l'équation algébrique sous forme de fraction rationnelle puis
de décomposition en éléments simples~:
\begin{eqnarray}
 \frac{Y\p{z}}{X\p{z}} = \frac{b_m\,z^{-m} + \ldots + b_0}{a_n\,z^{-n}+ \ldots +  a_1\,\zmu + a_0} = \caCest{\frac{\beta_0}{z-\alpha_0}}{\text{premier ordre}} + \ldots + \caCest{\frac{\mu_0+\nu_0\,z}{z^2+b_0\,z+c_0}}{\text{second ordre}} + \ldots
\end{eqnarray}

On décompose alors un système linéaire invariant comme une combinaison
linéaire de systèmes de premier ordre et de second ordre. La
résolution se fait alors par lecture de table de \teZ{} comme pour les
transformées de \Laplace{} dans le cas des systèmes continus.

\begin{exemple}
  \label{exemple:forward_euler}
  Dans l'exemple du \emph{différentiateur}
  \ref{exemple:differentiateur_lineaire} d'écrit par
  $T_e.y = x - \oret\b{x}$, nous pouvons remplacer $x$ et $y$ par les
  systèmes générateurs $X\p{z}$ et $Y\p{z}$ et finalement remplacer la
  composition avec $\oret$ par une multiplication par $\zmu$. On
  obtient la fonction de transfert du système différentiateur~:
  $$H_d\de{z}=\frac{Y\de{z}}{X\de{z}}= \frac{1-\zmu}{T_e}$$

  Il est alors facile de trouver l'opérateur \emph{intégrateur} $H_i$
  réciproque du différentiateur $H_d$ en se basant sur la propriété
  $H_i\circ H_d = H_d \circ H_i = \Id$ qui donne en équation
  algébrique~:
  $$
  H_i\de{z}\,H_d\de{z}=1 \implies H_i\de{z} = \frac{1}{H_d\de{z}} = \frac{Y\de{z}}{X\de{z}}=\frac{T_e}{1-\zmu}
  $$
  On obtient ainsi l'équation de récurrence de l'intégrateur dit \emph{Backward Euler}~:
  \begin{align}
    \label{eq:forward_euler}
    Y\de{z}\p{1-\zmu}=T_e\, X\de{z}  \iff & Y\de{z}=\zmu\,Y\de{z} + T_e\, X\de{z}&\nonumber\\
                                          &  y\b{k} = y\b{k-1} + T_e\, x\b{k} &, \forall k\in\Z \\
    \text{ou bien }   \quad                     &  y\b{k+1} = y\b{k} + T_e\, x\b{k+1} &, \forall k\in\Z \nonumber
  \end{align}
\end{exemple}

\begin{exercice}
  \exerciceTitre{Trois intégrateurs différents et trois différentiateurs associés}

  L'exemple~\ref{exemple:forward_euler} pécédent de l'intégrateur
  \emph{Backward Euler} est illustré ci-dessous avec deux autres
  méthodes. On identifie alors dans \eqref{eq:forward_euler} que
  l'incrément de surface $ds$ ajouté à l'intégrale de $x$ à l'instant
  $k+1$ est la surface du rectangle bleu~:
  $y\b{k+1} = y\b{k} + \caCest{T_e\,x\b{k+1}}{ds}$

  \graphe{0.9\textwidth}{integrales}

  \begin{itemize}
  \item Écrivez alors les récurrences correspondantes aux intégrateurs
    \emph{Forward Euler} et \emph{trapézoïdale} en adaptant la valeur
    de l'incrément de surface $ds$ en fonction de $T_e$, $x\b{k}$
    et/ou $x\b{k+1}$.
  \item De manière inverse à l'exemple précédent, retrouvez les
    fonctions de transfert $H_i\p{z}$ de ces trois intégrateurs
    (remplacer $x\b{k}$  par $X\p{z}$, $x\b{k+1}$ par $z\,X\p{z}$ car
    $z$ est associé à l'avance unitaire).
  \end{itemize}
  On remarque que l'écriture de la récurrence en
  $y\b{k+1}=y\b{k}+\ldots$ donne naturellement une fonction de
  transfert exprimée en $z$, alors que l'écriture en
  $y\b{k}=y\b{k-1}+\ldots$ donne une écriture en $\zmu$ parfaitement
  équivalente~: par exemple pour le \emph{Backward Euler} on obtient
  les fonctions de transfert
  $H_i\p{z}=\frac{T_e}{1-\zmu}=\frac{T_e\,z}{z-1}$

  \begin{itemize}
  \item On peut alors inverser algébriquement ces fonctions de
    transfert d'intégrateur $H_i$ pour obtenir des fonctions de
    transfert de dérivateurs $H_d\p{z}=H_i\p{z}^{-1}$ associées.
  \item On peut, de même, donner les récurrences $y_d\b{k}=\ldots$ à
    partir des fonctions de transfert $H_d\p{z}$ permettant d'obtenir
    différentes approximations de la dérivée du signal d'entrée $x$.
  \end{itemize}

  On obtient ainsi des approximations linéaires discrètes exprimées en
  $z$ (l'avance unitaire) de l'opérateur dérivée en continue $p$ (ou
  variable de \Laplace{} notée $s$)~:
  \begin{equation}
    \label{eq:approx_de_p}
    \oder=\dDtDe{} \leftrightarrow p \leftrightarrow \caCest{\frac{1}{T_e}\p{1-\zmu}}{\text{Forward Euler}}\leftrightarrow \caCest{\frac{1}{T_e}\p{z-1}}{\text{Backward Euler}} \leftrightarrow \caCest{\frac{2}{T_e}\frac{1-\zmu}{1+\zmu}}{\text{Bilinéaire ou Tustin}} 
  \end{equation}  
\end{exercice}
