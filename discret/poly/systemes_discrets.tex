\chapter{Systèmes discrets}


L'analogie avec les systèmes continus est forte, nous étudions de même
le cas des sysèmes linéaires invariants dans le temps avec une vision
par opérateurs.


\section{Systèmes linéaires}

\begin{definition}{Système}
  
  Un système discret (resp. continu) relie à chaque signal d'entrée $x$
  un signal de sortie unique $y$. Les signaux $x$ et $y$ sont des
  fonctions de la variables réelle discrète $k$ (resp. $t$) appartenant
  à un espace de fonction le plus général possible noté ici $L_E$. La
  relation entrée-sortie est donc modélisée par une application
  mathématique de $L_E$ dans $L_E$ notée $L$ (resp. $L_c$ en continu) et
  définie ainsi~:
  \begin{eqnarray}
    L : \qquad \application{L_E}{L_E}{x : \; k\mapsto x\b{k}\quad{}}{\quad y :\; k\mapsto y\b{k}} \\
    L_c: \qquad \application{L_E}{L_E}{x : \; t\mapsto x\p{t}\quad{}}{\quad y :\; t\mapsto y\p{t}} 
    % DONE : correction TYPAGE et remarque notation
  \end{eqnarray}
\end{definition}

Une classe de système fondamentale est la classe des systèmes
linéaires car elle offre de nombreux outils et propriétés
mathématiques.

\begin{definition}{Système linéaire}
  \label{def:linearite}
  
  Une système est dit linéaire si et seulement si
  l'application $L$ associée est linéaire, soit pour tout
  $\p{x_1,x_2,\lambda} \in L_E^2 \times \R$~:
  \begin{eqnarray}
    \label{eq:def_linearite}
    \forall t \in \R \qquad L\b{x_1 + \lambda\, x_2}(t) = L\b{x_1}(t) + \lambda\,L\b{y_2}(t) \nonumber\\
    \text{ou bien } \qquad L\b{x_1 + \lambda\,x_2} = L\b{x_1} +\lambda L\b{x_2 }
  \end{eqnarray}
\end{definition}

Une des conséquences de la linéarité est la possibilité d'appliquer le
principe de superposition cher à l'électronicienne~: la réponse du
système à une entrée est la combinaison linéaire des réponses à chaque
signaux composant l'entrée par combinaison la même combinaison
linéaire.

Les trois systèmes linéaires de base que nous considérons dans
l'étude des systèmes linéaires continus sont :
\begin{description}
\item[le gain $a.x$]~: $t \mapsto a\,x\p{t}$ où a est une constante
\item[le dérivateur $\oder\b{x}=\oder\circ x$]~: $ t \mapsto \dDtDe{x}\p{t} $
\item[l'intégrateur $\oint\b{x}=\oint\circ x$]~: $ t \mapsto \integ{0}{t}{x\p{\nu}\derivDe{\nu}}$ 
\end{description}
On peut aisément vérifier que ces systèmes respectent la condition de
linéarité~\ref{eq:def_linearite}. On aimerait que les opérateurs
dérivateur $\oder$ et intégrale $\oint$ commutent et soient
réciproque~: $\oder\circ\oint=\oint\circ\oder=\Id$. Pour que cela soit
vrai même avec les fonctions discontinues, il faut introduire les
distributions de \Dirac{} voir \chapref{sec:deriv_discontinues}.

Dans le cas des systèmes discret, les systèmes élémentaires sont :
\begin{description}
\item[le gain $a.x$]~: $ k \mapsto a\,x\b{k}$ où a est une constante.
\item[le retard unité $\oret\b{x}=\oret\circ x$]~: $ k \mapsto x\b{k-1}$.
\item[l'avance unité $\oavance\b{x}=\oavance\circ x$]~: $ k \mapsto x\b{k+1} $ 
\end{description}

On peut aisément vérifier que ces systèmes respectent la condition de
linéarité~\ref{eq:def_linearite}. La commutation et réciprocité des
opérateurs retard $\oret$ et avance $\oavance$ est évidente et ne pose
pas de problème théorique.

\begin{exemple}
  \label{exemple:differentiateur_lineaire}
  L'effet du système \emph{différentiateur} sur le signal d'entrée est d'écrit par
  l'opérateur ~: $$L : x \mapsto y=\frac{x-\oret\b{x}}{T_e}$$
  
  Vérifions d'abord que cet opérateur est linéaire~:
  \begin{eqnarray*}
    L\b{x_1+\lambda x_2}\b{k} &= L\b{x_1+\lambda x_2}\b{k}\\
                              &= \frac{x_1\b{k}+\lambda x_2\b{k} - \oret\b{x_1+\lambda x_2}\b{k}}{T_e}\\
                              &= \frac{x_1\b{k}+\lambda x_2\b{k} - \p{x_1\b{k-1}+\lambda x_2\b{k-1}}}{T_e} = A\\
    L\b{x_{1}}\b{k}+\lambda\,L\b{x_{2}}\b{k}  &= \frac{x_1\b{k} - x_1\b{k-1}}{T_e}+\lambda\,\frac{x_2\b{k} - x_2\b{k-1}}{T_e}=A
  \end{eqnarray*}
  Le système est donc linéaire.
\end{exemple}


% \begin{remarque}
%   Pour résoudre les complexes équations différentielles des
%   télégraphistes, \Heaviside{} utilise ces opérateurs de base et
%   introduit le \emph{calcul opérationnel}. Cela consiste à représenter
%   l'application de l'opérateur dérivée sur une fonction $f$ comme une
%   simple multiplication par un nombre $p$. Ainsi une équation
%   différentielle $a\,y'' + 2y' -y = 3\int x$ est associée à l'équation
%   symbolique $a\,p^2\,y + 2\,p\,y - y = \frac{3}{p}x$. Il est alors
%   possible de résoudre algébriquement l'équation sous forme de
%   fractions rationnelles ce qui donnerait avec notre exemple~:
%   $y = \frac{3/p}{a\,p^2+2p+1} x$. Rappelons que $x$ et $y$ sont des
%   fonctions et non des réels et que dans ce cas les opérations ne sont
%   pas de simple multiplication et addition de réels mais bien des
%   multiplications et additions de fonctions. La variable symbolique
%   $p$ est utilisée comme un nombre réel mais n'est en aucun cas un
%   réel...
% \end{remarque}

\section{Systèmes invariants}
Il est fréquent, et surtout théoriquement utile, qu'un système
réagisse de la même manière indépendemment de l'instant où est
appliqué le signal d'entrée. Ce qui conduit à la définition suivante~:
\begin{definition}{Système invariant dans le temps}
  Un système discrète (resp. continu) est dit invariant dans le temps si et seulement
  si son application associée $L$ (resp. $L_c$) vérifie~:
  \begin{eqnarray}
    \forall x\in L_E, \forall (k,k_0)\in \N^2, \quads L[k\mapsto x(k-k_{0})] = L[x](k-k_{0}) \\
    \forall x\in L_E, \forall (t,t_0)\in \R^2, \quads L_c[t\mapsto x(t-t_{0})] = L[x](t-t_{0}) 
  \end{eqnarray}
  
  En terme d'opérateur~;  un système $L$ est invariant dans
  le temps si, et seulement si, son opérateur commute avec  tout opérateur
  retard de $k_0$ noté $\oretDe{k_0}=\oret^{k_0}\b{x}= k\mapsto x\b{k-k_0}$~:
  \begin{eqnarray}
    \label{eq:sys_invariant}
    L\circ\oretDe{k_0} \; = \; \oretDe{k_0}\circ L  \qquad \iff \qquad  L\b{\oretDe{k_0}\b{x}} = \oretDe{k_0}\b{L\b{x}}
    \\
    L_c\circ\oretDe{\tau} \; = \; \oretDe{\tau}\circ L_c  \qquad \iff \qquad  L_c\b{\oretDe{\tau}\b{x}} = \oretDe{\tau}\b{L_c\b{x}}
  \end{eqnarray}
\end{definition}

Autrement dit \og{} la réponse du système à un signal retardé est le
retard de la réponse du système.\fg{} 

En d'autres termes, la réponse du système ne dépend pas de l'origine
des temps choisie.

\begin{exemple}
  Reprenons l'exemple du \emph{différentiateur} \ref{exemple:differentiateur_lineaire} d'écrit par l'opérateur ~: $$L : x \mapsto y=\frac{x-\oret\b{x}}{T_e}$$
  
  Vérifions qu'il est invariant~:
  \begin{eqnarray*}
    \forall k_0 \quads L\b{\oretDe{k_0}\b{x}}\b{k} &= L\b{k\mapsto x\b{k-k_0}}\b{k}\\
                                                   &= \frac{x\b{k-k_0}-x\b{k-k_0-1}}{T_e} = B\\
    \forall k_0 \quads \oretDe{k_0}\b{L\b{x}}\b{k} &= \oretDe{k_0}\b{k\mapsto \frac{x\b{k-1}-x\b{k}}{T_e}}\b{k}\\
                                                   &= \frac{x\b{k-k_0}-x\b{k-k_0-1}}{T_e} = B
  \end{eqnarray*}
  
  On a bien exprimé le fait que $L$ commute avec tout retard
  $\oretDe{k_0}$, en d'autres termes~: \emph{la différence du retard est le retard de la différence.}
\end{exemple}


\begin{remarque}
  Il est facile de vérifier que les systèmes discrets élémentaires que
  sont le gain~; le retard unitaire et l'avance unitaire (gain,
  dérivateur et intégrateur pour le continu) sont invariants.  Il en
  est de même pour tout système constitué de combinaisons linéaires
  et de composition de systèmes élémentaires.
  
  Il suffit alors de montrer que le système se décompose avec des
  \emph{coefficients constants} avec des systèmes élémentaires en le
  mettant sous forme \emph{d'équation aux différence} ou
  \emph{récurrence} à coefficients constants (\emph{equations
    différentielle} en continu) ou en un schéma bloc à coefficients
  constant.
\end{remarque}

\section{Calcul opérationnel~: \teZ{}}

Nous allons présenter les signaux discrets de base et assimiler un
signal à un système en utilisant la réponse impulsionnelle comme dans
le calcul opérationnel développé par \Heaviside.

Tout système discret linéaire invariant possédant une seule entrée $x$ et
une seule sortie $y$ se représente par une équation aux différences du
type suivant~:
\begin{eqnarray}
  \label{eq:systeme_recurrence}
  a_n\,y\b{k-n} \,+\, \ldots  \,+\,  a_1\,y\b{k-1} \,+\, a_0.y\b{k} \;=\; b_m\,x\b{k-m} \,+\, \ldots \,+\, b_0 x\b{k} \quad,\quad \forall k\in\Z
\end{eqnarray}
Les opérateurs discrets de base, retard unitaire, avance unitaire,
gain commutent entre-eux (pour les systèmes invariants) et se combinent
linéairement (pour les systèmes linéaires). On peut donc représenter la
relation entrée/sortie par une combinaison linéaire de ces opérateurs
de base~:
\begin{eqnarray}
  \label{eq:systeme_operationnel}
  \p{a_n\,\circ\,\caCest{\oret\circ...\circ \oret}{\text{n fois}}}\b{y}\; +\; \ldots \;+\; \caCest{\p{a_1\circ\oret}\b{y}}{k\mapsto a_1.y\b{k-1}} \;+\; \caCest{\p{a_0 \,\circ\, \Id}\b{y}}{a_0.y} \\
  =\quad \p{b_m\,\circ\,\caCest{\oret\circ...\circ \oret}{\text{m fois}}}\b{x} \;+\; \ldots \;+\; \p{b_0 \,\circ\,\Id}\b{x}
\end{eqnarray}

\begin{remarque}
  Remarquons bien que dans \eqref{eq:systeme_recurrence} les termes
  sont des scalaires réels ou complexe~; alors que dans l'écriture
  opérationnelle \eqref{eq:systeme_operationnel} les termes sont des
  fonctions (ou signaux ou plutôt des suites réelles ou complexes). Au
  lieu de prendre une égalité valable pour tout entier $k$~; nous
  passons à une équation de systèmes (ou opérateurs) prenant en
  argument des signaux.
\end{remarque}


Comme l'opérateur gain est invariant et qu'il commute avec l'opérateur
retard on peut noter la composition $\circ$ comme un simple produit
car elle possède les mêmes propriétés de commutativité, associativité
etc. La récurrence devient ainsi~:
\begin{eqnarray}
 \caCest{\p{a_n.\oret^n}}{\text{opérateur}}\caCest{\b{y}}{\text{de fonction}}\; +\; \ldots \;+\; \p{a_1.\oret}\b{y} \;+\; \caCest{a_0\b{y}}{k\mapsto a_0\,y\b{k}}  = \quad \p{b_m.\oret^m}\b{x} \;+\; \ldots \;+\; b_0\b{x}
\end{eqnarray}

\begin{exemple}
  Reprenons l'exemple du \emph{différentiateur} \ref{exemple:differentiateur_lineaire} d'écrit par l'opérateur ~: $$L : x \mapsto y=\frac{x-\oret\b{x}}{T_e}$$
  
  Nous obtenons avec la notation algébrique la relation entrée/sortie~:
  $$
  T_e.y = x - \oret\b{x}
  $$
  Qui correspond à l'équation aux différences~:
  $$
  y\b{k} = \caCest{\frac{1}{T_e}}{b_0}\,x\b{k}-\caCest{\frac{1}{T_e}}{b_1}\,x\b{k-1}
  $$
  
\end{exemple}

\begin{remarque}
  On ne peut pas noter des produits du type $a_n.T^n.y$, car cela
  signifierait que les termes de $a_n\circ T^n\circ y$, qui est bien
  la fonction $k\mapsto a_n\,y\b{k-n}$ attendue, commutent. Or les
  systèmes gain et retard commutent mais pas la composition avec la
  fonction $y$ qui n'est pas un opérateur de fonctions~:
  $a_n\circ T^n\circ y \neq y\circ a_n\circ T^n$ car
  $y\circ a_n\circ T^n$ n'as pas de sens comme $y$ est une fonction
  $\N\rightarrow \C$ qui doit avoir un entier en argument alors que
  $a_n\circ T^n$ est un opérateur qui renvoie une fonction.
\end{remarque}



Pour mener une approche par calcul opérationnel, il faut transformer
la fonction $y$ en un opérateur qui puisse commuter avec les autres.
Or une fonction discrète prend un entier en argument pour donner un
complexe, alors qu'un opérateur (ou système) prend une fonction pour
la transformer en fonction.

Pour contourner ce problème, on remplace le signal $y$ par un système
noté $Y$ dont la réponse à une excitation unitaire est le signal
$y$ lui-même. Dans le cas de systèmes discrets, on choisi comme signal unitaire l'impulsion unité $\delta_{0}$

\begin{definition}
  \label{def:impulsion_unite}
  L'impulsion unité, notée $\delta_0$ ou simplement $\delta$, est le signal discret tel que :
  $$
  \delta_0\b{k}=\pparMorceaux{1}{\text{si } k=0}{0}{\text{sinon}} \quad k\in\Z
  $$

  L'impulsion unités centrée en $a$ est notée $\delta_a$ et définie par~:
  $$
  \delta_a\b{k}=\delta_0\b{k-a} = \pparMorceaux{1}{\text{si } k=a}{0}{\text{sinon}}
  $$

  Bien qu'utilisant le même symbole $\delta$, il ne faut pas confondre
  l'impulsion unité discrète avec l'impulsion de \Dirac. L'impulsion
  unité est un signal discret tout à fait classique d'amplitude égale
  à $1$ alors que l'impulsion de \Dirac{} est une fonction généralisée
  ou distribution, voir \chapref{sec:dirac}, d'amplitude infinie et de
  poids unité.
\end{definition}

Ainsi au lieu de considérer un signal $y$, on considère le système discret $Y$ dont la réponse impulsionnelle est~:
\begin{equation}
  y = Y\!\b{\delta_0}
\end{equation}



On exprime ainsi l'équation aux différences sous la forme pure
d'opérateurs, ou systèmes, qui commutent entre-eux et se distribuent
avec l'addition tout comme une multiplication classique~:

\begin{eqnarray}
  a_n.\oret^n.Y\; +\; \ldots \;+\; a_1.\oret.Y \;+\; a_0.Y \quad  = \quad b_m.\oret^m.X \;+\; \ldots \;+\; b_0.X
\end{eqnarray}

\begin{remarque}
  Dans le cas des systèmes continus, on exprime les équations
  différentielles sous forme opérationnelle en remplaçant l'opérateur
  discret de retard $\oret$ par l'opérateur de dérivation $\oder$. Un
  signal $y$ est de même remplacé par un système $Y$ dont la réponse
  impulsionnelle (à une impulsion de \Dirac{} cette fois-ci) est le
  signal $y$.

  Initialement, \Heaviside{} avait introduit l'échelon unité, ou
  échelon éponyme, comme signal d'excitation de référence à la place
  de l'impulsion de \Dirac{} qui n'était pas encore définie à
  l'époque. Voir le \secref{sec:dirac_derivee} pour une définition de
  l'opérateur réciproque de la dérivée nécessitant l'impulsion de
  \Dirac{}.
\end{remarque}

Nous obtenons avec cette notation une écriture de l'équation aux
différences qui ressemble à une équation algébrique polynomiale
classique. Dans le calcul opérationnel, l'opérateur d'avance
$\oavance$ (resp. $\oret$) est assimilé à un nombre que l'on notera
$z$ (resp. $\zmu$), les signaux $x$ et $y$ sont remplacés par leurs
systèmes générateurs $X$ et $Y$ à partir de leur réponse
impulsionnelle. Les systèmes générateurs $X$ et $Y$ pouvant être
eux-même exprimés en fonction de l'opérateur $z$, ils sont représentés
comme des fonctions de $z$ soit $X\p{z}$ et $Y\p{z}$. Nous verrons
dans la suite que les fonctions $X\p{z}$ et $Y\p{z}$ sont les
transformées en $\TZ$ des signaux (ou systèmes) $x$ et $y$.

Nous obtenons finalement l'équation algébrique associée à la récurrence
\eqref{eq:systeme_recurrence}~:
\begin{eqnarray}
  \label{eq:systeme_algebrique}
  a_n\,z^{-n}\,Y\p{z}\; +\; \ldots \;+\; a_1\,\zmu\;Y\p{z} \;+\; a_0\,Y\p{z} \quad  = \quad b_m\,z^{-m}\,X\p{z} \;+\; \ldots \;+\; b_0\,X\p{z}
\end{eqnarray}

Les opérateurs réciproques $\oret$ et $\oavance$ sont associés aux
nombres $z$ et $\zmu$ car la division et la multiplication sont
réciproques~: comme la composition d'une avance et d'un retard
$\oavance \circ \oret = \Id$ donne le système identité, le produit
algébrique $z\,\zmu=z\,\frac{1}{z}=1$ donne l'unité. L'unité
algébrique $1$ est donc associée au \og{}système identité \fg{} (qui
ne change pas le signal) dont la réponse impulsionnelle est
l'impulsion unité $\delta_0$.

La résolution de l'équation aux différences peut alors se faire en
traitant l'équation algébrique sous forme de fraction rationnelle puis
de décomposition en éléments simples~:
\begin{eqnarray}
  \label{eq:systeme_algebrique}
 \frac{Y\p{z}}{X\p{z}} = \frac{b_m\,z^{-m} + \ldots + b_0}{a_n\,z^{-n}+ \ldots +  a_1\,\zmu + a_0} = \caCest{\frac{\beta_0}{z-\alpha_0}}{\text{premier ordre}} + \ldots + \caCest{\frac{\mu_0+\nu_0\,z}{z^2+b_0\,z+c_0}}{\text{second ordre}} + \ldots
\end{eqnarray}

On décompose alors un système linéaire invariant comme une combinaison
linéaire de systèmes de premier ordre et de second ordre. La
résolution se fait alors par lecture de table de \teZ{} comme pour les
transformées de \Laplace{} dans le cas des systèmes continus.

\begin{exemple}
  \label{exemple:forward_euler}
  Dans l'exemple du \emph{différentiateur}
  \ref{exemple:differentiateur_lineaire} d'écrit par
  $T_e.y = x - \oret\b{x}$, nous pouvons remplacer $x$ et $y$ par les
  systèmes générateurs $X\p{z}$ et $Y\p{z}$ et finalement remplacer la
  composition avec $\oret$ par une multiplication par $\zmu$. On
  obtient la fonction de transfert du système différentiateur~:
  $$H_d\de{z}=\frac{Y\de{z}}{X\de{z}}= \frac{1-\zmu}{T_e}$$

  Il est alors facile de trouver l'opérateur \emph{intégrateur} $H_i$
  réciproque du différentiateur $H_d$ en se basant sur la propriété
  $H_i\circ H_d = H_d \circ H_i = \Id$ qui donne en équation
  algébrique~:
  $$
  H_i\de{z}\,H_d\de{z}=1 \implies H_i\de{z} = \frac{1}{H_d\de{z}} = \frac{Y\de{z}}{X\de{z}}=\frac{T_e}{1-\zmu}
  $$
  On obtient ainsi l'équation de récurrence de l'intégrateur dit \emph{Backward Euler}~:
  \begin{align}
    \label{eq:forward_euler}
    Y\de{z}\p{1-\zmu}=T_e\, X\de{z}  \iff & Y\de{z}=\zmu\,Y\de{z} + T_e\, X\de{z}&\nonumber\\
                                          &  y\b{k} = y\b{k-1} + T_e\, x\b{k} &, \forall k\in\Z \\
    \text{ou bien }   \quad                     &  y\b{k+1} = y\b{k} + T_e\, x\b{k+1} &, \forall k\in\Z \nonumber
  \end{align}
\end{exemple}

\begin{exercice}
  \exerciceTitre{Trois intégrateurs différents et trois différentiateurs associés}

  L'exemple~\ref{exemple:forward_euler} pécédent de l'intégrateur
  \emph{Backward Euler} est illustré ci-dessous avec deux autres
  méthodes. On identifie alors dans \eqref{eq:forward_euler} que
  l'incrément de surface $ds$ ajouté à l'intégrale de $x$ à l'instant
  $k+1$ est la surface du rectangle bleu~:
  $y\b{k+1} = y\b{k} + \caCest{T_e\,x\b{k+1}}{ds}$

  \graphe{0.9\textwidth}{integrales}

  \begin{itemize}
  \item Écrivez alors les récurrences correspondantes aux intégrateurs
    \emph{Forward Euler} et \emph{trapézoïdale} en adaptant la valeur
    de l'incrément de surface $ds$ en fonction de $T_e$, $x\b{k}$
    et/ou $x\b{k+1}$.
  \item De manière inverse à l'exemple précédent, retrouvez les
    fonctions de transfert $H_i\p{z}$ de ces trois intégrateurs
    (remplacer $x\b{k}$  par $X\p{z}$, $x\b{k+1}$ par $z\,X\p{z}$ car
    $z$ est associé à l'avance unitaire).
  \end{itemize}
  On remarque que l'écriture de la récurrence en
  $y\b{k+1}=y\b{k}+\ldots$ donne naturellement une fonction de
  transfert exprimée en $z$, alors que l'écriture en
  $y\b{k}=y\b{k-1}+\ldots$ donne une écriture en $\zmu$ parfaitement
  équivalente~: par exemple pour le \emph{Backward Euler} on obtient
  les fonctions de transfert
  $H_i\p{z}=\frac{T_e}{1-\zmu}=\frac{T_e\,z}{z-1}$

  \begin{itemize}
  \item On peut alors inverser algébriquement ces fonctions de
    transfert d'intégrateur $H_i$ pour obtenir des fonctions de
    transfert de dérivateurs $H_d\p{z}=H_i\p{z}^{-1}$ associées.
  \item On peut, de même, donner les récurrences $y_d\b{k}=\ldots$ à
    partir des fonctions de transfert $H_d\p{z}$ permettant d'obtenir
    différentes approximations de la dérivée du signal d'entrée $x$.
  \end{itemize}

  On obtient ainsi des approximations linéaires discrètes exprimées en
  $z$ (l'avance unitaire) de l'opérateur dérivée en continue $p$ (ou
  variable de \Laplace{} notée $s$)~:
  \begin{equation}
    \label{eq:approx_de_p}
    \oder=\dDtDe{} \leftrightarrow p \leftrightarrow \caCest{\frac{1}{T_e}\p{1-\zmu}}{\text{Forward Euler}}\leftrightarrow \caCest{\frac{1}{T_e}\p{z-1}}{\text{Backward Euler}} \leftrightarrow \caCest{\frac{2}{T_e}\frac{1-\zmu}{1+\zmu}}{\text{Bilinéaire ou Tustin}} 
  \end{equation}  
\end{exercice}

\section{Transformée en \tZ{} de systèmes élémentaires}
Nous allons définir les signaux élémentaires, associés à des systèmes
élémentaires, permettant de constituer une table de transformées en
\tZ{} utile à la résolution d'équations aux différences
\eqref{eq:systeme_recurrence} par la méthode de calcul opérationnel
vue au chapitre précédent.

\subsection{Transformée en \tZ{} d'un signal quelconque}

Soit un signal quelconque $s\b{k}$, il peut être généré par des
retards et des avances unitaires de l'impulsion unité $\delta_0$
d'amplitude $s\b{k}$.

\begin{figure}[ht!]
  \centering
  \graphe{0.8\textwidth}{transformee_en_z}
  \caption{Décomposition d'un signal en combinaison linéaire
    d'opérateurs unitaires $\oret$ appliquée au signal unité
    $\delta_0$. La notation en calcul opérationnel, avec $\zmu$ le
    nombre associé à l'opérateur $\oret$, donne la \teZ.}
  \label{fig:decomposition_unite}
\end{figure}
Le système $S$, composé d'opérateurs avance
$\oavance$ et retard $\oret$ unitaires, générant le signal quelconque
$s$ en réponse à l'impulsion unité est donc le suivant~:
\begin{eqnarray}
  s = \ldots +  s\b{-1} \caCest{\delta_{-1}}{\oavance\b{\delta_0}} + s\b{0} \caCest{\delta_0}{\Id\b{\delta_0}} + s\b{1} \caCest{\delta_{1}}{\oret\b{\delta_0}} +  s\b{2} \delta_{2} + \ldots = \somme{k=-\infty}{\infty}{s\b{k}\delta_{k}}\\
  S\b{\oret} = \ldots + s\b{-1} \caCest{\oavance}{\oret^{-1}} + s\b{0}
  \caCest{\Id}{\oret^0} + s\b{1} \oret + s\b{2} \oret^2 + \ldots =
  \somme{k=-\infty}{\infty}{s\b{k}\oret^{k}}
\end{eqnarray}

On obtient ainsi la définition de la \teZ{} bilatérale en utilisant la
variable opérationnelle $\zmu\leftrightarrow \oret$ associée au retard
unitaire.

\begin{definition} La \emph{\teZ{} bilatérale} d'un signal discret $s$
  quelconque est la fonction holomorphe
\begin{equation}
  \label{eq:transformee_en_z}
  \TZ\b{s} = \TZde{S} : \quad z\mapsto \sum_{k=-\infty}^{+\infty}s\b{k}z^{-k} \quad\quad z \in \domDe{s}= \left\lbrace z\in\mathbb{C} \; \Big| \; \sum_{k=-\infty}^{+\infty}s\b{k}z^{-k} \quad \mathrm{converge}\right\rbrace
\end{equation}
où $\domDe{s}$ est le \emph{domaine de convergence} de la transformée.


La \emph{\teZ{} unilatérale} d'un signal discret $s$
  quelconque est la fonction holomorphe
\begin{equation}
  \label{eq:transformee_en_z_unilaterale}
  \TZ\b{s} = \TZde{S} : \quad z\mapsto \sum_{k=0}^{+\infty}s\b{k}z^{-k} \quad\quad z \in \domDe{s}
\end{equation}
\end{definition}

Tout comme la transformée de \Laplace{} bilatérale, la convergence sur
la branche en $-\infty$ est souvent assurée en considérant un signal
causal dont les termes sont tous nuls avant le rang $k=0$, sans perte
de généralité. Cela revient à utiliser systématiquement l'échelon
unité pour annuler les signaux considérés. L'écriture est alors
facilitée en utilisant la transformée unilatérale considérant par
définition le signal nul aux rangs négatifs.

\begin{remarque}
  La \teZ{} est bien une série entière de terme général
  $ u_n= a_n x^n$ où la suite $a_n$ est le signal $s\b{k}$~; et où
  la variable $x$ est la variable $z$ complexe. Rappelons que le
  domaine de convergence des séries entières possèdent un rayon de
  convergence $R$ pour lequel
  $$
  \begin{array}{ll}
    |z|<R \implies  & \sum a_n z^n \text{ converge} \\
    |z|>R \implies  & \sum a_n z^n \text{ diverge} \\
    |z|=R  & \text{ conclure au cas par cas} \\

  \end{array}
  $$

  On peut retrouver aisément le rayon
  $R$ et cette propriété en utilisant, par exemple, le critère de
  d'Alembert~: si $a_n\neq0$ à partir d'un certain rang et si $
  \lim\limits_{n\to\infty} \left|\frac{a_{n+1}}{a_n}\right|=l \in
  \overline{\R_+}$, alors $R=\frac{1}{l}$.
\end{remarque}

\subsection{Impulsion unité et retard}

L'impulsion unité $\delta_0$ définie dans \defref{def:impulsion_unite}
est utilisée comme signal de référence pour associer un système à un
signal en prenant sa réponse impulsionnelle~: $y=Y\b{\delta_0}$.

Le système dont la réponse impulsionnelle est l'impulsion unité
elle-même est donc l'opérateur identité $\Id$. On a donc pour ce
système $\frac{Y\p{z}}{X\p{z}}=\frac{X\p{z}}{X\p{z}}=1$ puisque la
sortie $y$ du système identité pour une entrée $x$ est le signal
$y=x$.

On a donc par définition~:
\begin{align}
  \label{eq:z_impulsion}
  \TZ\b{\delta_0}=\TZde{\delta_0} : \quad &z \mapsto 1 \nonumber\\
  &\TZde{\delta_0}\p{z} = 1 & \abs{z}<R = \infty
\end{align}

De même le système avance $\oavance$ (resp. retard $\oret$) est
associé par calcul opérationnel au nombre complexe $z$
(resp. $\zmu$). La réponse impulsionnelle du système retard $\oret^m$
est donc l'impulsion retardée de $m$ unitées de temps soit
$\delta_m$. On obtient ainsi la \teZ{} de $\delta_m$~:

\begin{align}
  \label{eq:z_retard}
  \TZ\b{\delta_m}=\TZde{\delta_m}: \quad &z \mapsto z^{-m} &, m\in\Z \nonumber\\
  &\TZde{\delta_m}\p{z} = z^{-m}  &  \abs{z}<R = \infty
\end{align}

\begin{exercice}
  Retrouvez les transformées \eqref{eq:z_impulsion} et \eqref{eq:z_retard} en employant la formule générale \eqref{eq:transformee_en_z}. Vous remarquerez la similitude de rôle de \og{} sélection ou mesure \fg{} de l'impulsion unité sous la somme avec le rôle du \Dirac{} sous l'intégrale vu au \secref{sec:dirac_sous_integrale}.
\end{exercice}
\subsection{Exponentielle complexe et suite géométrique}
\subsection{Echelon unité, rampes et ses intégrales}


%%% Local Variables:
%%% mode: latex
%%% TeX-master: "main"
%%% End:
