\chapter{Systèmes discrets}


L'analogie avec les systèmes continus est forte, nous étudions de même
le cas des sysèmes linéaires invariants dans le temps avec une vision
par opérateurs.


\section{Systèmes linéaires}

\begin{definition}{Système}

Un système discret (resp. continu) relie à chaque signal d'entrée $x$
un signal de sortie unique $y$. Les signaux $x$ et $y$ sont des
fonctions de la variables réelle discrète $k$ (resp. $t$) appartenant
à un espace de fonction le plus général possible noté ici $L_E$. La
relation entrée-sortie est donc modélisée par une application
mathématique de $L_E$ dans $L_E$ notée $L$ (resp. $L_c$ en continu) et
définie ainsi~:
\begin{eqnarray}
  L : \qquad \application{L_E}{L_E}{x : \; k\mapsto x\b{k}\quad{}}{\quad y :\; k\mapsto y\b{k}} \\
  L_c: \qquad \application{L_E}{L_E}{x : \; t\mapsto x\p{t}\quad{}}{\quad y :\; t\mapsto y\p{t}} 
    % DONE : correction TYPAGE et remarque notation
\end{eqnarray}
\end{definition}

Une classe de système fondamentale est la classe des systèmes
linéaires car elle offre de nombreux outils et propriétés
mathématiques.

\begin{definition}{Système linéaire}
  \label{def:linearite}
  
  Une système est dit linéaire si et seulement si
  l'application $L$ associée est linéaire, soit pour tout
  $\p{x_1,x_2,\lambda} \in L_E^2 \times \R$~:
  \begin{eqnarray}
    \label{eq:def_linearite}
    \forall t \in \R \qquad L\b{x_1 + \lambda\, x_2}(t) = L\b{x_1}(t) + \lambda\,L\b{y_2}(t) \nonumber\\
    \text{ou bien } \qquad L\b{x_1 + \lambda\,x_2} = L\b{x_1} +\lambda L\b{x_2 }
  \end{eqnarray}
\end{definition}

Une des conséquences de la linéarité est la possibilité d'appliquer le
principe de superposition cher à l'électronicienne~: la réponse du
système à une entrée est la combinaison linéaire des réponses à chaque
signaux composant l'entrée par combinaison la même combinaison
linéaire.
	
Les trois systèmes linéaires de base que nous considérons dans
l'étude des systèmes linéaires continus sont :
\begin{description}
\item[le gain $a.x$]~: $t \mapsto a\,x\p{t}$ où a est une constante
\item[le dérivateur $\oder\b{x}=\oder\circ x$]~: $ t \mapsto \dDtDe{x}\p{t} $
\item[l'intégrateur $\oint\b{x}=\oint\circ x$]~: $ t \mapsto \integ{0}{t}{x\p{\nu}\derivDe{\nu}}$ 
\end{description}
On peut aisément vérifier que ces systèmes respectent la condition de
linéarité~\ref{eq:def_linearite}. On aimerait que les opérateurs
dérivateur $\oder$ et intégrale $\oint$ commutent et soient
réciproque~: $\oder\circ\oint=\oint\circ\oder=\Id$. Pour que cela soit
vrai même avec les fonctions discontinues, il faut introduire les
distributions de \Dirac{} voir \chapref{sec:deriv_discontinues}.

Dans le cas des systèmes discret, les systèmes élémentaires sont :
\begin{description}
\item[le gain $a.x$]~: $ k \mapsto a\,x\b{k}$ où a est une constante.
\item[le retard unité $\oret\b{x}=\oret\circ x$]~: $ k \mapsto x\b{k-1}$.
\item[l'avance unité $\oavance\b{x}=\oavance\circ x$]~: $ k \mapsto x\b{k+1} $ 
\end{description}

On peut aisément vérifier que ces systèmes respectent la condition de
linéarité~\ref{eq:def_linearite}. La commutation et réciprocité des
opérateurs retard $\oret$ et avance $\oavance$ est évidente et ne pose
pas de problème théorique.

\begin{exemple}
  \label{exemple:differentiateur_lineaire}
  L'effet du système \emph{différentiateur} sur le signal d'entrée est d'écrit par
  l'opérateur ~: $$L : x \mapsto y=\frac{x-\oret\b{x}}{T_e}$$

  Vérifions d'abord que cet opérateur est linéaire~:
  \begin{eqnarray*}
    L\b{x_1+\lambda x_2}\b{k} &= L\b{x_1+\lambda x_2}\b{k}\\
                         &= \frac{x_1\b{k}+\lambda x_2\b{k} - \oret\b{x_1+\lambda x_2}\b{k}}{T_e}\\
                         &= \frac{x_1\b{k}+\lambda x_2\b{k} - \p{x_1\b{k-1}+\lambda x_2\b{k-1}}}{T_e} = A\\
   L\b{x_{1}}\b{k}+\lambda\,L\b{x_{2}}\b{k}  &= \frac{x_1\b{k} - x_1\b{k-1}}{T_e}+\lambda\,\frac{x_2\b{k} - x_2\b{k-1}}{T_e}=A
  \end{eqnarray*}
  Le système est donc linéaire.
\end{exemple}


% \begin{remarque}
%   Pour résoudre les complexes équations différentielles des
%   télégraphistes, \Heaviside{} utilise ces opérateurs de base et
%   introduit le \emph{calcul opérationnel}. Cela consiste à représenter
%   l'application de l'opérateur dérivée sur une fonction $f$ comme une
%   simple multiplication par un nombre $p$. Ainsi une équation
%   différentielle $a\,y'' + 2y' -y = 3\int x$ est associée à l'équation
%   symbolique $a\,p^2\,y + 2\,p\,y - y = \frac{3}{p}x$. Il est alors
%   possible de résoudre algébriquement l'équation sous forme de
%   fractions rationnelles ce qui donnerait avec notre exemple~:
%   $y = \frac{3/p}{a\,p^2+2p+1} x$. Rappelons que $x$ et $y$ sont des
%   fonctions et non des réels et que dans ce cas les opérations ne sont
%   pas de simple multiplication et addition de réels mais bien des
%   multiplications et additions de fonctions. La variable symbolique
%   $p$ est utilisée comme un nombre réel mais n'est en aucun cas un
%   réel...
% \end{remarque}
	
\section{Systèmes invariants}
Il est fréquent, et surtout théoriquement utile, qu'un système
réagisse de la même manière indépendemment de l'instant où est
appliqué le signal d'entrée. Ce qui conduit à la définition suivante~:
\begin{definition}{Système invariant dans le temps}
  Un système discrète (resp. continu) est dit invariant dans le temps si et seulement
  si son application associée $L$ (resp. $L_c$) vérifie~:
  \begin{eqnarray}
    \forall x\in L_E, \forall (k,k_0)\in \N^2, \quads L[k\mapsto x(k-k_{0})] = L[x](k-k_{0}) \\
    \forall x\in L_E, \forall (t,t_0)\in \R^2, \quads L_c[t\mapsto x(t-t_{0})] = L[x](t-t_{0}) 
  \end{eqnarray}
  
  En terme d'opérateur~;  un système $L$ est invariant dans
  le temps si, et seulement si, son opérateur commute avec  tout opérateur
  retard de $k_0$ noté $\oretDe{k_0}=\oret^{k_0}\b{x}= k\mapsto x\b{k-k_0}$~:
  \begin{eqnarray}
    \label{eq:sys_invariant}
    L\circ\oretDe{k_0} \; = \; \oretDe{k_0}\circ L  \qquad \iff \qquad  L\b{\oretDe{k_0}\b{x}} = \oretDe{k_0}\b{L\b{x}}
    \\
    L_c\circ\oretDe{\tau} \; = \; \oretDe{\tau}\circ L_c  \qquad \iff \qquad  L_c\b{\oretDe{\tau}\b{x}} = \oretDe{\tau}\b{L_c\b{x}}
  \end{eqnarray}
\end{definition}

Autrement dit \og{} la réponse du système à un signal retardé est le
retard de la réponse du système.\fg{} 

En d'autres termes, la réponse du système ne dépend pas de l'origine
des temps choisie.
        
\begin{exemple}
  Reprenons l'exemple du \emph{différentiateur} \ref{exemple:differentiateur_lineaire} d'écrit par
  l'opérateur ~: $$L : x \mapsto y=\frac{x-\oret\b{x}}{T_e}$$
  
  Vérifions qu'il est invariant~:
  \begin{eqnarray*}
    \forall k_0 \quads L\b{\oretDe{k_0}\b{x}}\b{k} &= L\b{k\mapsto x\b{k-k_0}}\b{k}\\
                                                   &= \frac{x\b{k-k_0}-x\b{k-k_0-1}}{T_e} = B\\
    \forall k_0 \quads \oretDe{k_0}\b{L\b{x}}\b{k} &= \oretDe{k_0}\b{k\mapsto \frac{x\b{k-1}-x\b{k}}{T_e}}\b{k}\\
                                                   &= \frac{x\b{k-k_0}-x\b{k-k_0-1}}{T_e} = B
  \end{eqnarray*}
  
  On a bien exprimé le fait que $L$ commute avec tout retard
  $\oretDe{k_0}$, en d'autres termes~: \emph{la différence du retard est le retard de la différence.}
\end{exemple}


\begin{remarque}
  Il est facile de vérifier que les systèmes discrets élémentaires que
  sont le gain~; le retard unitaire et l'avance unitaire (gain,
  dérivateur et intégrateur pour le continu) sont invariants.  Il en
  est de même pour tout système constitué de combinaisons linéaires
  et de composition de systèmes élémentaires.

  Il suffit alors de montrer que le système se décompose avec des
  \emph{coefficients constants} avec des systèmes élémentaires en le
  mettant sous forme \emph{d'équation aux différence} ou
  \emph{récurrence} à coefficients constants (\emph{equations
    différentielle} en continu) ou en un schéma bloc à coefficients
  constant.
\end{remarque}



%%% Local Variables:
%%% mode: latexqn
%%% TeX-master: "main"
%%% End:
