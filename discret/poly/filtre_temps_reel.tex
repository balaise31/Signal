Cette propriété est souvent liée à la propriété de \emph{causalité}
d'un système~:
\begin{definition}{Système causal}
  
  Un système $L$ (resp. $L_c$ en continu) est dit causal en $k_0$
  (resp. en $t_0$ pour le continus) si toute réponse à un signal nul
  avant l'instant $k_0$ (resp. $t_0$) est nulle aux instants $k<k_0$
  (resp. $t<t_0$).

  Dans le cas de systèmes invariants, cette définition peut s'écrire
  en prenant arbitrairement $0$ comme instant de référence~:

  \begin{align}
    \label{eq:systeme_causal}    
    \forall x\in L_e, \forall k\in\N^*, \quad &x\b{-k}=0 \quad \implies \quad L\b{x}\b{-k}=y\b{-k}=0 \\
    \forall x\in L_c, \forall t\in\R^{+*}, \quad &x\p{-t}=0 \quad \implies \quad L_c\b{x}\p{-t}=y\p{-t}=0   
  \end{align}
\end{definition}

Autrement dit \og{}l'effet ne précède pas la cause.\fg{}
\begin{exemple}
  Reprenons l'exemple du \emph{différentiateur} \ref{exemple:differentiateur_lineaire} d'écrit par l'opérateur ~: $$L : x \mapsto y=\frac{x-\oret\b{x}}{T_e}$$
  
  Vérifions qu'il est invariant~:
  \begin{eqnarray*}
    \forall k_0 \quads L\b{\oretDe{k_0}\b{x}}\b{k} &= L\b{k\mapsto x\b{k-k_0}}\b{k}\\
                                                   &= \frac{x\b{k-k_0}-x\b{k-k_0-1}}{T_e} = B\\
    \forall k_0 \quads \oretDe{k_0}\b{L\b{x}}\b{k} &= \oretDe{k_0}\b{k\mapsto \frac{x\b{k}-x\b{k-1}}{T_e}}\b{k}\\
                                                   &= \frac{x\b{k-k_0}-x\b{k-k_0-1}}{T_e} = B
  \end{eqnarray*}
  
  On a bien exprimé le fait que $L$ commute avec tout retard
  $\oretDe{k_0}$, en d'autres termes~: \emph{la différence du retard est le retard de la différence.}
\end{exemple}


