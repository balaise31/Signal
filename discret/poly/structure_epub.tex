
\usepackage{graphicx} % Required for including pictures
\graphicspath{{./poly/Pictures/}} % Specifies the directory where pictures are stored


\usepackage{tikz} % Required for drawing custom shapes




%----------------------------------------------------------------------------------------
%	BIBLIOGRAPHY AND INDEX
%----------------------------------------------------------------------------------------
\usepackage{csquotes}
\usepackage[style=numeric,citestyle=numeric,sorting=nyt,sortcites=true,autopunct=true,autolang=hyphen,hyperref=true,abbreviate=false,backref=true,backend=biber]{biblatex}
\addbibresource{bibliography.bib} % BibTeX bibliography file
\defbibheading{bibempty}{}


%----------------------------------------------------------------------------------------
%	MAIN TABLE OF CONTENTS
%----------------------------------------------------------------------------------------

\usepackage{titletoc} % Required for manipulating the table of contents

%----------------------------------------------------------------------------------------
%	MINI TABLE OF CONTENTS IN PART HEADS
%----------------------------------------------------------------------------------------


%----------------------------------------------------------------------------------------
%	THEOREM STYLES
%----------------------------------------------------------------------------------------

\usepackage{amsmath,amsfonts,amssymb,amsthm} % For math equations, theorems, symbols, etc

\newcommand{\intoo}[2]{\mathopen{]}#1\,;#2\mathclose{[}}
\newcommand{\ud}{\mathop{\mathrm{{}d}}\mathopen{}}
\newcommand{\intff}[2]{\mathopen{[}#1\,;#2\mathclose{]}}
\renewcommand{\qedsymbol}{$\blacksquare$}
\newtheorem{notation}{Notation}[chapter]


% Defines the theorem text style for each type of theorem to one of the three styles above
\newcounter{dummy} 
\numberwithin{dummy}{section}
\newtheorem{theoremeT}[dummy]{Theorème}
\newtheorem{probleme}{Problème}[chapter]
\newtheorem{exercisseT}{Exercisse}[chapter]
\newtheorem{exempleT}{Exemple}[chapter]
\newtheorem{vocabulary}{Vocabulary}[chapter]
\newtheorem{definitionT}{Definition}[section]
\newtheorem{corollaryT}[dummy]{Corollary}
\newtheorem{proposition}[dummy]{Proposition}

% Creates an environment for each type of theorem and assigns it a theorem text style from the "Theorem Styles" section above and a colored box from above
\newenvironment{theoreme}{\begin{tBox}\begin{theoremeT}}{\end{theoremeT}\end{tBox}}
\newenvironment{exercise}{\begin{eBox}\begin{exerciseT}}{\hfill{\color{ocre}\tiny\ensuremath{\blacksquare}}\end{exerciseT}\end{eBox}}				  
\newenvironment{definition}{}{}	
\newenvironment{exemple}{\begin{exempleT}}{\hfill{\tiny\ensuremath{\blacksquare}}\end{exempleT}}		
\newenvironment{corollary}{\begin{cBox}\begin{corollaryT}}{\end{corollaryT}\end{cBox}}	

%----------------------------------------------------------------------------------------
%	REMARK ENVIRONMENT
%----------------------------------------------------------------------------------------

\newenvironment{remarque}{\par\vspace{10pt}\small % Vertical white space above the remark and smaller font size
}

%----------------------------------------------------------------------------------------
%	LINKS
%----------------------------------------------------------------------------------------

\usepackage{hyperref}
\hypersetup{hidelinks,colorlinks=false,breaklinks=true,urlcolor=ocre,bookmarksopen=false}


%%% Local Variables:
%%% mode: latex
%%% TeX-master: "main"
%%% End:
