
\subsection{Transformée en Z de systèmes élémentaires}
\label{sec:transformee_en_z}
Nous allons définir les signaux élémentaires, associés à des systèmes
élémentaires, permettant de constituer une table de transformées en
\tZ{} utile à la résolution d'équations aux différences
\eqref{eq:systeme_recurrence} par la méthode de calcul opérationnel
vue au chapitre précédent.

\subsubsection{Transformée en \tZ{} d'un signal quelconque}

Soit un signal quelconque $s\b{k}$, il peut être généré par des
retards et des avances unitaires de l'impulsion unité $\delta_0$
d'amplitude $s\b{k}$.

\begin{figure}[ht!]
  \centering
  \graphe{\textwidth}{transformee_en_z}
  \caption{Décomposition d'un signal en combinaison linéaire
    d'opérateurs unitaires $\oret$ appliquée au signal unité
    $\delta_0$. La notation en calcul opérationnel, avec $\zmu$ le
    nombre associé à l'opérateur $\oret$, donne la \teZ.}
  \label{fig:decomposition_unite}
\end{figure}
Le système $S$, composé d'opérateurs avance
$\oavance$ et retard $\oret$ unitaires, générant le signal quelconque
$s$ en réponse à l'impulsion unité est donc le suivant~:
\begin{equation}
  \label{eq:decomposition_unite}
  \begin{array}{cll}
    \caCest{s\b{k}}{\text{scalaire}\in\C} &= \ldots +  s\b{-1} \caCest{\delta_0\b{k+1}}{\delta_{-1}\b{k}} + s\b{0} \delta_0\b{k} + s\b{1} \caCest{\delta_0\b{k-1}}{\delta_1\b{k}} + \ldots &= \somme{j=-\infty}{\infty}{\caCest{s\b{j}}{\in\C}.\caCest{\delta_0\b{k-j}}{\in\C}} \\
    \caCest{s}{\text{signal}} &= \ldots +  s\b{-1} \caCest{\delta_{-1}}{\oavance\b{\delta_0}} + s\b{0} \caCest{\delta_0}{\Id\b{\delta_0}} + s\b{1} \caCest{\delta_{1}}{\oret\b{\delta_0}} +  \ldots &= \somme{j=-\infty}{\infty}{\caCest{s\b{j}}{\in\C}.\caCest{\delta_{j}}{\text{signal de k}}} \\
    \caCest{S\p{\oret}}{\text{opérateur}} &= \ldots + s\b{-1} \caCest{\oavance}{\oret^{-1}} + s\b{0}
                                            \caCest{\Id}{\oret^0} + s\b{1} \oret + \ldots &= \somme{j=-\infty}{\infty}{\caCest{s\b{j}}{\in\C}\caCest{\oret^{j}}{\text{opérateur}}}\\
    \caCest{\TZlui{S}\p{z}}{\text{scalaire}\in\C} &= \ldots + s\b{-1} \caCest{z}{\text{avance}} + s\b{0}
    \caCest{1}{z^0} + s\b{1} \caCest{\zmu}{\text{retard}}+ \ldots &= \somme{j=-\infty}{\infty}{\caCest{s\b{j}}{\in\C}\caCest{z^{-j}}{\in\C}}   
  \end{array}
\end{equation}

On obtient ainsi la définition de la \teZ{} bilatérale en utilisant la
variable opérationnelle $\zmu\leftrightarrow \oret$ associée au retard
unitaire.

\begin{definition} La \emph{\teZ{} bilatérale} d'un signal discret $s$
  quelconque est la fonction holomorphe
\begin{equation}
  \label{eq:transformee_en_z}
  \TZ\b{s} = \TZlui{S} : \quad z\mapsto \sum_{k=-\infty}^{+\infty}s\b{k}z^{-k} \quad\quad z \in \domDe{s}= \left\lbrace z\in\mathbb{C} \; \Big| \; \sum_{k=-\infty}^{+\infty}s\b{k}z^{-k} \quad \mathrm{converge}\right\rbrace
\end{equation}
où $\domDe{s}$ est le \emph{domaine de convergence} de la transformée.


La \emph{\teZ{} unilatérale} d'un signal discret $s$
  quelconque est la fonction holomorphe
\begin{equation}
  \label{eq:transformee_en_z_unilaterale}
  \TZ\b{s} = \TZlui{S} : \quad z\mapsto \sum_{k=0}^{+\infty}s\b{k}z^{-k} \quad\quad z \in \domDe{s}
\end{equation}
\end{definition}

Tout comme la transformée de \Laplace{} bilatérale, la convergence sur
la branche en $-\infty$ est souvent assurée en considérant un signal
causal dont les termes sont tous nuls avant le rang $k=0$, sans perte
de généralité. Cela revient à utiliser systématiquement l'échelon
unité pour annuler les signaux considérés. L'écriture est alors
facilitée en utilisant la transformée unilatérale considérant par
définition le signal nul aux rangs négatifs.

\begin{remarque}
  La \teZ{} est bien une série entière de terme général
  $ u_n= a_n x^n$ où la suite $a_n$ est le signal $s\b{k}$~; et où
  la variable $x$ est la variable $z$ complexe. Rappelons que le
  domaine de convergence des séries entières possèdent un rayon de
  convergence $R$ pour lequel
  $$
  \begin{array}{ll}
    |z|<R \implies  & \sum a_n z^n \text{ converge} \\
    |z|>R \implies  & \sum a_n z^n \text{ diverge} \\
    |z|=R  & \text{ conclure au cas par cas} \\

  \end{array}
  $$

  On peut retrouver aisément le rayon
  $R$ et cette propriété en utilisant, par exemple, le critère de
  d'Alembert~: si $a_n\neq0$ à partir d'un certain rang et si $
  \lim\limits_{n\to\infty} \left|\frac{a_{n+1}}{a_n}\right|=l \in
  \overline{\R_+}$, alors $R=\frac{1}{l}$.
\end{remarque}

\subsubsection{Impulsion unité et retard}

L'impulsion unité $\delta_0$ définie dans \defref{def:impulsion_unite}
est utilisée comme signal de référence pour associer un système à un
signal en prenant sa réponse impulsionnelle~: $y=Y\b{\delta_0}$.

Le système dont la réponse impulsionnelle est l'impulsion unité
elle-même est donc l'opérateur identité $\Id$. On a donc pour ce
système $\frac{Y\p{z}}{X\p{z}}=\frac{X\p{z}}{X\p{z}}=1$ puisque la
sortie $y$ du système identité pour une entrée $x$ est le signal
$y=x$.

On a donc par définition~:
\begin{align}
  \label{eq:z_impulsion}
  \TZ\b{\delta_0}=\TZlui{\delta_0} : \quad &z \mapsto 1 \nonumber\\
  &\TZlui{\delta_0}\p{z} = 1 & \abs{z}<R = \infty
\end{align}

De même le système avance $\oavance$ (resp. retard $\oret$) est
associé par calcul opérationnel au nombre complexe $z$
(resp. $\zmu$). La réponse impulsionnelle du système retard $\oret^m$
est donc l'impulsion retardée de $m$ unitées de temps soit
$\delta_m$. On obtient ainsi la \teZ{} de $\delta_m$~:

\begin{align}
  \label{eq:z_retard}
  \TZ\b{\delta_m}=\TZlui{\delta_m}: \quad &z \mapsto z^{-m} &, m\in\Z \nonumber\\
  &\TZlui{\delta_m}\p{z} = z^{-m}  &  \abs{z}<R = \infty
\end{align}

\begin{exercice}
  Retrouvez les transformées \eqref{eq:z_impulsion} et \eqref{eq:z_retard} en employant la formule générale \eqref{eq:transformee_en_z}. Vous remarquerez la similitude de rôle de \og{} sélection ou mesure \fg{} de l'impulsion unité sous la somme avec le rôle du \Dirac{} sous l'intégrale vu au \secref{sec:dirac_sous_integrale}.
\end{exercice}
\subsubsection{Exponentielle complexe et suite géométrique}
\subsubsection{Echelon unité, rampes et ses intégrales}
